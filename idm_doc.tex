%\documentclass[11pt,english,ngerman,pointlessnubers, abstraction, headsepline,liststotoc]{scrreprt}
%\documentclass[11pt]{report}
\documentclass[11pt,ngerman,titlepage,oneside, headsepline,listof=totoc]{scrreprt}
\usepackage[a4paper]{geometry}
%\geometry{verbose, tmargin=3cm, bmargin=3cm, lmargin=3.25cm, rmargin=2.5cm, headheight=1cm, headsep=0.666cm, footskip=1cm}
\usepackage[T1]{fontenc}
\usepackage[german]{babel}
\selectlanguage{german}
\usepackage[utf8x]{inputenc}

\usepackage{lmodern}
\renewcommand{\sfdefault}{lmss}
\renewcommand{\ttdefault}{lmtt}
%\usepackage{booktabs}
\usepackage{amsmath}
%\usepackage{syntax}
\usepackage{amsfonts}
\usepackage{amssymb}
\usepackage{amsthm}
\usepackage{graphicx}
\usepackage{empheq}
\usepackage{url}
\usepackage{footmisc}
\usepackage{tabularx}
\usepackage[table]{xcolor}
%\usepackage{relsize}
\usepackage{setspace}
%\setstretch{1.4}
\usepackage{caption}
\usepackage{xcolor}
\usepackage{listings}
\usepackage{nameref}
\usepackage{varwidth}
\usepackage{enumitem}
\usepackage{tikz}
\usetikzlibrary{positioning,automata,shapes}
\usetikzlibrary{positioning}
\usetikzlibrary{decorations.text}
\usetikzlibrary{decorations.pathmorphing,backgrounds,calc}
\usetikzlibrary{decorations.pathreplacing,angles,quotes}
\usetikzlibrary{shapes,arrows,external,shapes.callouts}
	\tikzstyle{element} = [rectangle, font=\fontsize{0.75em},font=\footnotesize, draw, text centered, minimum height=8mm, minimum width=15mm]
	\definecolor{light-gray}{gray}{0.98}
	\definecolor{CPgray}{gray}{0.96}
	\colorlet{macBorder}{black!25}
	\colorlet{dataBorder}{black!50}
	%\definecolor{macBorder}{RGB}{53,125,145}
	%\definecolor{dataBorder}{RGB}{133,137,63}
	\tikzstyle{macElement} = [rectangle, font=\footnotesize, draw, text centered, minimum height=12mm, minimum width=30mm,text=black]
	\tikzstyle{macElementField} =[macElement, color=macBorder,  text centered, text width=27mm, line width=2pt,text=black]
	\tikzstyle{macElementDeteil} =[macElement, color=macBorder, text centered, text width=25mm, minimum height=12mm,  line width=1pt,text=black]
	\tikzstyle{dataElementField} =[macElement, color=dataBorder,  text centered, text width=27mm, line width=2pt,text=black]
	\tikzstyle{dataElementDeteil} =[macElement, color=dataBorder, text centered, text width=25mm, minimum height=12mm, line width=1pt,text=black]
	
	
\usepackage{caption}
\usepackage{subcaption}
\usepackage{float}
\usepackage{wrapfig}
\usepackage{multirow}
\usepackage{ragged2e}
\usepackage{array}
\usepackage{longtable}
\usepackage[export]{adjustbox} %used for image alignment on title page
%\floatstyle{boxed}
%\restylefloat{figure}

\lstdefinestyle{BashInputStyle}{
	language=sh,
	basicstyle=\small\sffamily,
	numbers=left,
	numberstyle=\tiny,
	breaklines=true,
	numbersep=3pt,
	frame=tb,
	columns=fullflexible,
	backgroundcolor=\color{yellow!20},
	linewidth=0.9\linewidth,
	xleftmargin=0.1\linewidth
}
\colorlet{punct}{red!60!black}
\definecolor{background}{HTML}{EEEEEE}
\definecolor{delim}{RGB}{20,105,176}
\colorlet{numb}{magenta!60!black}

\lstdefinelanguage{json}{
    basicstyle=\normalfont\ttfamily,
    numbers=left,
    numberstyle=\scriptsize,
    stepnumber=1,
    numbersep=8pt,
    showstringspaces=false,
    breaklines=true,
    frame=lines,
    backgroundcolor=\color{background},
    literate=
     *{0}{{{\color{numb}0}}}{1}
      {1}{{{\color{numb}1}}}{1}
      {2}{{{\color{numb}2}}}{1}
      {3}{{{\color{numb}3}}}{1}
      {4}{{{\color{numb}4}}}{1}
      {5}{{{\color{numb}5}}}{1}
      {6}{{{\color{numb}6}}}{1}
      {7}{{{\color{numb}7}}}{1}
      {8}{{{\color{numb}8}}}{1}
      {9}{{{\color{numb}9}}}{1}
      {:}{{{\color{punct}{:}}}}{1}
      {,}{{{\color{punct}{,}}}}{1}
      {\{}{{{\color{delim}{\{}}}}{1}
      {\}}{{{\color{delim}{\}}}}}{1}
      {[}{{{\color{delim}{[}}}}{1}
      {]}{{{\color{delim}{]}}}}{1},
}

 \usepackage[nounderscore]{syntax}
 \newenvironment{BNF}{\captionsetup{type=lstlisting}}{}
 
\usepackage{acronym}
\setcounter{secnumdepth}{3}
\setcounter{tocdepth}{3}
%\setcounter{tocdepth}{5}
%\setcounter{secnumdepth}{4}

%\usepackage{titlesec}

\usepackage[Algorithmus]{algorithm}
\usepackage[noend]{algpseudocode}
\usepackage[titletoc,title]{appendix}
\usepackage[nottoc]{tocbibind}

\newcommand{\reftablle}[1]{\textit{Tabelle \ref{#1}, \nameref{#1}, }}
\newcommand{\reftablleb}[1]{\textit{Tabelle \ref{#1}, \nameref{#1}}}
\newcommand{\reftabllec}[1]{\textit{Tabelle \ref{#1}}}
\newcommand{\refabb}[1]{\textit{Abbildung \ref{#1}}}
\newcommand{\refabschnitt}[1]{\textit{Abschnitt \ref{#1}: \nameref{#1}}}
\newcommand{\refabschnittb}[1]{\textit{Abschnitt \ref{#1}}}
\newcommand{\refunterabschnitt}[1]{\textit{Unterabschnitt \ref{#1}}}
\newcommand{\refuntuntererabschnitt}[1]{\textit{Unterunterabschnitt \ref{#1}}}
\newcommand{\reflisting}[1]{\textit{Listing \ref{#1}}}
\newcommand{\refalgo}[1]{\textit{Algorithmus \ref{#1}}}
\newcommand{\refanhang}[1]{\textit{Anhang \ref{#1}}}
\renewcommand{\listalgorithmname}{Algorithmenverzeichnis}


\begin{document}
% Hintergründe, Grundlagen, Unterteilung des Dokuments in Abschnitte (Client/Server, UART, Funk, LC)
% Client/Server Struktur
% Python-Skript und UART Kommunikation
% Funkkommunikation und LCs
\begingroup
  %\renewcommand*{\chapterpagestyle}{empty}
  \pagestyle{plain}
	\setcounter{page}{1}
	\pagenumbering{roman}
  \tableofcontents
  \clearpage
\endgroup
\pagenumbering{arabic}
\pagestyle{plain}
\setcounter{page}{1}
\chapter{Einleitung}
Ziel dieses Dokument ist es, eine REST-API und die darauf verwendeten Datenobjekte zu spezifieren bzw. zu modellieren, um Daten aus einem IdM-System beziehen zu können.\\
\\
Der IdM-Provider ist verpflichtet, mindestens die geforderten Schnittstellen und Protokolle mit der angegebenen Verfügbarkeitsanforderung bereitzustellen. 
Im Gegenzug erhält der IdM-Provider über viele Objekte die Hoheit der zentralen ID-Vergabe. 
Eine zentralisierte ID-Vergabe ist notwendig, damit verschiedene nachgelagerte Systeme (z.B. Schulverwaltungssoftwares, Lernplattformen) interoperabel sind.\\
\\
Es werden die notwendigen Endpunkte definiert, die jeweils zugelassenen Operationen nebst verwendeter HTTP-Methode festgelegt, die Datenobjekte in JSON modelliert sowie der Workflow darauf dargestellt.

\section{Benutzerrollen}
Im Nachfolgenden werden die Benutzerrollen des Systems aufgelistet. 
Später werden für jede Benutzerrolle erlaubte und unerlaubte Aktionen festgelegt.


\begin{table}[htb]
	\begin{tabularx}{\textwidth}{|c|X|}
		\hline
\textbf{Rollenname} & \textbf{Beschreibung der Rolle} \\ \hline
guest & Gäste und nicht authentifizierte Benutzer \\ \hline
user & authentifizierte Benutzer \\ \hline
studenst & Schüler \\ \hline
guardians & Eltern, Erziehungsberechtigte, Vormünder von Schülern \\ \hline
teacher & Lehre \\ \hline
principal & Schulleitung \\ \hline
school-admin & Schule Administrator \\ \hline
school-board & Schulträger \\ \hline
fed-school-board & Mittarbeiter/Mitarbeiterin des Schulministerums \\ \hline
sync-systems & Systeme welchen ein Sync mit allen Daten erlaubt ist \\ \hline

	\end{tabularx}

		\caption{Benutzerrollen die im Zentralen IDM vorgesehen sind}
		\label{tab:intro:roles}
\end{table}
\chapter{Authentifizierung und Autorisierung}
Um den Zugriff auf Zielsystem und REST-API gleichermaßen zu steuern und dabei im Zielsystem keine Anmeldeinformationen vorhalten zu müssen, wird eine Authentifizierung und Autorisierung per OAuth2-Protokoll und OpenID Connect vorgeschrieben.\\
\\
Der Ablauf der Autorisierung per OAuth2-Protokoll im Authorization-Code-Flow ist in der folgenden Grafik dargestellt. 
\\
to be inserted\\
\\
Über den Access Token erhält der Benutzer schließlich Zugriff auf das Zielsystem bzw. auf die REST-API, über den ID Token auf die ID (uuid) und ggf. weitere personenbezogene Daten des Benutzers. 
Sofern die personenbezogenen Daten für die Anlage eines Benutzerkontos (z.B. Vor- und Nachname, E-Mail-Adresse) benötigt werden, können diese durch die geeignete Wahl von scopes und claims im Aufruf der Autorisierungs-URL vom Resource Server abgefragt werden und initial zur Benutzerkontoerstellung genutzt werden.\\
\\
Der IDM-Provider muss dabei den Autorisierungsserver (OpenID Provider) und Resource Server zur Verfügung stellen, zudem müssen folgende Endpunkte definiert sein:\\

\begin{table}[htb]
    \begin{tabularx}{\textwidth}{|c|X|}
        \hline
\textbf{Endpunkt} & \textbf{Funktion des Endpunkts} \\ \hline
Autorisierung & Initiierung der Autorisierung und Benutzerzustimmung mit definierten Parametern (u.a. scopes/claims und Redirect-URL) \\ \hline
Token & Liefert ID und Access Token zurück \\ \hline
Revocation & Endpunkt, um den Access Token zu widerrufen \\ \hline
UserInfo & Kann definierte Benutzerdaten zur Verfügung stellen (z.B. uuid, Vor- und Nachname, E-Mail-Adresse) \\ \hline
    \end{tabularx}

        \caption{Endpunkte, die durch den IDM-Provider für die Anmeldung per OAuth2 in Verbindung mit OpenID Connect zur Verfügung gestellt werden müssen}
        \label{tab:auth:endpoints}
\end{table}
\\

Die Spezifikation der REST-API sieht vor, dass die Sichtbarkeit der Daten an den definierten Endpunkte bereits im JSON-Objekt gemäß der Berechtigung des anfragenden Benutzers berücksichtigt ist. 
Die ID des Benutzers im Access Token allein reicht nicht aus, um die Sichtbarkeit der Daten korrekt zu beschränken, da einem Benutzer im IDM mehrere Institut-Rolle-Kombinationen zugewiesen sein können (z.B. an verschiedenen Institutionen als Lehrkraft tätig oder an derselben Institution sowohl Lehrkraft als auch Elternteil eines Schülers). 
Daher muss die Information, mit welcher Kombination von Institution und Rolle innerhalb dieser Institution der Aufruf eines REST-API-Endpunkts durchgeführt wird, an den IDM-Provider übermittelt werden. 
Dies geschieht durch die Erweiterung der Scopes-Liste (um jeweils den Scope für die Institution und die Rolle) als Parameterübergabe beim Autorisierungsvorgang. 
Die zusätzlichen Scopes für Institution und Rolle sind somit im Access Token hinterlegt und können bei der Generierung des zurückzuliefernden JSON-Objekts verwendet werden. \\
\\
Ein Sonderfall stellen dabei Benutzerkonten mit der Rolle ''sync-system'' dar. 
Diese haben bei Aufruf eines REST-API-Endpunkts grundsätzlich keine Beschränkung auf eine einzelne Institution. 
Um die Tokengenerierung zu vereinheitlichen, wird für Benutzerkonten mit der Rolle ''sync-system'' statt des Institutionsnamens die Bezeichnung der IDM-Domäne verwendet.\\
\\
Beispiel:\\

\begin{lstlisting}
https://<URl zu Autorisierungsendpunkt>?response_type=code
&client_id=<Identifier von Client-App>
&redirect_uri=<Redirect-URL>
&scope=openid sync-system idm-domain
&state=<Undurchschaubarer Wert fuer Sicherheitszwecke>
\end{lstlisting}
\\
Sofern für ein Benutzerkonto im IDM mehrere Institution-Rolle-Kombinationen hinterlegt sein können, müssen Institution und Rolle bei Start des Authentifizierungsprozesses gegebenenfalls durch den Benutzer wählbar sein. 
Für einen Instituiton-Rolle-Wechsel ist eine Neuanmeldung bzw. Aktualisierung des Access Tokens bezüglich der Scopes für Institution und Rolle notwendig. \\
\\
\\
Für den Logout bzw. Aktualisierungen aus dem IDM kann der Access Token am Autorisierungsserver widerrufen werfen (Token Revocation.)
\section{REST API Definition}
\label{sec:RESTAPIDefinition}

\section{Schnittstellen für Schulfächer}
Die Schnittstelle für Schulfächer listet die Teilmenge der abgestimmten Referenzschulfächer auf welche von dem IDM unterstützt werden.
Die gesamt Menge aller abgestimmten Referenzschulfächer kann aus \reftabllec{tab:rest:api:school-subjects:list} entnommen werden.

\begin{longtable}{|p{0.1\textwidth}|p{0.2\textwidth}|p{0.64\textwidth}|}
\caption{Liste der abgestimmten Referenzschulfächer}
\endfoot
\caption{Liste der abgestimmten Referenzschulfächer}
		\label{tab:rest:api:school-subjects:list}
\endlastfoot 
\hline
\textbf{ID} & \textbf{Abkürzung} & \textbf{Ausgeschriebener Name} \\ \hline
\endhead
 & &  \\ \hline
	\end{longtable}

\subsection{Endpunkt in der REST-API: /api/school-subjects}
\label{sec:end:rest:api:school-subjects}
Die \reftabllec{tab:end:rest:api:school-subjects:meth} listet die zugelassenen Methoden. 
In der Tabelle <REF> ist eine Liste der vollst�ndigen F�cher enthalten.


\begin{table}[htbp]
	\begin{tabular}{|c|c|c|}
		\hline
			\textbf{Methode} & \textbf{Zugelassen} & \textbf{HTTP Methode} \\ \hline
			GET & Zugelassen & GET \\ \hline
			UPDATE & Nicht Zugelassen & \\ \hline 
			CREATE & Nicht Zugelassen & \\ \hline 
			DELETE & Nicht Zugelassen & \\ \hline
	\end{tabular}

		\caption{Zugelassen Methoden auf /api/school-subjects}
		\label{tab:end:rest:api:school-subjects:meth}
\end{table}
\input{rest/school-subjects/get}
\input{rest/school-subjects/create}
\input{rest/school-subjects/update}
\input{rest/school-subjects/delete}



\chapter{Weiterführende Konzepte}

\section{Löschkonzept}

Bei Verwendung der in diesem Dokument definierten REST-API muss der Client ein Löschkonzept auf die über die REST-API bezogenen Daten definieren bzw. diesem mit dem IDM-Provider abstimmen.\\

Der Client hat die Möglichkeit, über ein Benutzerkonto mit der Rolle "sync-system" die im Client vorhandenen Datenobjekte und Referenzen darauf über die REST-API beim IDM-Provider zu revalidieren. 
Existiert diese Möglichkeit der Rücksynchronisation im Client nicht, so müssen Überlegungen getroffen werden, wann Daten als veraltet gelten und nach welchen Aufbewahrungsfristen diese auf Client-Seite gelöscht werden. 
Für Benutzerkonten kann ein Kriterium sein, wie lange sich eine Person nicht mehr im Client eingeloggt hat, bevor Maßnahmen getroffen werden, den Datenstand aktuell zu halten (Ankündigung der Deaktivierung/Löschung des Benutzerkontos, Löschung des Benutzerkontos). \\

Bei Löschung eines Benutzerkontos im Client ist zu prüfen, inwiefern Daten vorliegen, die dem geistigen Eigentum der Inhaberin bzw. des Inhabers des Benutzerkontos unterliegen, und diese für einen definierten Zeitraum nach Löschung des Benutzerkontos durch die Inhaberin bzw. den Inhaber zur Sicherung abrufbar sind.
\section{Glossar}
\label{glossary}

\subsection{Begriffe der Fachdomäne}

\subsubsection*{Schüler}
Unterrichtsteilnehmer an einer Schule

\subsubsection*{Externe Schüler}
Schüler, die einer anderen Schule angehören und bei einer anderen Schule schulübergreifende Unterrichtsangebote wahrnehmen.

\subsubsection*{Eltern}
die für die Person der minderjährigen Schülerin oder des minderjährigen Schülers einzeln oder gemeinsam Sorgeberechtigten oder ihnen nach diesem Gesetz gleichgestellte Personen

\subsubsection*{Lehrer}
Lehrer und pädagogische Fachkräfte an Schulen, die Unterricht erteilen

\subsubsection*{Schulleitung}
Lehrkraft einer Schule sowie ggf. weitere Vertreter aus dem Lehrerkollegium

\subsubsection*{Schulen}
Einrichtungen, in denen unabhängig vom Wechsel der Lehrkräfte, Schülerinnen und Schüler durch planmäßiges und gemeinsames Lernen und durch das gemeinsame Schulleben bestimmte Erziehungs- und Bildungsziele erreicht werden sollen.

\subsubsection*{Schulträger}
Eine juristische oder natürliche Person, die für die Errichtung, Organisation und Verwaltung der einzelnen Schule rechtlich unmittelbar die Verantwortung trägt und zur Unterhaltung der Schule eigene Leistungen erbringt.

\subsubsection*{Schuljahr}
Beschreibt einen Zeitraum in dem Klassenstufen und Klassen für gewöhnlich definiert sind.

\subsubsection*{Klasse}
Lerngemeinschaft aus Schülern, welche für einen Zeitraum (z.B. Schuljahr 2020/21) an einer Schule unter Betreuung von Lehrkräften am Unterricht im Klassenverband teilnimmt. Eine Klassenstufe umfasst mehrere Klassen einer Stufe, z.B. (5a, 5b, 5c im Schuljahr 2020/21).

\subsubsection*{Unterrichtsfach}
In unterschiedlichen Fächern werden Unterrichtsinhalte den Schülern durch eine oder mehrere Lehrkräfte zu bestimmten Zeiten in einem bestimmten Zeitraum vermittelt.  


\subsection{Technische Begriffe}

\subsubsection*{REST API}

Technische Schnittstelle zum Daten


- rest api 
- endpoint, route --> url path
- IDM, IDM-Provider
- ID
- workflow
- authorization vs. authentication 
- authorization code
- authorization server 
- server
- Client
- JSON
- rest, http methode
- (sync-) system

\section{Datenmodell-Beispiele}
\label{Datenmodelle}
\lstset{
  language=json,
  tabsize=2,
  captionpos=b,
  numbers=left,
  commentstyle=\color{green},
  backgroundcolor=\color{white},
  numberstyle=\color{gray},
  keywordstyle=\color{blue} \textbf,%otherkeywords={xdata},
  keywordstyle=[2]\color{red}\textbf,
  identifierstyle=\color{black},
  stringstyle=\color{red}\ttfamily,
  basicstyle = \ttfamily \color{black} \footnotesize,
  showstringspaces=false,
	breakatwhitespace=false,         % sets if automatic breaks should only happen at whitespace
  breaklines=true, 
}
\begin{lstlisting}[caption={Beispiel Benutzer mit Rolle 'student'},frame=tlrb]
{
 id: "USER-01",
 studentId: "12345",
 name: "Leming",
 surname: "Zobel",
 birtdate: "2003-01-03",
 sex: "male",
 memberships: [
 {
  school: "SCHULE-01",
	role: "student",
	start: "2009-09-01",
	end: "2016-08-31",
	school-years: ["SJ-09/10","SJ-10/11","SJ-11/12","SJ-13/14","SJ-14/15","SJ-15/16"]
 }, {
  school: "SCHULE-04",
	role: "student",
	start: "2016-09-01",
	school-years: ["SJ-16/17","SJ-17/18","SJ-18/19","SJ-19/20","SJ-20/21"]
 },{
  school: "SCHULE-02",
	role: "external-student",
	start: "2019-09-01",
	end: "2020-08-31",
	school-years: ["SJ-19/20"]
 },
 ],
 guardians: [
 {
	user: "USER-02",
	start: "2009-09-01",
	end: "2020-01-03",
 },  {
	user: "USER-04",
	start: "2009-09-01",
	end: "2020-01-03",
 }
 ],
 classes: [
  {
	 class: "KLASSE-0001",
   school: "SCHULE-01",
	 schoolYear: "SJ-09/10",
	 start: "2009-09-01",
	 end: "2010-08-31",
	}, {
	 class: "KLASSE-0002",
   school: "SCHULE-01",
	 schoolYear: "SJ-10/11",
	 start: "2010-09-01",
	 end: "2011-08-31",
	}, {
	 class: "KLASSE-0003",
   school: "SCHULE-01",
	 schoolYear: "SJ-10/11",
	 start: "2010-09-01",
	 end: "2011-08-31",
	}, 
 ],
 subjects: [
  {
	 subject_id: "SUBJECT-0001",
	 subject_ref_id: "DE",
   school: "SCHULE-01",
	 schoolYear: "SJ-09/10",
	 start: "2009-09-01",
	 end: "2010-02-28".
	 time_tabel [
	  {
		 day: "1",
	   start: "08:00:00",
	   end: "08:45:00",
		 repeat: "weekly"
		}, {
		 day: "2",
	   start: "08:00:00",
	   end: "08:45:00",
		 repeat: "weekly"
		}, {
		 day: "3",
	   start: "08:50:00",
	   end: "09:35:00",
		 repeat: "biweekly",
		 start: "week-1"
		}, {
		 day: "4",
	   start: "08:50:00",
	   end: "09:35:00",
		 repeat: "biweekly",
		 start: "week-2"
		}, {
		 day: "3",
	   start: "08:50:00",
	   end: "09:35:00",
		 repeat: "once",
		 date: "2009-10-30"
		}
	}, {
	 subject_id: "SUBJECT-0002",
	 subject_ref_id: "MA",
   school: "SCHULE-01",
	 schoolYear: "SJ-09/10",
	 start: "2009-09-01",
	 end: "2010-08-31"
	}, 
	 
 ]
}
\end{lstlisting}

\lstset{
  language=json,
  tabsize=2,
  captionpos=b,
  numbers=left,
  commentstyle=\color{green},
  backgroundcolor=\color{white},
  numberstyle=\color{gray},
  keywordstyle=\color{blue} \textbf,%otherkeywords={xdata},
  keywordstyle=[2]\color{red}\textbf,
  identifierstyle=\color{black},
  stringstyle=\color{red}\ttfamily,
  basicstyle = \ttfamily \color{black} \footnotesize,
  showstringspaces=false,
	breakatwhitespace=false,         % sets if automatic breaks should only happen at whitespace
  breaklines=true, 
}
\begin{lstlisting}[caption={Beispiel für Benutzer mit Rollen 'teacher' und 'guardian'},frame=tlrb]
{
 id: "USER-02",
 name: "Altes Leming 1",
 surname: "Zobel",
 birtdate: "2003-01-03",
 sex: "female",
 schools: [
 {
  school: "SCHULE-01",
	role: "guardian",
	start: "2009-09-01",
	end: "2016-08-31",
	schoolPeriods: ["SJ-09/10","SJ-10/11","SJ-11/12","SJ-13/14","SJ-14/15","SJ-15/16"]
 }, {
  school: "SCHULE-04",
	role: "guardian",
	start: "2016-09-01",
	schoolPeriods: ["SJ-16/17","SJ-17/18","SJ-18/19","SJ-19/20","SJ-20/21"]
 }, {
  school: "SCHULE-02",
	role: "guardian",
	start: "2019-09-01",
	end: "2020-08-31",
	schoolPeriods: ["SJ-19/20"]
 }, {
  school: "SCHULE-02",
	role: "teacher",
	start: "2019-09-01"
 }
 ],
 children: [
 {
	user: "USER-01",
	start: "2009-09-01",
	end: "2020-01-03",
 },  {
	user: "USER-03",
	start: "2009-09-01",
	end: "2020-01-03",
 }
 ],
 classes: [
  {
	 class: "KLASSE-0031",
   school: "SCHULE-02",
	 schoolPeriod: "SJ-09/10",
	 start: "2009-09-01",
	 end: "2010-08-31",
	}, {
	 class: "KLASSE-0032",
   school: "SCHULE-02",
	 schoolPeriod: "SJ-20/21",
	 start: "2020-09-01",
	 end: "2021-08-31",
	}, {
	 class: "KLASSE-0033",
   school: "SCHULE-02",
	 schoolPeriod: "SJ-20/21",
	 start: "2020-09-01",
	 end: "2021-08-31",
	}, 
 ]
}
\end{lstlisting}

\input{appendix/datamodel/user_03}
\input{appendix/datamodel/user_04}
\input{appendix/datamodel/user_05}
\lstset{
  language=json,
  tabsize=2,
  captionpos=b,
  numbers=left,
  commentstyle=\color{green},
  backgroundcolor=\color{white},
  numberstyle=\color{gray},
  keywordstyle=\color{blue} \textbf,%otherkeywords={xdata},
  keywordstyle=[2]\color{red}\textbf,
  identifierstyle=\color{black},
  stringstyle=\color{red}\ttfamily,
  basicstyle = \ttfamily \color{black} \footnotesize,
  showstringspaces=false,
	breakatwhitespace=false,         % sets if automatic breaks should only happen at whitespace
  breaklines=true, 
}
\begin{lstlisting}[caption={Beispiel eines Schulfachs},frame=tlrb]
{
 subject: "SUBJECT-0001",
 name: "Deutsch 1-A"
 schoolSubject: ["DE",],
 school: "SCHULE-01",
 schoolYear: "SJ-09/10",
 start: "2009-09-01",
 end: "2010-02-28",
 classes: [ "KLASSE-01","KLASSE-03","KLASSE-05" ],
 grade: [ "1" ],
 students: [
  { 
   user: "USER-01",
	 start: "2009-09-01",
	 end: "2010-02-28",
	}, { 
	 user: "USER-06",
	 start: "2009-09-01",
	 end: "2010-02-28",
	}, { 
	 user: "USER-07",
	 start: "2009-09-01",
	 end: "2009-12-31",
	},
 ],
 teachers: [
  { 
   user: "USER-08",
	 start: "2009-09-01",
	 end: "2010-02-28",
	}, { 
	 user: "USER-09",
	 start: "2009-09-01",
	 end: "2009-12-31",
	}, { 
	 user: "USER-10",
	 start: "2009-10-05",
	 end: "2009-10-05",
	},
 ],
 timetable [
  {
	 day: "1",
	 start: "08:00:00",
	 end: "08:45:00",
	 repeat: "weekly"
  }, {
	 day: "2",
	 start: "08:00:00",
	 end: "08:45:00",
	 repeat: "weekly"
	}, {
	 day: "3",
	 start: "08:50:00",
	 end: "09:35:00",
	 repeat: "biweekly",
	 week: "week-1"
	}, {
	 day: "4",
	 start: "08:50:00",
	 end: "09:35:00",
	 repeat: "biweekly",
	 week: "week-2"
	}, {
	 day: "3",
	 start: "08:50:00",
	 end: "09:35:00",
	 repeat: "onetime",
	 date: "2009-10-30"
	}	 
 ]
}
\end{lstlisting}

\lstset{
  language=json,
  tabsize=2,
  captionpos=b,
  numbers=left,
  commentstyle=\color{green},
  backgroundcolor=\color{white},
  numberstyle=\color{gray},
  keywordstyle=\color{blue} \textbf,%otherkeywords={xdata},
  keywordstyle=[2]\color{red}\textbf,
  identifierstyle=\color{black},
  stringstyle=\color{red}\ttfamily,
  basicstyle = \ttfamily \color{black} \footnotesize,
  showstringspaces=false,
	breakatwhitespace=false,         % sets if automatic breaks should only happen at whitespace
  breaklines=true, 
}
\begin{lstlisting}[caption={Klassen-Datenmodell
 Beispiel 2: Grundschulklasse},frame=tlrb]
{
 class: "KLASSE-01",
 name: "Klasse 1-A",
 school: "SCHULE-01",
 schoolPeriod: "SJ-09/10",
 start: "2009-09-01",
 end: "2010-08-31",
 grade: [ "1", ],
 subjects: [ "SUBJECT-0001", "SUBJECT-0003", "SUBJECT-0004", ],
 students: [
  { 
   user: "USER-01",
	}, { 
	 user: "USER-06",
	}, { 
	 user: "USER-07",
	 end: "2009-12-31",
	}, { 
	 user: "USER-10",
	}, { 
	 user: "USER-11",
	},
 ],
 teachers: [
  { 
   user: "USER-208",
	 order: [
	  {
		 order: "1",
		},
	 ],
	}, { 
	 user: "USER-209",
	 order: [
	  {
		 order: "2",
		},
	 ],
	},
 ],
 representatives: [
  {
	 user: "USER-10",
	 role: "student",
	 order: "1",	 
	}, {
	 user: "USER-07",
	 end: "2009-12-31",
	 role: "student",
	 order: "2",	 
	}, {
	 user: "USER-01",
	 start: "2010-01-04",
	 role: "student",
	 order: "2",	 
	}, {
	 user: "USER-114",
	 role: "guardian",
	 order: "1",	 
	}, {
	 user: "USER-115",
	 role: "guardian",
	 order: "2",	 
	},  
 ]
}
\end{lstlisting}

\lstset{
  language=json,
  tabsize=2,
  captionpos=b,
  numbers=left,
  commentstyle=\color{green},
  backgroundcolor=\color{white},
  numberstyle=\color{gray},
  keywordstyle=\color{blue} \textbf,%otherkeywords={xdata},
  keywordstyle=[2]\color{red}\textbf,
  identifierstyle=\color{black},
  stringstyle=\color{red}\ttfamily,
  basicstyle = \ttfamily \color{black} \footnotesize,
  showstringspaces=false,
	breakatwhitespace=false,         % sets if automatic breaks should only happen at whitespace
  breaklines=true, 
}
\begin{lstlisting}[caption={Klassen-Datenmodell Beispiel 2: Jahrgangstufe 11, Sekundarstufe 2},frame=tlrb]
{
 class: "KLASSE-11",
 name: "Jarganstuffe 11",
 school: "SCHULE-04",
 schoolPeriod: "SJ-20/21",
 start: "2020-09-01",
 end: "2021-08-31",
 grade: [ "11", ],
 subjects: [  ],
 students: [
  { 
   user: "USER-01",
	}, { 
	 user: "USER-30",
	}, { 
	 user: "USER-31",
	}, { 
	 user: "USER-32",
	}, { 
	 user: "USER-33",
	},
 ],
 teachers: [
  { 
   user: "USER-228",
	 end: "2021-01-14",
	 order: [
	  {
		 order: "1",
		 end: "2021-01-14",
		},
	 ],
	}, { 
	 user: "USER-229",
	 order: [
	  {
		 order: "2",
		 end: "2021-01-14",
		}, {
		 order: "1",
		 start: "2021-01-15",
		},
	 ], { 
	 user: "USER-230",
	 start: "2021-01-15",
	 order: [
	  {
		 order: "2",
		 start: "2021-01-15",
		},
	 ],
	},
 ],
 representatives: [
  {
	 user: "USER-01",
	 role: "student",
	 order: "1",	 
	}, {
	 user: "USER-31",
	 role: "student",
	 order: "2",	 
	}, {
	 user: "USER-141",
   start: "2020-09-10",
	 role: "guardian",
	 order: "1",	 
	}, {
	 user: "USER-142",
   start: "2020-09-10",
	 role: "guardian",
	 order: "2",	 
	},  
 ]
}
\end{lstlisting}

\lstset{
  language=json,
  tabsize=2,
  captionpos=b,
  numbers=left,
  commentstyle=\color{green},
  backgroundcolor=\color{white},
  numberstyle=\color{gray},
  keywordstyle=\color{blue} \textbf,%otherkeywords={xdata},
  keywordstyle=[2]\color{red}\textbf,
  identifierstyle=\color{black},
  stringstyle=\color{red}\ttfamily,
  basicstyle = \ttfamily \color{black} \footnotesize,
  showstringspaces=false,
	breakatwhitespace=false,         % sets if automatic breaks should only happen at whitespace
  breaklines=true, 
}
\begin{lstlisting}[caption={Klassen-Datenmodell Beispiel 3: Jahrgangsübergreifende Klasse},frame=tlrb]
{
 class: "KLASSE-21",
 name: "Klasse 3,4 A",
 school: "SCHULE-07",
 schoolYear: "SJ-15/16",
 start: "2015-09-01",
 end: "2016-08-31",
 grade: [ "3", "4", ],
 subjects: [ "SUBJECT-0101", "SUBJECT-0102", "SUBJECT-0103", ],
 students: [
  { 
   user: "USER-51",
	}, { 
	 user: "USER-52",
	}, { 
	 user: "USER-53",
	}, { 
	 user: "USER-55",
	}, { 
	 user: "USER-54",
	 start: "2016-03-01",
	},
 ],
 teachers: [
  { 
   user: "USER-256",
	 order: [
	  {
		 order: "1",
		},
	 ],
	}, { 
	 user: "USER-257",
	 order: [
	  {
		 order: "1",
		},
	 ],
	}, { 
	 user: "USER-258",
	 order: [
	  {
		 order: "2",
		},
	 ],
	},
 ],
 representatives: [
  {
	 user: "USER-55",
	 role: "student",
	 order: "1",	 
	}, {
	 user: "USER-52",
	 role: "student",
	 order: "2",	 
	}, {
	 user: "USER-158",
	 start: "2015-09-10",
	 role: "guardian",
	 order: "1",	 
	}, {
	 user: "USER-159",
	 start: "2015-09-10",
	 role: "guardian",
	 order: "2",	 
	},  
 ]
}
\end{lstlisting}

\lstset{
  language=json,
  tabsize=2,
  captionpos=b,
  numbers=left,
  commentstyle=\color{green},
  backgroundcolor=\color{white},
  numberstyle=\color{gray},
  keywordstyle=\color{blue} \textbf,%otherkeywords={xdata},
  keywordstyle=[2]\color{red}\textbf,
  identifierstyle=\color{black},
  stringstyle=\color{red}\ttfamily,
  basicstyle = \ttfamily \color{black} \footnotesize,
  showstringspaces=false,
	breakatwhitespace=false,         % sets if automatic breaks should only happen at whitespace
  breaklines=true, 
}
\begin{lstlisting}[caption={Schulen-Datenmodell Beispiel 1},frame=tlrb]
{
 school: "SCHULE-01",
 name: "Zobel Grundschule",
 classes: [ 
  {
   class: "KLASSE-0001",
	 start: "2020-09-01",
   end: "2021-08-31",
  }, {
   class: "KLASSE-0002",
	 start: "2010-09-01",
	 end: "2011-08-31",
  }, {
   class: "KLASSE-0003", 
	 start: "2010-09-01",
	 end: "2011-08-31",
  },
 ],
 users: [
  {
   user: "USER-02",
 	 role: "guardian",
	 start: "2009-09-01",
	 end: "2016-08-31",
	 school-years: ["SJ-09/10","SJ-10/11","SJ-11/12","SJ-13/14","SJ-14/15","SJ-15/16"],
  }, {
   user: "USER-01",
	 role: "student",
	 start: "2009-09-01",
	 end: "2016-08-31",
	 school-years: ["SJ-09/10","SJ-10/11","SJ-11/12","SJ-13/14","SJ-14/15","SJ-15/16"],
  }, {
   user: "USER-208",
	 role: "teacher",
	 start: "2001-09-01",
  }, {
   user: "USER-209",
   role: "teacher",
 	 start: "2007-09-01",
  },
 ],
 subjects: [
  {
   subject: "SUBJECT-0001",
   start: "2009-09-01",
   end: "2010-02-28",
  }, {
   subject: "SUBJECT-0101",
   start: "2009-09-01",
   end: "2010-02-28",
  }, 
 ],
}
\end{lstlisting}

\input{appendix/datamodel/school_02}


\bibliographystyle{ieeetr}
\bibliography{rfc}

\listoffigures
%\addcontentsline{toc}{chapter}{\protect\numberline{}Abbildungsverzeichnis}%

\listoftables
%\addcontentsline{toc}{chapter}{\protect\numberline{}Tabellenverzeichnis}%
 
\lstlistoflistings

%\listofalgorithms
%\addcontentsline{toc}{chapter}{\protect\numberline{}Listingsverzeichnisses}%
%
\end{document}
