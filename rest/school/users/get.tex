\subsubsection{Get}
\label{sec:end:rest:api:school:users:get}
Es sind nur Anfragen mit der HTTP-GET-Methode für ein GET der Daten zugelassen.
Die Antwort erfolgt mit den JSON-Objekt \reflisting{lst:code:end:rest:api:school:users:get:ret}. Die einzelnen Felder der Antwort werden in Tabelle \reftabllec{tab:end:rest:api:school:users:get:ret:json} beschrieben.
Die Berechtingungen auf den Entdpoint können aus \reftabllec{tab:end:rest:api:school:users:get:right} entnommen werden.
\lstset{
  language=json,
  tabsize=2,
  captionpos=b,
  numbers=left,
  commentstyle=\color{green},
  backgroundcolor=\color{white},
  numberstyle=\color{gray},
  keywordstyle=\color{blue} \textbf,%otherkeywords={xdata},
  keywordstyle=[2]\color{red}\textbf,
  identifierstyle=\color{black},
  stringstyle=\color{red}\ttfamily,
  basicstyle = \ttfamily \color{black} \footnotesize,
  showstringspaces=false,
	breakatwhitespace=false,         % sets if automatic breaks should only happen at whitespace
  breaklines=true, 
}
\begin{lstlisting}[caption={JSON-Antwort für einen GET-Aufruf der Route /api/school/users},label={lst:code:end:rest:api:school:users:get:ret},frame=tlrb]
[
 {
  school_id: "<STRING>",
  user_id: "<STRING>",
	rolle: "<STRING>"
 },
...
 {
  school_id: "<STRING>",
  user_id: "<STRING>",
	rolle: "<STRING>"
 },
]
\end{lstlisting}

\begin{table}[htb]
	\begin{tabularx}{\textwidth}{|c|c|X|}
		\hline
			\textbf{Feldname} & \textbf{Datentyp} & \textbf{Beschreibung} \\ \hline
			school\_id & STRING & ID der Schule. \\ \hline
			user\_id & STRING & ID des Benutzers \\ \hline
			rolle & STRING & Rolle des Benutzers an der Schule. \\ \hline
	\end{tabularx}

		\caption{Beschreibung der Felder in eine JSON-Objekt für ein Benutzer an einer Schulle}
		\label{tab:end:rest:api:school:users:get:ret:json}
\end{table}

\begin{table}[htb]
	\begin{tabularx}{\textwidth}{|c|X|}
		\hline
\textbf{Benutzerguppen} & \textbf{Zugelassene Daten} \\ \hline
guest & Darf den Endpunkt nicht aufrufen und keine Daten vom Endpunkt erhalten. \\ \hline
students  & Bekommt eine Liste seine Daten, 
           den Daten ihrer Erziehungsberechtigten ( nur rolle parents ), 
					 den Daten seiner Lehrer (nur Rolle teacher) und 
					 den Daten Schulleiter ( nur Rolle principal ) wieder.\\ \hline
parents & Bekommt eine Liste mit seinen Daten, 
          den Daten ihre Kinder unter 18 oder Personen für die sie die Vormundschaft haben ( nur Rolle studenst), 
					den Daten der Lehrer ihre Kinder unter 18 oder Personen für die sie die Vormundschaft haben ( nur Rolle teacher ) und
					den Daten der Schulleiter ihre Kinder unter 18 oder Personen für die sie die Vormundschaft haben ( nur Rolle principal )
					wieder.\\ \hline
teacher & Bekommt eine Liste mit Seinen Daten, 
          den Daten aller Schüler die Sie Unterrichten ( nur Rolle students  ),
					den Daten aller Erziehungsberechtigten von Schülern unter 18 oder Schülern  mit einem Vormund die Sie Unterrichten ( nur Rolle parents ) und
					den Daten aller Kollegen ( nur mit den Rollen: teacher, principal, school-admin ) 
					wieder.\\ \hline
principal & Bekommt eine Liste mit Seinen Daten,
            den Daten aller Schüler von Schulen wo Sie Schulleiter sind ( Rolle students  ),
						den Daten aller Erziehungsberechtigten von Schülern von Schulen wo Sie Schulleiter sind ( nur Rolle parents ) und
					  den Daten aller Kollegen ( nur mit den Rollen: teacher, principal, school-admin )
						wieder.\\ \hline
school-admin & Bekommt eine Liste mit Seinen Daten und
               den Daten aller Personen von den Schulen wo er die Rolle school-admin hat mit den Rollen students, parents, teacher, principal und school-admin in den rollen students, parents, teacher, principal und school-admin wieder.  \\ \hline
school-board & \textcolor[rgb]{1,0.41,0.13}{\textbf{Muss geklärt werden, welche Einschränkungen auf den Daten es hier gibt}} \\ \hline
fed-school-board & \textcolor[rgb]{1,0.41,0.13}{\textbf{Muss geklärt werden, welche Einschränkungen auf den Daten es hier gibt}} \\ \hline
sync-systems & Liste mit allen Personen in den Rollen der Personen an den Schulen von den es Syncen darf. \\ \hline

	\end{tabularx}

		\caption{Berechtigungen auf dem Endpunkt}
		\label{tab:end:rest:api:school:users:get:right}
\end{table}