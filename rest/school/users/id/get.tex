\subsubsection{Get}
\label{sec:end:rest:api:school:users:get:id}
Es sind nur Anfragen mit der HTTP-GET-Methode für ein GET der Daten zugelassen.
Bei diesem Aufruf wird eine Liste mit von Objecten auf welche der aktuelle Benutzer zugriff hat eingeschränkt auf die Schulle welche über die ID ausgewählt wurde übermittelt.
Ein Object aus der liste Enthält immer für genau eine Benutzer die Information der Benutzer ID, der Schul ID, der Rollen ID, eines Zeitpunktes ab wann das Object gültig war und optional ein Zeitpunkt bis wann das Object gültig ist.
Gibt es Unterbrechungen in den Zeiträumen in den ein Benutzer eine Rolle in einer Schule eingenommen hat, so muss es für jeden Zeitraum ein eigens Object geben.
Zum Abrufen von Daten mit IDs müssen die Anwendung über die entsprechenden spezial Routen gehen.
Die Antwort erfolgt mit dem JSON-Objekt \reflisting{lst:code:end:rest:api:school:users:get:id:ret}. Die einzelnen Felder der Antwort werden in Tabelle \reftabllec{tab:end:rest:api:school:users:get:id:ret:json} beschrieben.
Die Berechtigungungen auf den Endpoint können aus \reftabllec{tab:end:rest:api:school:users:get:id:right} entnommen werden.

\lstset{
  language=json,
  tabsize=2,
  captionpos=b,
  numbers=left,
  commentstyle=\color{green},
  backgroundcolor=\color{white},
  numberstyle=\color{gray},
  keywordstyle=\color{blue} \textbf,%otherkeywords={xdata},
  keywordstyle=[2]\color{red}\textbf,
  identifierstyle=\color{black},
  stringstyle=\color{red}\ttfamily,
  basicstyle = \ttfamily \color{black} \footnotesize,
  showstringspaces=false,
	breakatwhitespace=false,         % sets if automatic breaks should only happen at whitespace
  breaklines=true, 
}
\begin{lstlisting}[caption={JSON-Antwort für einen GET-Aufruf der Route /api/school/users/\$id},label={lst:code:end:rest:api:school:users:get:id:ret},frame=tlrb]
[
 {
  school_id: "<STRING>",
  user_id: "<STRING>",
	role: "<STRING>",
	start: "<TIMESTAMP>",
	end: "<TIMESTAMP>"
 },
...
 {
  school_id: "<STRING>",
  user_id: "<STRING>",
	role: "<STRING>",
	start: "<TIMESTAMP>",
	end: "<TIMESTAMP>"
 },
]
\end{lstlisting}

\begin{table}[htb]
	\begin{tabularx}{\textwidth}{|c|c|X|}
		\hline
			\textbf{Feldname} & \textbf{Datentyp} & \textbf{Beschreibung} \\ \hline
			school\_id & STRING & ID der Schule. \\ \hline
			user\_id & STRING & ID des Benutzers \\ \hline
			role & STRING & Rolle des Benutzers an der Schule. \\ \hline
			start & TIMESTAMP & Zeitpunkt ab wann der Benutzer diese Rolle innen hat. \\ \hline
			end & TIMESTAMP & Optional, wenn das Feld da ist gibt es an bis wan ein Benutzer diese Rolle innen hatte. \\ \hline
	\end{tabularx}

		\caption{Beschreibung der Felder in eine JSON-Objekt für ein Benutzer an einer Schulle}
		\label{tab:end:rest:api:school:users:get:id:ret:json}
\end{table}

\begin{longtable}{|c|p{0.7\textwidth}|}
		\caption{Berechtigungen auf dem Endpunkt}
		\label{tab:end:rest:api:school:users:get:id:right} \\
\hline
\textbf{Benutzergruppen} & \textbf{Zugelassene Daten} \\ \hline
\endfirsthead
\caption{Berechtigungen auf dem Endpunkt}\\
\hline
\textbf{Benutzergruppen} & \textbf{Zugelassene Daten} \\ \hline
\endhead
guest & Darf den Endpunkt nicht aufrufen und keine Daten vom Endpunkt erhalten. \\ \hline
user & Bekommt eine Liste mit ihren Daten von der ausgewählten Schule \\ \hline 
students & die Daten aller iherer Mitschüller mit den Sie gemeinsame Klassen oder Kurse haben, von der ausgewählten Schule ( nur Rollen: \textit{students} und \textit{external-students} ),
           den Daten ihrer Erziehungsberechtigten (nur Rolle \textit{parents}), 
					 den Daten ihrer Lehrer, von der ausgewählten Schule (nur Rolle \textit{teacher}) und 
					 den Daten ihrer Schulleiter, von der ausgewählten Schule, (nur Rolle \textit{principal}) wieder.\\ \hline
external-students &  die Daten aller iherer Mitschüller mit den Sie gemeinsame Klassen oder Kurse haben, von der ausgewählten Schule ( nur Rollen: \textit{students} und \textit{external-students} ),
           den Daten ihrer Lehrer, von der ausgewählten Schule (nur Rolle \textit{teacher}) und 
					 den Daten ihres Schulleiters, von der ausgewählten Schule (nur Rolle \textit{principal}) wieder.\\ \hline
parents & den Daten ihrer Kinder unter 18, von der ausgewählten Schule, ( nur Rollen: \textit{students} und \textit{external-students}), 
					den Daten der Personen für die Sie Vormund ist, von der ausgewählten Schule, ( nur Rollen: \textit{students} und \textit{external-students}), 
					den Daten der Lehrer ihrer Kinder unter 18, von der ausgewählten Schule, (nur Rolle \textit{teacher}),
					den Daten der Lehrer von Personen für die Sie Vormund ist, von der ausgewählten Schule, (nur Rolle \textit{teacher}),
					den Daten der Schulleiter ihrer Kinder unter 18, von der ausgewählten Schule, (nur Rolle \textit{principal}) und
					die Daten der Schulleiter von Personen für die sie die Vormundschaft hat, von der ausgewählten Schule, (nur Rolle \textit{principal})
					wieder.\\ \hline
teacher & den Daten aller Schüler, die sie unterrichten, von der ausgewählten Schule, ( nur Rollen: \textit{students} und \textit{external-students}),
					den Daten aller Erziehungsberechtigten von Schülern unter 18 die Sie Unterrichten, von der ausgewählten Schule, (nur Rolle \textit{parents}),
					den Daten aller Erziehungsberechtigten von Schülern mit einem Vormund die Sie Unterrichten, von der ausgewählten Schule, (nur Rolle \textit{parents}) und
					den Daten aller Kollegen, von der ausgewählten Schule, (nur mit den Rollen \textit{teacher}, \textit{principal}, \textit{school-admin}) 
					wieder.\\ \hline
principal & Bekommt eine Liste mit ihren Daten,
            den Daten aller Schüler, von der ausgewählten Schule, an denen Sie Schulleiter sind ( nur Rollen: \textit{students} und \textit{external-students}),
						den Daten aller Erziehungsberechtigten von Schülern, von der ausgewählten Schule, an denen Sie Schulleiter sind (nur Rolle \textit{parents}) und
					  den Daten aller Kollegen, von der ausgewählten Schule, (nur mit den Rollen \textit{teacher}, \textit{principal}, \textit{school-admin})
						wieder.\\ \hline
school-admin & Bekommt eine Liste mit ihren Daten und
               den Daten aller Personen, von der ausgewählten Schule, an denen er die Rolle \textit{school-admin} hat mit den Rollen \textit{students}, \textit{external-students}, \textit{parents}, \textit{teacher}, \textit{principal} und \textit{school-admin} wieder.  \\ \hline
school-board & \textcolor[rgb]{1,0.41,0.13}{\textbf{Muss geklärt werden, welche Einschränkungen auf den Daten es hier gibt}} \\ \hline
fed-school-board & \textcolor[rgb]{1,0.41,0.13}{\textbf{Muss geklärt werden, welche Einschränkungen auf den Daten es hier gibt}} \\ \hline
sync-systems & Liste mit allen Personen, in den Rollen der Personen an den Schulen, von den es syncen darf. \\ \hline
	\end{longtable}