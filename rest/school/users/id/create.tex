\subsubsection{CREATE}
\label{sec:rest:api:school:users:create:id}
Es sind nur Anfragen mit der HTTP-POST-Methode für ein CREATE der Daten zugelassen.
Die Aufgabe von CREATE ist es ein Benutzer einer Schule in einer bestimmten Rolle zuzufügen.
Die Felder im JSON Object, \reflisting{lst:code:rest:api:school:users:id:create:request}, im HTTP-Post Body wird für das CREATE wird in \reftabllec{tab:rest:api:school:users:id:create:obj}  beschrieben.
Im Erfolgsfall wird der Request mit den HTTP-Statuscode 200 - OK Beantwortet.
In allen anderen Fällen werden Anfragen mit den HTTP-Statuscode 403 - Forbidden Beantwortet.
Es gilt für diesen Endpunkt das der Zugriff explizit Erlaubt sein muss, \reftabllec{tab:rest:api:school:users:id:create:right}, ansonsten gibt der Endpunkt den HTTP-Statuscode 403 - Forbidden wieder.

Wird ein Datensatz an den Endpunkt für ein Benutzer mit der Rolle \textit{students} gesendet, gilt das wenn der Benutzer schon ein Eintrag mit der Rolle \textit{students} hat der noch Aktiv ist so wird dessen Enddatum auf das Startdatum des neuen Eintrages gestellt.
Wenn für Benutzer mit der Rolle \textit{students}, der unter 18 ist, ein Eintrag erstellt wird, muss für seine Eltern, Erziehungsberechtigten oder Vormünder vom System aus entsprechende Einträge für diese Personen mit der Rolle \textit{guardians} Erzeugt werden müssen.
Wird für ein Benutzer über 18 mit der Rolle \textit{students} ein Eintrag erstellt und sind vom einen Gericht ein oder mehrere Vormünder bestimmt wurden so muss vom System aus die Entsprechenden Einträger für diese Person mit der Rolle \textit{guardians} Erzeugt werden müssen.

Wird für ein Benutzer mit der Rolle \textit{external-students} ein Datensatz an dem Endpunkt gesendet, muss für seine Eltern, Erziehungsberechtigten oder Vormünder vom System aus entsprechende Einträge für diese Personen mit der Rolle \textit{parents} Erzeugt werden müssen.


\lstset{
  language=json,
  tabsize=2,
  captionpos=b,
  numbers=left,
  commentstyle=\color{green},
  backgroundcolor=\color{white},
  numberstyle=\color{gray},
  keywordstyle=\color{blue} \textbf,%otherkeywords={xdata},
  keywordstyle=[2]\color{red}\textbf,
  identifierstyle=\color{black},
  stringstyle=\color{red}\ttfamily,
  basicstyle = \ttfamily \color{black} \footnotesize,
  showstringspaces=false,
	breakatwhitespace=false,         % sets if automatic breaks should only happen at whitespace
  breaklines=true, 
}
\begin{lstlisting}[caption={Felder im JSON-Object einer CREATE anfrage per HTTP-POST auf der Route /api/school/users/\$id},label={lst:code:rest:api:school:users:id:create:request},frame=tlrb]
{
 user_id: "<STRING>",
 role: "<STRING>",
 start: "<TIMESTAMP>"
 school-years: ["<STRING>",...,"<STRING>"]
}
\end{lstlisting}


\begin{longtable}{|p{0.2\textwidth}|p{0.2\textwidth}|p{0.58\textwidth}|}
		\caption{Beschreibung der Felder in einem JSON-Objekt für das Zuordnen eines Benutzer in einer Rolle zu einer Schule}
\endfoot
		\caption{Beschreibung der Felder in einem JSON-Objekt für das Zuordnen eines Benutzer in einer Rolle zu einer Schule}
		\label{tab:rest:api:school:users:id:create:obj}
\endlastfoot 
\hline
			\textbf{Feldname} & \textbf{Datentyp} & \textbf{Beschreibung} \\ \hline
\endhead
			user\_id & STRING & ID des Benutzers \\ \hline
			role & STRING & Rolle des Benutzers an der Schule. \\ \hline
			start & TIMESTAMP & Zeitpunkt ab wann der Benutzer diese Rolle innen hat. \\ \hline
			school-years & LISTE of STRINGS & Optional, das Feld wird nur benötigt wenn die Rolle \textit{students} oder \textit{external-students} ist, Enthält eine Liste von Schuljahres IDs. \\ \hline 
\end{longtable}
	
\begin{longtable}{|c|p{0.7\textwidth}|}
\caption{Berechtigungen auf dem Endpunkt}
\endfoot
		\caption{Berechtigungen auf dem Endpunkt}
		\label{tab:rest:api:school:users:id:create:right}
\endlastfoot
\hline
\textbf{Benutzergruppen} & \textbf{Zugelassene Daten} \\ \hline
\endhead
principal & Darf Einträge für Personen mit den Rollen \textit{students}, \textit{teacher}, \textit{principal} und \textit{school-admin} an der ausgewählten Schule anlegen.
            Darf für User mit der Rolle \textit{students} von seiner Schule, an ausgewählten Schule, mit der Rolle \textit{external-students} anlegen und somit diesen Schüler für ausgewählten Schule freigeben.\\ \hline
school-admin & Darf Einträge für Personen mit den Rollen \textit{students}, \textit{teacher}, \textit{principal} und \textit{school-admin} an der ausgewählten Schule anlegen.
            Darf für User mit der Rolle \textit{students} von seiner Schule, an ausgewählten Schule, mit der Rolle \textit{external-students} anlegen und somit diesen Schüler für ausgewählten Schule freigeben.  \\ \hline
school-board & Darf Einträge für Personen mit den Rollen \textit{students}, \textit{teacher}, \textit{principal} und \textit{school-admin} an der ausgewählten Schule anlegen, wenn der Schulträger für die Schule Zuständig ist.
            Darf für User mit der Rolle \textit{students} für Schulen, von Schulen zu dessen Schulträger er gehört, an der ausgewählten Schule, Einträge mit der Rolle \textit{external-students} anlegen und somit diesen Schüler für ausgewählten Schule freigeben. \\ \hline
fed-school-board & Darf Einträge für User mit den Rollen \textit{students}, \textit{external-students}, \textit{teacher}, \textit{principal} und \textit{school-admin} an der ausgewählten Schule anlegen.\\ \hline
	\end{longtable}