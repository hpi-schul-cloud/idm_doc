\subsubsection{READ}
\label{sec:rest:api:school-years:read}
Es sind nur Anfragen mit der HTTP-GET-Methode für ein READ auf die Daten zugelassen.
Unter diesem Pfad wird eine Liste mit allen im System existierenden Schuljahren zurückgegeben.
Jedes Schuljahr hat eine eindeutigen Identifikator, einen Zeitraum, von wann bis wann das Schuljahr andauerte, und ein Anzeigenamen.

Die Antwort erfolgt mit dem JSON-Objekt \reflisting{lst:code:rest:api:school-years:read:ret}. 
Die einzelnen Felder der Antwort werden in \reftabllec{tab:rest:api:school-years:read:ret:json} beschrieben.
\lstset{
  language=json,
  tabsize=2,
  captionpos=b,
  numbers=left,
  commentstyle=\color{green},
  backgroundcolor=\color{white},
  numberstyle=\color{gray},
  keywordstyle=\color{blue} \textbf,%otherkeywords={xdata},
  keywordstyle=[2]\color{red}\textbf,
  identifierstyle=\color{black},
  stringstyle=\color{red}\ttfamily,
  basicstyle = \ttfamily \color{black} \footnotesize,
  showstringspaces=false,
	breakatwhitespace=false,         % sets if automatic breaks should only happen at whitespace
  breaklines=true, 
}
\begin{lstlisting}[caption={JSON-Antwort für einen GET-Aufruf des Pfads /api/school-years},label={lst:code:rest:api:school-years:read:ret},frame=tlrb]
[
{
    id: "<STRING>",
    displayName: "<STRING>",
    start: "<DATE>",
    end: "<DATE>"
},
...
]
\end{lstlisting}

\begin{longtable}{|p{0.16\textwidth}|p{0.16\textwidth}|p{0.58\textwidth}|}
		\caption{Beschreibung der Felder in einem JSON-Objekt für ein Schuljahr}
\endfoot
		\caption{Beschreibung der Felder in einem JSON-Objekt für ein Schuljahr}
		\label{tab:rest:api:school-years:read:ret:json}
\endlastfoot 
\hline
			\textbf{Feldname} & \textbf{Datentyp} & \textbf{Beschreibung} \\ \hline
\endhead
 id & STRING & Eine eindeutige Zeichenkette, welche das Schuljahr im IdM identifiziert. \\ \hline
 displayName & STRING & Eine Name für das Schuljahr wie z.b "2020-2021" \\ \hline
 start & DATE & Eine Datumsangabe nach ISO-8601 im Format YYYY-MM-DD. Gibt an, wann das Schuljahr begonnen hat. \\ \hline
 end & DATE & Eine Datumsangabe nach ISO-8601 im Format YYYY-MM-DD. Gibt an, wann das Schuljahr endet. \\ \hline
\end{longtable}
