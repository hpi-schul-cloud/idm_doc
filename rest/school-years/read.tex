\subsubsection{READ}
\label{sec:rest:api:school-years:read}
Es sind nur Anfragen mit der HTTP-GET-Methode für ein READ auf die Daten zugelassen.
Unter diesem Pfad wird eine Liste mit allen im System existierenden Schuljahren zurückgegeben.
Jedes Schuljahr hat eine eindeutigen Identifikator, einen Zeitraum, von wann bis wann das Schuljahr andauerte, und ein Anzeigenamen.

Die Antwort erfolgt mit dem JSON-Objekt \reflisting{lst:code:rest:api:school-years:read:ret}. 
Die einzelnen Felder der Antwort werden in \reftabllec{tab:rest:api:school-years:read:ret:json} beschrieben.
Die Berechtigungen auf den Endpunkt können \reftabllec{tab:rest:api:school-years:read:right} entnommen werden.
\lstset{
  language=json,
  tabsize=2,
  captionpos=b,
  numbers=left,
  commentstyle=\color{green},
  backgroundcolor=\color{white},
  numberstyle=\color{gray},
  keywordstyle=\color{blue} \textbf,%otherkeywords={xdata},
  keywordstyle=[2]\color{red}\textbf,
  identifierstyle=\color{black},
  stringstyle=\color{red}\ttfamily,
  basicstyle = \ttfamily \color{black} \footnotesize,
  showstringspaces=false,
	breakatwhitespace=false,         % sets if automatic breaks should only happen at whitespace
  breaklines=true, 
}
\begin{lstlisting}[caption={JSON-Antwort für einen GET-Aufruf der Route /api/school-years},label={lst:code:rest:api:school-years:read:ret},frame=tlrb]
[
{
 school-year: "<STRING>",
 start: "<DATE>",
 end: "<DATE>",
 name: "<STRING>",
},
...
]
\end{lstlisting}

\begin{longtable}{|p{0.16\textwidth}|p{0.16\textwidth}|p{0.58\textwidth}|}
		\caption{Beschreibung der Felder in einem JSON-Objekt aus der Liste der Schuljahre}
\endfoot
		\caption{Beschreibung der Felder in einem JSON-Objekt aus der Liste der Schuljahre}
		\label{tab:rest:api:school-years:read:ret:json}
\endlastfoot 
\hline
			\textbf{Feldname} & \textbf{Datentyp} & \textbf{Beschreibung} \\ \hline
\endhead
 school-year & STRING & Eine eindeutige Zeichenkette, welche das Schuljahr im IDM identifiziert. \\ \hline
 start & DATE & Ein Datumsangabe nach ISO-8601 in dem Format YYYY-MM-DD, welche angibt, wann das Schuljahr begonnen hat. \\ \hline
 ende & DATE & Ein Datumsangabe nach ISO-8601 in dem Format YYYY-MM-DD, welche angibt, wann das Schuljahr endet. \\ \hline
 name & STRING & Eine Name für das Schuljahr wie z.b.: 2020-2021 \\ \hline
\end{longtable}
