\subsubsection{READ}
\label{sec:rest:api:user:childs:read}
Es sind nur Anfragen mit der HTTP-GET-Methode für ein READ auf die Daten zugelassen.
Die Antwort erfolgt mit dem JSON-Objekt \reflisting{lst:code:rest:api:user:childs:read:ret}. Die einzelnen Felder der Antwort werden in Tabelle \reftabllec{tab:rest:api:user:childs:read:ret:json} beschrieben.
Die Berechtigungen auf den Endpoint können \reftabllec{tab:rest:api:user:childs:read:right} entnommen werden.
\lstset{
  language=json,
  tabsize=2,
  captionpos=b,
  numbers=left,
  commentstyle=\color{green},
  backgroundcolor=\color{white},
  numberstyle=\color{gray},
  keywordstyle=\color{blue} \textbf,%otherkeywords={xdata},
  keywordstyle=[2]\color{red}\textbf,
  identifierstyle=\color{black},
  stringstyle=\color{red}\ttfamily,
  basicstyle = \ttfamily \color{black} \footnotesize,
  showstringspaces=false,
	breakatwhitespace=false,         % sets if automatic breaks should only happen at whitespace
  breaklines=true, 
}
\begin{lstlisting}[caption={JSON-Antwort für einen GET-Aufruf der Route /api/user/childs},label={lst:code:rest:api:user:childs:read:ret},frame=tlrb]
[
 id: "<STRING>",
...
 id: "<STRING>",
]
\end{lstlisting}
