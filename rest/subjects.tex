\section{Schnittstellen für Unterrichtsfächer an Schulen}
Ein Unterrichtsfach beschreibt die regelmäßige Wissensvermittlung durch eine Lehrperson an einer Schule (s. \refabschnitt{sec:end:rest:api:schools}) in Bezug auf eine oder mehrere Klassen/Klassenstufen (s. \refabschnitt{sec:end:rest:api:classes:id}) innerhalb eines Schulturnus (s. \refabschnitt{sec:end:rest:api:school-periods}). Dabei wird ein bestimmtes Thema abgedeckt, das durch das Referenzschulfach (s. \refabschnitt{sec:end:rest:api:reference-subjects}) angegeben werden kann.


%\subsection{Endpunkt in der REST-API: /api/subjects}
\label{sec:end:rest:api:schools-periods}
An diesem Endpunkt werden Informationen zu Unterrichtsfächern verarbeitet. Ein Unterrichtsfach beschreibt die regelmäßige Wissensvermittlung durch eine Lehrperson an einer Schule (s. \refabschnitt{sec:end:rest:api:schools}) in Bezug auf eine Klasse/Klassenstufe (s. \refabschnitt{sec:end:rest:api:classes}) innerhalb einer schulischen Periode (s. \refabschnitt{sec:end:rest:api:schools-periods}). Dabei wird ein bestimmtes Thema abgedeckt, das durch das Referenzschulfach (s. \refabschnitt{sec:end:rest:api:reference-subjects}) angegeben werden kann.

Die \reftabllec{tab:end:rest:api:subjects:meth} listet auf, welche Operationen zugelassen sind und welche HTTP-Methoden dabei verwendet werden. 

\begin{table}[!htbp]
	\begin{tabular}{|c|c|c|}
		\hline
			\textbf{Operation} & \textbf{Zugelassen?} & \textbf{HTTP-Methode} \\ \hline
			CREATE & Nein &  \\ \hline 
			READ & Ja & GET \\ \hline
			UPDATE & Nein & \\ \hline 
			DELETE & Nein & \\ \hline
	\end{tabular}

		\caption{Zugelassene Operationen auf /api/subjects}
		\label{tab:end:rest:api:subjects:meth}
\end{table}

\subsubsection{READ}
\label{sec:end:rest:api:subjects:read}
Es sind nur Anfragen mit der HTTP-GET-Methode für ein READ auf die Daten zugelassen.
Es werden nur die IDs der Daten präsentiert, die für den anfragendem Benutzer in seinem Kontext existieren.

Die Antwort erfolgt mit dem JSON-Objekt \reflisting{lst:code:rest:api:subjects:read:ret}. 
Die einzelnen Felder der Antwort werden in \reftabllec{tab:rest:api:subjects:read:ret} beschrieben.
Die Berechtigungen auf den Endpunkt können \reftabllec{tab:rest:api:subjects:read:right} entnommen werden.

\lstset{
  language=json,
  tabsize=2,
  captionpos=b,
  numbers=left,
  commentstyle=\color{green},
  backgroundcolor=\color{white},
  numberstyle=\color{gray},
  keywordstyle=\color{blue} \textbf,%otherkeywords={xdata},
  keywordstyle=[2]\color{red}\textbf,
  identifierstyle=\color{black},
  stringstyle=\color{red}\ttfamily,
  basicstyle = \ttfamily \color{black} \footnotesize,
  showstringspaces=false,
	breakatwhitespace=false,         % sets if automatic breaks should only happen at whitespace
  breaklines=true, 
}
\begin{lstlisting}[caption={JSON-Antwort für einen GET-Aufruf des Pfads /api/subjects},label={lst:code:rest:api:subjects:read:ret},frame=tlrb]
[
 "<STRING>",
 ...
]
\end{lstlisting}


\begin{longtable}{|p{0.16\textwidth}|p{0.16\textwidth}|p{0.58\textwidth}|}
		\caption{Beschreibung der Zeichenkete in der JSON Liste für Unterrichtsfächer.}
\endfoot
		\caption{Beschreibung der Zeichenkete in der JSON Liste für Unterrichtsfächer.}
		\label{tab:rest:api:subjects:read:ret}
\endlastfoot 
\hline
			\textbf{Feldname} & \textbf{Datentyp} & \textbf{Beschreibung} \\ \hline
\endhead
 & STRING & Zeichenkette, mit der weitere Informationen zu einem Unterrichtsfach unter /api/subjects/\$id abgefragt werden kann. \\ \hline
\end{longtable}
\subsection{Endpunkt in der REST-API: /api/subjects/\$id}
Die \reftabllec{tab:rest:api:subjects:id:meth} listet auf, welche Operationen zugelassen sind und welche HTTP-Methoden dabei verwendet werden. 

\begin{table}[!htbp]
	\begin{tabular}{|c|c|c|}
		\hline
			\textbf{Operation} & \textbf{Zugelassen?} & \textbf{HTTP-Methode} \\ \hline
			CREATE & Ja & POST \\ \hline 
			READ & Ja & GET \\ \hline
			UPDATE & Ja & POST \\ \hline 
			DELETE & Ja & POST \\ \hline
	\end{tabular}

		\caption{Zugelassene Operationen auf /api/subjects/\$id}
		\label{tab:rest:api:subjects:id:meth}
\end{table}

\subsubsection{READ}
\label{sec:rest:api:subjects:id:read}
Es sind nur Anfragen mit der HTTP-GET-Methode für ein READ auf die Daten zugelassen.
Dieser Pfad gibt bei einer READ-Anfrage für das per ID ausgewählte Unterrichtsfach die allgemeinen Informationen davon wieder.
Die Daten, welche wiedergegeben werden, sind durch den Kontext des anfragenden Benutzers eingeschränkt.

Die Antwort erfolgt mit dem JSON-Objekt \reflisting{lst:code:rest:api:subjects:id:read:ret}. 
Die einzelnen Felder der Antwort werden in \reftabllec{tab:rest:api:subjects:id:read:ret:json} beschrieben.

\lstset{
  language=json,
  tabsize=2,
  captionpos=b,
  numbers=left,
  commentstyle=\color{green},
  backgroundcolor=\color{white},
  numberstyle=\color{gray},
  keywordstyle=\color{blue} \textbf,%otherkeywords={xdata},
  keywordstyle=[2]\color{red}\textbf,
  identifierstyle=\color{black},
  stringstyle=\color{red}\ttfamily,
  basicstyle = \ttfamily \color{black} \footnotesize,
  showstringspaces=false,
	breakatwhitespace=false,         % sets if automatic breaks should only happen at whitespace
  breaklines=true, 
}
\begin{lstlisting}[caption={JSON-Antwort für einen GET-Aufruf des Pfads /api/subjects/\$id},label={lst:code:rest:api:subjects:id:read:ret},frame=tlrb]
{
    id: "<STRING>",
    displayName: "<STRING>",
    referenceSubject: [
    {
        id: "<STRING>",
        displayName: "<STRING>"
    },
    ...
    ],
    school: {
        id: "<STRING>",
        displayName: "<STRING>"
    },
    schoolPeriod: {
  y          id: "<STRING>",
            displayName: "<STRING>"
    },
    grade: [
        "STRING",
        ...
    ],
    users: [
    {
        id: "<STRING>",
        roleWithinSubject: <STRING>",
        start: "<DATE>",
        end: "<DATE>"    
    },
    ...
    ]
 }
\end{lstlisting}

\begin{longtable}{|p{0.16\textwidth}|p{0.16\textwidth}|p{0.58\textwidth}|}
		\caption{Beschreibung der Felder in einem JSON-Objekt für ein Unterrichtsfach}
\endfoot
		\caption{Beschreibung der Felder in einem JSON-Objekt für ein Unterrichtsfach}
		\label{tab:rest:api:subjects:id:read:ret:json}
\endlastfoot 
\hline
			\textbf{Feldname} & \textbf{Datentyp} & \textbf{Beschreibung} \\ \hline
\endhead
id & STRING & Eindeutige Zeichenkette, die das Unterrichtsfach an einer Schule repräsentiert.  \\ \hline
displayName & STRING & Name des Unterrichtsfaches \\ \hline
referenceSubject & Array of Objects & OPTIONAL; Liste von Unterrichtsfächern und den Rollen, welcher der Benutzer in diesen innehat. Der Aufbau der Objekte kann \reftabllec{tab:rest:api:user:read:ret:subjects} entnommen werden. \\ \hline
school & Object & Objekt mit Informationen zu Schule zu der die Klasse gehört \\ \hline
school.id & STRING & ID der Schule, mit der unter /api/schools/\$id weitere Informationen abgefragt werden können. \\ \hline
school.displayName & STRING & OPTIONAL; Anzeigename der Schule \\ \hline
schoolPeriod & Object & Objekt mit Informationen zum Schuljahr, in welchem der Benutzer die Klasse besucht hat. \\ \hline
schoolPeriod.id & STRING & ID des Schuljahres, mit der unter /api/school-periods/\$id weitere Informationen abgefragt werden können. \\ \hline
schoolPeriod.displayName & STRING & OPTIONAL; Anzeigename des Schuljahres \\\hline
grade & List of STRINGs & Enthält eine Liste von Klassenstufen, die dieser Klasse zugeordnet sind. \\ \hline
users & Array of Objects & Liste von Rollen in Bezug auf ein Unterrichtsfach. Der Aufbau der Objekte kann \reftabllec{tab:rest:api:subjects:id:users:read:ret:json} entnommen werden. \\ \hline
\end{longtable}
\subsection{Endpunkt in der REST-API: /api/subjects/\$id/classes}
Die \reftabllec{tab:rest:api:subjects:id:classes:meth} listet auf, welche Operationen zugelassen sind und welche HTTP-Methoden dabei verwendet werden. 

\begin{table}[!htbp]
	\begin{tabular}{|c|c|c|}
		\hline
			\textbf{Operation} & \textbf{Zugelassen?} & \textbf{HTTP-Methode} \\ \hline
			CREATE & Nein & \\ \hline 
			READ & Ja & GET \\ \hline
			UPDATE & Nein & \\ \hline 
			DELETE & Nein & \\ \hline
	\end{tabular}

		\caption{Zugelassene Operationen auf /api/subjects/\$id/classes}
		\label{tab:rest:api:subjects:id:classes:meth}
\end{table}

\subsubsection{READ}
\label{sec:rest:api:subjects:id:classes:read}
Es sind nur Anfragen mit der HTTP-GET-Methode für ein READ auf die Daten zugelassen.
Diese Route gibt bei einer READ-Anfrage eine Liste von Klassen-IDs wieder, welche dem per ID ausgewählten Unterrichtsfach zugeordnet sind.
Für die Daten gilt, dass diese anhand des Kontextes des anfragenden Users Einschränkungen unterliegen.

Die Antwort erfolgt mit dem JSON-Objekt \reflisting{lst:code:rest:api:subjects:id:classes:read:ret}. 
Die einzelnen Felder der Antwort werden in \reftabllec{tab:rest:api:subjects:id:classes:read:ret:json} beschrieben.
Die Berechtigungen auf den Endpunkt können \reftabllec{tab:rest:api:subjects:id:classes:read:right} entnommen werden.

\lstset{
  language=json,
  tabsize=2,
  captionpos=b,
  numbers=left,
  commentstyle=\color{green},
  backgroundcolor=\color{white},
  numberstyle=\color{gray},
  keywordstyle=\color{blue} \textbf,%otherkeywords={xdata},
  keywordstyle=[2]\color{red}\textbf,
  identifierstyle=\color{black},
  stringstyle=\color{red}\ttfamily,
  basicstyle = \ttfamily \color{black} \footnotesize,
  showstringspaces=false,
	breakatwhitespace=false,         % sets if automatic breaks should only happen at whitespace
  breaklines=true, 
}
\begin{lstlisting}[caption={JSON-Antwort für einen GET-Aufruf der Route /api/subjects/\$id/classes},label={lst:code:rest:api:subjects:id:classes:read:ret},frame=tlrb]
[
 "<STRING>",
 ...
]
\end{lstlisting}
\begin{longtable}{|p{0.2\textwidth}|p{0.2\textwidth}|p{0.58\textwidth}|}
		\caption{Beschreibung der Zeichenkette in der JSON-Liste, welche alle Klasse die ein Unterrichtsfach haben umfasst}
\endfoot
		\caption{Beschreibung der Zeichenkette in der JSON-Liste, welche alle Klasse die ein Unterrichtsfach haben umfasst}
		\label{tab:rest:api:subjects:id:classes:read:ret:json}
\endlastfoot 
\hline
			\textbf{Feldname} & \textbf{Datentyp} & \textbf{Beschreibung} \\ \hline

\endhead
 & STRING & ID der Klasse wie dem Unterrichtsfach zugeordnet wurde, die Klasse kann damit unter /api/classes/\$id abgerufen werden. \\ \hline
\end{longtable}


\begin{longtable}{|c|p{0.7\textwidth}|}
\caption{Berechtigungen auf dem Endpunkt}
\endfoot
		\caption{Berechtigungen auf dem Endpunkt}
		\label{tab:rest:api:subjects:id:classes:read:right}
\endlastfoot
\hline
\textbf{Benutzergruppen} & \textbf{Zugelassene Daten} \\ \hline
\endhead
guest & Darf den Endpunkt nicht aufrufen und keine Daten vom Endpunkt erhalten. \\ \hline
user &  \\ \hline 
students & \\ \hline
external-students & \\ \hline
guardians & \\ \hline
teacher & \\ \hline
principal & \\ \hline
school-admin & \\ \hline
school-board & \\ \hline
fed-school-board & \\ \hline
sync-systems & \\ \hline
	\end{longtable}
%\subsection{Endpunkt in der REST-API: /api/subjects/\$id/schools}
Die \reftabllec{tab:rest:api:subjects:id:schools:meth} listet auf, welche Operationen zugelassen sind und welche HTTP-Methoden dabei verwendet werden. 

\begin{table}[!htbp]
	\begin{tabular}{|c|c|c|}
		\hline
			\textbf{Operation} & \textbf{Zugelassen?} & \textbf{HTTP-Methode} \\ \hline
			CREATE & Ja & POST \\ \hline 
			READ & Ja & GET \\ \hline
			UPDATE & Ja & POST \\ \hline 
			DELETE & Ja & POST \\ \hline
	\end{tabular}

		\caption{Zugelassene Operationen auf /api/subjects/\$id/schools}
		\label{tab:rest:api:subjects:id:schools:meth}
\end{table}

\subsubsection{READ}
\label{sec:rest:api:subjects:id:schools:read}
Es sind nur Anfragen mit der HTTP-GET-Methode für ein READ auf die Daten zugelassen.

Die Antwort erfolgt mit dem JSON-Objekt \reflisting{lst:code:rest:api:subjects:id:schools:read:ret}.
Die einzelnen Felder der Antwort werden in \reftabllec{tab:rest:api:subjects:id:schools:read:ret:json} beschrieben.
\lstset{
  language=json,
  tabsize=2,
  captionpos=b,
  numbers=left,
  commentstyle=\color{green},
  backgroundcolor=\color{white},
  numberstyle=\color{gray},
  keywordstyle=\color{blue} \textbf,%otherkeywords={xdata},
  keywordstyle=[2]\color{red}\textbf,
  identifierstyle=\color{black},
  stringstyle=\color{red}\ttfamily,
  basicstyle = \ttfamily \color{black} \footnotesize,
  showstringspaces=false,
	breakatwhitespace=false,         % sets if automatic breaks should only happen at whitespace
  breaklines=true, 
}
\begin{lstlisting}[caption={JSON-Antwort für einen GET-Aufruf des Pfads /api/subjects/\$id/schools},label={lst:code:rest:api:subjects:id:schools:read:ret},frame=tlrb]
[
{
	subject: "<STRING>",
	school: "<STRING>"
}
]
\end{lstlisting}
\begin{longtable}{|p{0.2\textwidth}|p{0.2\textwidth}|p{0.58\textwidth}|}
		\caption{Beschreibung der Felder in einem JSON-Objekt für das Zuordnen eines Benutzer in einer Rolle zu einer Schule}
\endfoot
		\caption{Beschreibung der Felder in einem JSON-Objekt für das Zuordnen eines Benutzer in einer Rolle zu einer Schule}
		\label{tab:rest:api:subjects:id:schools:read:ret:json}
\endlastfoot 
\hline
			\textbf{Feldname} & \textbf{Datentyp} & \textbf{Beschreibung} \\ \hline
\endhead
			 &  &  \\ \hline
\end{longtable}

% \subsection{Endpunkt in der REST-API: /api/subjects/\$id/students}
Die \reftabllec{tab:rest:api:subjects:id:students:meth} listet auf, welche Operationen zugelassen sind und welche HTTP-Methoden dabei verwendet werden. 

\begin{table}[!htbp]
	\begin{tabular}{|c|c|c|}
		\hline
			\textbf{Operation} & \textbf{Zugelassen?} & \textbf{HTTP-Methode} \\ \hline
			CREATE & Nein & \\ \hline 
			READ & Ja & GET \\ \hline
			UPDATE & Nein & \\ \hline 
			DELETE & Nein & \\ \hline
	\end{tabular}

		\caption{Zugelassene Operationen auf /api/subjects/\$id/students}
		\label{tab:rest:api:subjects:id:students:meth}
\end{table}

\subsubsection{READ}
\label{secrest:api:subjects:id:students:read}
Es sind nur Anfragen mit der HTTP-GET-Methode für ein READ auf die Daten zugelassen.
Eine READ-Anfrage auf diese Route gibt eine Liste mit Objekten, welche Schüler von wann bis wann an dem per ID ausgewählten Unterrichtsfach teilgenommen haben, wieder.
Die Daten werden durch den Kontext des anfragenden Benutzers eingeschränkt.

Die Antwort erfolgt mit dem JSON-Objekt \reflisting{lst:code:rest:api:subjects:id:students:read:ret}. 
Die einzelnen Felder der Antwort werden in Tabelle \reftabllec{tab:rest:api:subjects:id:students:read:ret:json} beschrieben.
Die Berechtigungen auf den Endpoint können \reftabllec{tab:rest:api:subjects:id:students:read:right} entnommen werden.
\lstset{
  language=json,
  tabsize=2,
  captionpos=b,
  numbers=left,
  commentstyle=\color{green},
  backgroundcolor=\color{white},
  numberstyle=\color{gray},
  keywordstyle=\color{blue} \textbf,%otherkeywords={xdata},
  keywordstyle=[2]\color{red}\textbf,
  identifierstyle=\color{black},
  stringstyle=\color{red}\ttfamily,
  basicstyle = \ttfamily \color{black} \footnotesize,
  showstringspaces=false,
	breakatwhitespace=false,         % sets if automatic breaks should only happen at whitespace
  breaklines=true, 
}
\begin{lstlisting}[caption={JSON-Antwort für einen GET-Aufruf des Pfads /api/subjects/\$id/students},label={lst:code:rest:api:subjects:id:students:read:ret},frame=tlrb]
[
 {
  subject: "<STRING>", 
  user: "<STRING>",
	start: "<DATE>",
	end: "<DATE>",
 },
 ...
]
\end{lstlisting}

\begin{longtable}{|p{0.2\textwidth}|p{0.2\textwidth}|p{0.58\textwidth}|}
		\caption{Beschreibung der Felder in einem JSON-Objekt mit der Liste aller Schüler eines Unterrichtsfaches.}
\endfoot
		\caption{Beschreibung der Felder in einem JSON-Objekt mit in der Liste aller Schüler eines Unterrichtsfaches.}
		\label{tab:rest:api:subjects:id:students:read:ret:json}
\endlastfoot 
\hline
			\textbf{Feldname} & \textbf{Datentyp} & \textbf{Beschreibung} \\ \hline
\endhead
subject & STRING & Zeichenkette, mit der das Unterrichtsfach unter /api/subjects/\$id abgefragt werden kann. \\ \hline
user & STRING &  Zeichenkette, mit der User unter /api/users/\$id abfragt werden kann. \\ \hline
start & DATE & Optional; wird gesetzt, falls der Schüler nach Start des Zeitraumes eines Unterrichtsfaches dem Unterrichtsfach hinzugefügt wurde. \\ \hline
end & DATE & Optional; wird gesetzt, falls ein Schüler vor Ablauf des Zeitraumes eines Unterrichtsfaches aus dem Unterrichtsfach entfernt wurde. \\ \hline
\end{longtable}

% \subsection{Endpunkt in der REST-API: /api/subjects/\$id/teachers}
Die \reftabllec{tab:rest:api:subjects:id:teachers:meth} listet auf, welche Operationen zugelassen sind und welche HTTP-Methoden dabei verwendet werden. 

\begin{table}[!htbp]
	\begin{tabular}{|c|c|c|}
		\hline
			\textbf{Operation} & \textbf{Zugelassen?} & \textbf{HTTP-Methode} \\ \hline
			CREATE & Nein & \\ \hline 
			READ & Ja & GET \\ \hline
			UPDATE & Nein & \\ \hline 
			DELETE & Nein & \\ \hline
	\end{tabular}

		\caption{Zugelassene Operationen auf /api/subjects/\$id/teachers}
		\label{tab:rest:api:subjects:id:teachers:meth}
\end{table}

\subsubsection{READ}
\label{secrest:api:subjects:id:teachers:read}
Es sind nur Anfragen mit der HTTP-GET-Methode für ein READ auf die Daten zugelassen.
Bei einer READ-Abfrage auf diesen Pfad wird eine Liste von Objekten mit Metainformationen zu den unterrichtenden Lehrkräften des per ID ausgewählten Unterrichtsfaches zurückgegeben.
Für die Daten gilt, das diese anhand des Kontextes des anfragenden Users Einschränkungen unterliegen.

Die Antwort erfolgt mit dem JSON-Objekt \reflisting{lst:code:rest:api:subjects:id:teachers:read:ret}. 
Die einzelnen Felder der Antwort werden in Tabelle \reftabllec{tab:rest:api:subjects:id:teachers:read:ret:json} beschrieben.
\lstset{
  language=json,
  tabsize=2,
  captionpos=b,
  numbers=left,
  commentstyle=\color{green},
  backgroundcolor=\color{white},
  numberstyle=\color{gray},
  keywordstyle=\color{blue} \textbf,%otherkeywords={xdata},
  keywordstyle=[2]\color{red}\textbf,
  identifierstyle=\color{black},
  stringstyle=\color{red}\ttfamily,
  basicstyle = \ttfamily \color{black} \footnotesize,
  showstringspaces=false,
	breakatwhitespace=false,         % sets if automatic breaks should only happen at whitespace
  breaklines=true, 
}
\begin{lstlisting}[caption={JSON-Antwort für einen GET-Aufruf des Pfads /api/subjects/\$id/teachers},label={lst:code:rest:api:subjects:id:teachers:read:ret},frame=tlrb]
[
{ 
	subject: "<STRING>",
	user: "<STRING>",
	start: "<DATE>",
	end: "<DATE>"
},
...
]
\end{lstlisting}
\begin{longtable}{|p{0.2\textwidth}|p{0.2\textwidth}|p{0.58\textwidth}|}
		\caption{Beschreibung der Felder in einem JSON-Objekt mit der Liste aller unterrichtenden Lehrkräfte eines Unterrichtsfaches.}
\endfoot
		\caption{Beschreibung der Felder in einem JSON-Objekt mit der Liste aller unterrichtenden Lehrkräfte eines Unterrichtsfaches.}
		\label{tab:rest:api:subjects:id:teachers:read:ret:json}
\endlastfoot 
\hline
			\textbf{Feldname} & \textbf{Datentyp} & \textbf{Beschreibung} \\ \hline
\endhead
subject & STRING & Zeichenkette, mit der das Unterrichtsfach unter /api/subjects/\$id abgefragt werden kann. \\ \hline
user & STRING &  Zeichenkette, mit der die Lehrkraft unter /api/users/\$id abfragt werden kann. \\ \hline
start & DATE & Optional; Eine Datumsangabe nach ISO-8601 im Format YYYY-MM-DD. Wird gesetzt, falls die Lehrkraft ach Start des Zeitraumes eines Unterrichtsfaches dem Unterrichtsfach hinzugefügt wurde. \\ \hline
end & DATE & Optional; Eine Datumsangabe nach ISO-8601 im Format YYYY-MM-DD. Wird gesetzt, falls die Lehrkräften vor Ablauf des Zeitraumes eines Unterrichtsfaches aus dem Unterrichtsfach entfernt wurde. \\ \hline
\end{longtable}

\subsection{Endpunkt in der REST-API: /api/subjects/\$id/users}
Die \reftabllec{tab:rest:api:subjects:id:users:meth} listet auf, welche Operationen zugelassen sind und welche HTTP-Methoden dabei verwendet werden. 

\begin{table}[!htbp]
	\begin{tabular}{|c|c|c|}
		\hline
			\textbf{Operation} & \textbf{Zugelassen?} & \textbf{HTTP-Methode} \\ \hline
			CREATE & Nein & \\ \hline 
			READ & Ja & GET \\ \hline
			UPDATE & Nein & \\ \hline 
			DELETE & Nein & \\ \hline
	\end{tabular}

		\caption{Zugelassene Operationen auf /api/subjects/\$id/users}
		\label{tab:rest:api:subjects:id:users:meth}
\end{table}

\subsubsection{READ}
\label{secrest:api:subjects:id:users:read}
Es sind nur Anfragen mit der HTTP-GET-Methode für ein READ auf die Daten zugelassen.
Bei einer READ-Abfrage auf diesen Pfad wird eine Liste von Objekten mit Metainformationen zu Benutzern und ihrer Rolle innerhalb eines Unterrichtsfaches zurückgegeben.

Die Antwort erfolgt mit dem JSON-Objekt \reflisting{lst:code:rest:api:subjects:id:users:read:ret}. 
Die einzelnen Felder der Antwort werden in Tabelle \reftabllec{tab:rest:api:subjects:id:users:read:ret:json} beschrieben.
\lstset{
  language=json,
  tabsize=2,
  captionpos=b,
  numbers=left,
  commentstyle=\color{green},
  backgroundcolor=\color{white},
  numberstyle=\color{gray},
  keywordstyle=\color{blue} \textbf,%otherkeywords={xdata},
  keywordstyle=[2]\color{red}\textbf,
  identifierstyle=\color{black},
  stringstyle=\color{red}\ttfamily,
  basicstyle = \ttfamily \color{black} \footnotesize,
  showstringspaces=false,
	breakatwhitespace=false,         % sets if automatic breaks should only happen at whitespace
  breaklines=true, 
}
\begin{lstlisting}[caption={JSON-Antwort für einen GET-Aufruf des Pfads /api/subjects/\$id/users},label={lst:code:rest:api:subjects:id:users:read:ret},frame=tlrb]
[
{
    id: "<STRING>",
    displayName: "<STRING>",
    roleWithinSubject: <STRING>",
    start: "<DATE>",
    end: "<DATE>"    
},
...
]
\end{lstlisting}
\begin{longtable}{|p{0.2\textwidth}|p{0.2\textwidth}|p{0.58\textwidth}|}
		\caption{Beschreibung der Felder in einem JSON-Objekt mit der Liste aller Benutzer und ihrer Rolle innerhalb eines Unterrichtsfaches.}
\endfoot
		\caption{Beschreibung der Felder in einem JSON-Objekt mit der Liste aller Benutzer und ihrer Rolle innerhalb eines Unterrichtsfaches.}
		\label{tab:rest:api:subjects:id:users:read:ret:json}
\endlastfoot 
\hline
			\textbf{Feldname} & \textbf{Datentyp} & \textbf{Beschreibung} \\ \hline
\endhead
id & STRING & Zeichenkette, mit der das Unterrichtsfach unter /api/subjects/\$id abgefragt werden kann. \\ \hline
displayName & STRING &  Zeichenkette, mit der die Lehrkraft unter /api/users/\$id abfragt werden kann. \\ \hline
roleWithinSubject & STRING & Rolle des Benutzer innerhalb des Unterrichtsfaches. \\ \hline
start & DATE & Optional; Eine Datumsangabe nach ISO-8601 im Format YYYY-MM-DD. Wird gesetzt, falls der Benutzer nach Start des Zeitraumes eines Unterrichtsfaches dem Unterrichtsfach hinzugefügt wurde. \\ \hline
end & DATE & Optional; Eine Datumsangabe nach ISO-8601 im Format YYYY-MM-DD. Wird gesetzt, falls der Benutzer vor Ablauf des Zeitraumes eines Unterrichtsfaches aus dem Unterrichtsfach entfernt wurde. \\ \hline
\end{longtable}

% \subsection{Endpunkt in der REST-API: /api/subjects/\$id/timetable}
Die \reftabllec{tab:rest:api:subjects:id:timetable:meth} listet auf, welche Operationen zugelassen sind und welche HTTP-Methoden dabei verwendet werden. 

\begin{table}[!htbp]
	\begin{tabular}{|c|c|c|}
		\hline
			\textbf{Operation} & \textbf{Zugelassen?} & \textbf{HTTP-Methode} \\ \hline
			CREATE & Nein & \\ \hline 
			READ & Ja & GET \\ \hline
			UPDATE & Nein & \\ \hline 
			DELETE & Nein & \\ \hline
	\end{tabular}

		\caption{Zugelassene Operationen auf /api/subjects/\$id/timetable}
		\label{tab:rest:api:subjects:id:timetable:meth}
\end{table}

\subsubsection{READ}
\label{secrest:api:subjects:id:timetable:read}
Es sind nur Anfragen mit der HTTP-GET-Methode für ein READ auf die Daten zugelassen.
Es werden für das per ID ausgewählte Unterrichtsfach die Daten zu für den Stundenplan ausgegeben.
Die Daten sind durch den Kontext des anfragenden Users eingeschränkt.

Die Antwort erfolgt mit dem JSON-Objekt \reflisting{lst:code:rest:api:subjects:id:timetable:read:ret}. 
Die einzelnen Felder der Antwort werden in Tabelle \reftabllec{tab:rest:api:subjects:id:timetable:read:ret:json} beschrieben.
Die Berechtigungen auf den Endpoint können \reftabllec{tab:rest:api:subjects:id:timetable:read:right} entnommen werden.
\lstset{
  language=json,
  tabsize=2,
  captionpos=b,
  numbers=left,
  commentstyle=\color{green},
  backgroundcolor=\color{white},
  numberstyle=\color{gray},
  keywordstyle=\color{blue} \textbf,%otherkeywords={xdata},
  keywordstyle=[2]\color{red}\textbf,
  identifierstyle=\color{black},
  stringstyle=\color{red}\ttfamily,
  basicstyle = \ttfamily \color{black} \footnotesize,
  showstringspaces=false,
	breakatwhitespace=false,         % sets if automatic breaks should only happen at whitespace
  breaklines=true, 
}
\begin{lstlisting}[caption={JSON-Antwort für einen GET-Aufruf der Route /api/subjects/\$id/timetable},label={lst:code:rest:api:subjects:id:timetable:read:ret},frame=tlrb]
[
 {
  subject: "<STRING>",
	day: "<ENUM>",
	start: "<TIME>",
	end: "<TIME>",
	repeat: "<ENUM>",
	date: "<CALENDARDATE>",
	week: "<ENUM>",
 },
 ...
]
\end{lstlisting}
\begin{longtable}{|p{0.2\textwidth}|p{0.2\textwidth}|p{0.58\textwidth}|}
		\caption{Beschreibung der Felder in einem JSON-Objekt für den Stundenplan eines Schulfachs}
\endfoot
		\caption{Beschreibung der Felder in einem JSON-Objekt für den Stundenplan eines Schulfachs}
		\label{tab:rest:api:subjects:id:timetable:read:ret:json}
\endlastfoot 
\hline
			\textbf{Feldname} & \textbf{Datentyp} & \textbf{Beschreibung} \\ \hline
\endhead
			 &  &  \\ \hline
\end{longtable}


\begin{longtable}{|c|p{0.7\textwidth}|}
\caption{Berechtigungen auf dem Endpunkt}
\endfoot
		\caption{Berechtigungen auf dem Endpunkt}
		\label{tab:rest:api:subjects:id:timetable:read:right}
\endlastfoot
\hline
\textbf{Benutzergruppen} & \textbf{Zugelassene Daten} \\ \hline
\endhead
guest & Darf den Endpunkt nicht aufrufen und keine Daten vom Endpunkt erhalten. \\ \hline
user &  \\ \hline 
students & \\ \hline
external-students & \\ \hline
guardians & \\ \hline
teacher & \\ \hline
principal & \\ \hline
school-admin & \\ \hline
school-board & \\ \hline
fed-school-board & \\ \hline
sync-systems & \\ \hline
	\end{longtable}

