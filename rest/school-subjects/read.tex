\subsubsection{READ}
\label{sec:rest:api:school-subjects:read}
Es sind nur Anfragen mit der HTTP-GET-Methode für ein READ auf die Daten zugelassen.
Bei einer lesenden Anfrage wird eine Liste der Teilmenge der Referenzschulfächer, welche vom IDM unterstützt werden, zurückgegeben.

Die Antwort erfolgt mit dem JSON-Objekt \reflisting{lst:code:rest:api:school-subjects:read:ret}. Die einzelnen Felder der Antwort werden in \reftabllec{tab:rest:api:school-subjects:read:ret:json} beschrieben.
Die Berechtigungen auf den Endpunkt können \reftabllec{tab:rest:api:school-subjects:read:right} entnommen werden.
\lstset{
  language=json,
  tabsize=2,
  captionpos=b,
  numbers=left,
  commentstyle=\color{green},
  backgroundcolor=\color{white},
  numberstyle=\color{gray},
  keywordstyle=\color{blue} \textbf,%otherkeywords={xdata},
  keywordstyle=[2]\color{red}\textbf,
  identifierstyle=\color{black},
  stringstyle=\color{red}\ttfamily,
  basicstyle = \ttfamily \color{black} \footnotesize,
  showstringspaces=false,
	breakatwhitespace=false,         % sets if automatic breaks should only happen at whitespace
  breaklines=true, 
}
\begin{lstlisting}[caption={JSON-Antwort für einen GET-Aufruf der Route /api/school-subjects},label={lst:code:rest:api:school-subjects:read:ret},frame=tlrb]
[
{
 id: "<STRING>",
 short_name: "<STRING>",
 name: "<STRING>"
},
...
]
\end{lstlisting}
\begin{table}[!htb]
	\begin{tabularx}{\textwidth}{|c|c|X|}
		\hline
			\textbf{Feldname} & \textbf{Datentyp} & \textbf{Beschreibung} \\ \hline
			id & STRING & Eine eindeutige Zeichenkette, die vom IDM-Provider vergeben wird. Sie darf nur aus alphanumerischen Zeichen und Bindestrich bestehen.\\ \hline
			name & STRING & Der ausgeschriebene Name eines Schulfaches \\ \hline
	\end{tabularx}

		\caption{Beschreibung der Felder in einem JSON-Objekt für ein Schulfach}
		\label{tab:rest:api:school-subjects:read:ret:json}
\end{table}
\begin{longtable}{|c|p{0.7\textwidth}|}
\caption{Berechtigungen auf dem Endpunkt}
\endfoot
		\caption{Berechtigungen auf dem Endpunkt}
		\label{tab:rest:api:school-subjects:read:right}
\endlastfoot 
\hline
\textbf{Benutzergruppen} & \textbf{Zugelassene Daten} \\ \hline
\endhead
guest & Darf den Endpunkt nicht aufrufen und keine Daten vom Endpunkt erhalten. \\ \hline
user & Darf den Endpunkt mit GET aufrufen und Endpunkt gibt alle im System vorhandenen Schulfächer in einer Liste zurück. \\ \hline
students &  \\ \hline
external-students &  \\ \hline
guardians &  \\ \hline
teacher &  \\ \hline
principal &  \\ \hline
school-admin &  \\ \hline
school-board &  \\ \hline
fed-school-board &  \\ \hline
sync-systems &  \\ \hline
	\end{longtable}
