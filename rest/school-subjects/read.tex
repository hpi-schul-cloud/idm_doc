\subsubsection{READ}
\label{sec:rest:api:school-subjects:read}
Es sind nur Anfragen mit der HTTP-GET-Methode für ein READ auf die Daten zugelassen.
Unter diesem Pfad wird eine Liste der Teilmenge der Referenzschulfächer, welche vom IDM unterstützt werden, zurückgegeben.

Die Antwort erfolgt mit dem JSON-Objekt \reflisting{lst:code:rest:api:school-subjects:read:ret}. Die einzelnen Felder der Antwort werden in \reftabllec{tab:rest:api:school-subjects:read:ret:json} beschrieben.
\lstset{
  language=json,
  tabsize=2,
  captionpos=b,
  numbers=left,
  commentstyle=\color{green},
  backgroundcolor=\color{white},
  numberstyle=\color{gray},
  keywordstyle=\color{blue} \textbf,%otherkeywords={xdata},
  keywordstyle=[2]\color{red}\textbf,
  identifierstyle=\color{black},
  stringstyle=\color{red}\ttfamily,
  basicstyle = \ttfamily \color{black} \footnotesize,
  showstringspaces=false,
	breakatwhitespace=false,         % sets if automatic breaks should only happen at whitespace
  breaklines=true, 
}
\begin{lstlisting}[caption={JSON-Antwort für einen GET-Aufruf des Pfads /api/school-subjects},label={lst:code:rest:api:school-subjects:read:ret},frame=tlrb]
[
{
    id: "<STRING>",
    short-name: "<STRING>",
    display-name: "<STRING>"
},
...
]
\end{lstlisting}

\begin{longtable}{|p{0.16\textwidth}|p{0.16\textwidth}|p{0.58\textwidth}|}
		\caption{Beschreibung der Felder in einem JSON-Objekt für ein Schulfach}
\endfoot
		\caption{Beschreibung der Felder in einem JSON-Objekt für ein Schulfach}
		\label{tab:rest:api:school-subjects:read:ret:json}
\endlastfoot 
\hline
			\textbf{Feldname} & \textbf{Datentyp} & \textbf{Beschreibung} \\ \hline
\endhead
 id & STRING & Eine eindeutige Zeichenkette, die vom IDM-Provider vergeben wird. Sie darf nur aus alphanumerischen Zeichen und Bindestrich bestehen.\\ \hline
 short-name & STRING & Die Kurzbezeichnung eines Schulfaches \\ \hline
 display-name & STRING & Der ausgeschriebene Name bzw. Anzeigename eines Schulfaches \\ \hline
\end{longtable}
