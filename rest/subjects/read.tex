\subsubsection{READ}
\label{sec:end:rest:api:subjects:read}
Es sind nur Anfragen mit der HTTP-GET-Methode für ein READ auf die Daten zugelassen.
Es werden nur die IDs der Daten präsentiert, die für den anfragendem Benutzer in seinem Kontext existieren.

Die Antwort erfolgt mit dem JSON-Objekt \reflisting{lst:code:rest:api:subjects:read:ret}. 
Die einzelnen Felder der Antwort werden in \reftabllec{tab:rest:api:subjects:read:ret} beschrieben.

\lstset{
  language=json,
  tabsize=2,
  captionpos=b,
  numbers=left,
  commentstyle=\color{green},
  backgroundcolor=\color{white},
  numberstyle=\color{gray},
  keywordstyle=\color{blue} \textbf,%otherkeywords={xdata},
  keywordstyle=[2]\color{red}\textbf,
  identifierstyle=\color{black},
  stringstyle=\color{red}\ttfamily,
  basicstyle = \ttfamily \color{black} \footnotesize,
  showstringspaces=false,
	breakatwhitespace=false,         % sets if automatic breaks should only happen at whitespace
  breaklines=true, 
}
\begin{lstlisting}[caption={JSON-Antwort für einen GET-Aufruf des Pfads /api/subjects},label={lst:code:rest:api:subjects:read:ret},frame=tlrb]
[
 "<STRING>",
 ...
]
\end{lstlisting}


\begin{longtable}{|p{0.16\textwidth}|p{0.16\textwidth}|p{0.58\textwidth}|}
		\caption{Beschreibung der Zeichenkete in der JSON Liste für Unterrichtsfächer.}
\endfoot
		\caption{Beschreibung der Zeichenkete in der JSON Liste für Unterrichtsfächer.}
		\label{tab:rest:api:subjects:read:ret}
\endlastfoot 
\hline
			\textbf{Feldname} & \textbf{Datentyp} & \textbf{Beschreibung} \\ \hline
\endhead
 & STRING & Zeichenkette, mit der weitere Informationen zu einem Unterrichtsfach unter /api/subjects/\$id abgefragt werden kann. \\ \hline
\end{longtable}
