\subsection{Endpunkt in der REST-API: /api/subjects}
\label{sec:end:rest:api:schools-periods}
An diesem Endpunkt werden Informationen zu Unterrichtsfächern verarbeitet. Ein Unterrichtsfach beschreibt die regelmäßige Wissensvermittlung durch eine Lehrperson an einer Schule (s. \refabschnitt{sec:end:rest:api:schools}) in Bezug auf eine Klasse/Klassenstufe (s. \refabschnitt{sec:end:rest:api:classes}) innerhalb einer schulischen Periode (s. \refabschnitt{sec:end:rest:api:schools-periods}). Dabei wird ein bestimmtes Thema abgedeckt, das durch das Referenzschulfach (s. \refabschnitt{sec:end:rest:api:reference-subjects}) angegeben werden kann.

Die \reftabllec{tab:end:rest:api:subjects:meth} listet auf, welche Operationen zugelassen sind und welche HTTP-Methoden dabei verwendet werden. 

\begin{table}[!htbp]
	\begin{tabular}{|c|c|c|}
		\hline
			\textbf{Operation} & \textbf{Zugelassen?} & \textbf{HTTP-Methode} \\ \hline
			CREATE & Nein &  \\ \hline 
			READ & Ja & GET \\ \hline
			UPDATE & Nein & \\ \hline 
			DELETE & Nein & \\ \hline
	\end{tabular}

		\caption{Zugelassene Operationen auf /api/subjects}
		\label{tab:end:rest:api:subjects:meth}
\end{table}

\subsubsection{READ}
\label{sec:end:rest:api:subjects:read}
Es sind nur Anfragen mit der HTTP-GET-Methode für ein READ auf die Daten zugelassen.
Es werden nur die IDs der Daten präsentiert, die für den anfragendem Benutzer in seinem Kontext existieren.

Die Antwort erfolgt mit dem JSON-Objekt \reflisting{lst:code:rest:api:subjects:read:ret}. 
Die einzelnen Felder der Antwort werden in \reftabllec{tab:rest:api:subjects:read:ret} beschrieben.
Die Berechtigungen auf den Endpunkt können \reftabllec{tab:rest:api:subjects:read:right} entnommen werden.

\lstset{
  language=json,
  tabsize=2,
  captionpos=b,
  numbers=left,
  commentstyle=\color{green},
  backgroundcolor=\color{white},
  numberstyle=\color{gray},
  keywordstyle=\color{blue} \textbf,%otherkeywords={xdata},
  keywordstyle=[2]\color{red}\textbf,
  identifierstyle=\color{black},
  stringstyle=\color{red}\ttfamily,
  basicstyle = \ttfamily \color{black} \footnotesize,
  showstringspaces=false,
	breakatwhitespace=false,         % sets if automatic breaks should only happen at whitespace
  breaklines=true, 
}
\begin{lstlisting}[caption={JSON-Antwort für einen GET-Aufruf des Pfads /api/subjects},label={lst:code:rest:api:subjects:read:ret},frame=tlrb]
[
 "<STRING>",
 ...
]
\end{lstlisting}


\begin{longtable}{|p{0.16\textwidth}|p{0.16\textwidth}|p{0.58\textwidth}|}
		\caption{Beschreibung der Zeichenkete in der JSON Liste für Unterrichtsfächer.}
\endfoot
		\caption{Beschreibung der Zeichenkete in der JSON Liste für Unterrichtsfächer.}
		\label{tab:rest:api:subjects:read:ret}
\endlastfoot 
\hline
			\textbf{Feldname} & \textbf{Datentyp} & \textbf{Beschreibung} \\ \hline
\endhead
 & STRING & Zeichenkette, mit der weitere Informationen zu einem Unterrichtsfach unter /api/subjects/\$id abgefragt werden kann. \\ \hline
\end{longtable}