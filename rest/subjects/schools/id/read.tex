\subsubsection{READ}
\label{sec:end:rest:api:school-subjects:read}
Es sind nur Anfragen mit der HTTP-GET-Methode für ein READ auf die Daten zugelassen.
Die Antwort erfolgt mit dem JSON-Objekt \reflisting{lst:code:end:rest:api:school-subjects:read:ret}. Die einzelnen Felder der Antwort werden in Tabelle \reftabllec{tab:end:rest:api:school-subjects:read:ret:json} beschrieben.
Die Berechtigungen auf den Endpoint können \reftabllec{tab:end:rest:api:school-subjects:read:right} entnommen werden.
\lstset{
  language=json,
  tabsize=2,
  captionpos=b,
  numbers=left,
  commentstyle=\color{green},
  backgroundcolor=\color{white},
  numberstyle=\color{gray},
  keywordstyle=\color{blue} \textbf,%otherkeywords={xdata},
  keywordstyle=[2]\color{red}\textbf,
  identifierstyle=\color{black},
  stringstyle=\color{red}\ttfamily,
  basicstyle = \ttfamily \color{black} \footnotesize,
  showstringspaces=false,
	breakatwhitespace=false,         % sets if automatic breaks should only happen at whitespace
  breaklines=true, 
}
\begin{lstlisting}[caption={JSON-Antwort für einen GET-Aufruf der Route /api/school-subjects},label={lst:code:end:rest:api:school-subjects:read:ret},frame=tlrb]
[
{
 id: "<STRING>",
 name: "<STRING>"
},
...
{
 id: "<STRING>",
 name: "<STRING>"
}
]
\end{lstlisting}
\begin{table}[htb]
	\begin{tabularx}{\textwidth}{|c|c|X|}
		\hline
			\textbf{Feldname} & \textbf{Datentyp} & \textbf{Beschreibung} \\ \hline
			%	id & STRING & Eine eindeutige Zeichenkette, die vom IDM Provider vergeben wird. Sie darf nur aus alphanumerischen Zeichen und Bindestrichen bestehen. \\ \hline
			%	start & TIMESTAMP & Ein Datumsangabe nach ISO-8601 in dem Format YYYY-MM-DD \\ \hline
			%	ende & TIMESTAMP & Ein Datumsangabe nach ISO-8601 in dem Format YYYY-MM-DD \\ \hline
			% name & STRING & Eine Name für das Schuljahr wie z.b.: 2020-2021 \\ \hline
			id & STRING & Eine eindeutige Zeichenkette, die vom IDM-Provider vergeben wird. Sie darf nur aus alphanumerischen Zeichen und Bindestrich bestehen.\\ \hline
			name & STRING & Der Ausgeschrieben name eines Schulfaches \\ \hline
	\end{tabularx}

		\caption{Beschreibung der Felder in eine JSON-Objekt für ein Schulfach}
		\label{tab:end:rest:api:school-subjects:read:ret:json}
\end{table}
\begin{table}[htb]
	\begin{tabularx}{\textwidth}{|c|X|}
		\hline
\textbf{Benutzergruppen} & \textbf{Zugelassene Daten} \\ \hline
guest & Darf den Endpunkt nicht aufrufen und keine Daten vom Endpunkt erhalten. \\ \hline
students & Darf den Endpunkt mit GET aufrufen und Endpunkt gibt alle im System vorhandenen Schulfächer in einer Liste zurück. \\ \hline
parents & Darf den Endpunkt mit GET aufrufen und Endpunkt gibt alle im System vorhandenen Schulfächer in einer Liste zurück. \\ \hline
teacher & Darf den Endpunkt mit GET aufrufen und Endpunkt gibt alle im System vorhandenen Schulfächer in einer Liste zurück. \\ \hline
principal & Darf den Endpunkt mit GET aufrufen und Endpunkt gibt alle im System vorhandenen Schulfächer in einer Liste zurück. \\ \hline
school-admin & Darf den Endpunkt mit GET aufrufen und Endpunkt gibt alle im System vorhandenen Schulfächer in einer Liste zurück. \\ \hline
school-board & Darf den Endpunkt mit GET aufrufen und Endpunkt gibt alle im System vorhandenen Schulfächer in einer Liste zurück. \\ \hline
fed-school-board & Darf den Endpunkt mit GET aufrufen und Endpunkt gibt alle im System vorhandenen Schulfächer in einer Liste zurück. \\ \hline
sync-systems & Darf den Endpunkt mit GET aufrufen und Endpunkt gibt alle im System vorhandenen Schulfächer in einer Liste zurück. \\ \hline

	\end{tabularx}

		\caption{Berechtigungen auf dem Endpunkt}
		\label{tab:end:rest:api:school-subjects:read:right}
\end{table}