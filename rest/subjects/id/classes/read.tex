\subsubsection{READ}
\label{sec:rest:api:subjects:id:classes:read}
Es sind nur Anfragen mit der HTTP-GET-Methode für ein READ auf die Daten zugelassen.
Diese Route gibt bei einer READ-Anfrage eine Liste von Klassen-IDs wieder, welche dem per ID ausgewählten Unterrichtsfach zugeordnet sind.
Für die Daten gilt, dass diese anhand des Kontextes des anfragenden Users Einschränkungen unterliegen.

Die Antwort erfolgt mit dem JSON-Objekt \reflisting{lst:code:rest:api:subjects:id:classes:read:ret}. 
Die einzelnen Felder der Antwort werden in \reftabllec{tab:rest:api:subjects:id:classes:read:ret:json} beschrieben.

\lstset{
  language=json,
  tabsize=2,
  captionpos=b,
  numbers=left,
  commentstyle=\color{green},
  backgroundcolor=\color{white},
  numberstyle=\color{gray},
  keywordstyle=\color{blue} \textbf,%otherkeywords={xdata},
  keywordstyle=[2]\color{red}\textbf,
  identifierstyle=\color{black},
  stringstyle=\color{red}\ttfamily,
  basicstyle = \ttfamily \color{black} \footnotesize,
  showstringspaces=false,
	breakatwhitespace=false,         % sets if automatic breaks should only happen at whitespace
  breaklines=true, 
}
\begin{lstlisting}[caption={JSON-Antwort für einen GET-Aufruf des Pfads /api/subjects/\$id/classes},label={lst:code:rest:api:subjects:id:classes:read:ret},frame=tlrb]
[
 "<STRING>",
 ...
]
\end{lstlisting}
\begin{longtable}{|p{0.2\textwidth}|p{0.2\textwidth}|p{0.58\textwidth}|}
		\caption{Beschreibung der Zeichenkette in der JSON-Liste, welche alle Klassen, die ein Unterrichtsfach haben, umfasst}
\endfoot
		\caption{Beschreibung der Zeichenkette in der JSON-Liste, welche alle Klassen, die ein Unterrichtsfach haben, umfasst}
		\label{tab:rest:api:subjects:id:classes:read:ret:json}
\endlastfoot 
\hline
			\textbf{Feldname} & \textbf{Datentyp} & \textbf{Beschreibung} \\ \hline

\endhead
 & STRING & ID der Klasse, die dem Unterrichtsfach zugeordnet wurde. Die Klasse kann damit unter /api/classes/\$id abgerufen werden. \\ \hline
\end{longtable}
