\subsubsection{READ}
\label{secrest:api:subjects:id:students:read}
Es sind nur Anfragen mit der HTTP-GET-Methode für ein READ auf die Daten zugelassen.
Eine READ-Anfrage auf diese Route gibt eine Liste mit Objekten, welche Schüler von wann bis wann an dem per ID ausgewählten Unterrichtsfach teilgenommen haben, wieder.
Die Daten werden durch den Kontext des anfragenden Benutzers eingeschränkt.

Die Antwort erfolgt mit dem JSON-Objekt \reflisting{lst:code:rest:api:subjects:id:students:read:ret}. 
Die einzelnen Felder der Antwort werden in Tabelle \reftabllec{tab:rest:api:subjects:id:students:read:ret:json} beschrieben.
Die Berechtigungen auf den Endpoint können \reftabllec{tab:rest:api:subjects:id:students:read:right} entnommen werden.
\lstset{
  language=json,
  tabsize=2,
  captionpos=b,
  numbers=left,
  commentstyle=\color{green},
  backgroundcolor=\color{white},
  numberstyle=\color{gray},
  keywordstyle=\color{blue} \textbf,%otherkeywords={xdata},
  keywordstyle=[2]\color{red}\textbf,
  identifierstyle=\color{black},
  stringstyle=\color{red}\ttfamily,
  basicstyle = \ttfamily \color{black} \footnotesize,
  showstringspaces=false,
	breakatwhitespace=false,         % sets if automatic breaks should only happen at whitespace
  breaklines=true, 
}
\begin{lstlisting}[caption={JSON-Antwort für einen GET-Aufruf der Route /api/subjects/\$id/students},label={lst:code:rest:api:subjects:id:students:read:ret},frame=tlrb]
[
 {
  subject: "<STRING>", 
  user: "<STRING>",
	start: "<CALENDARDATE>",
	end: "<CALENDARDATE>",
 },
 ...
]
\end{lstlisting}

\begin{longtable}{|p{0.2\textwidth}|p{0.2\textwidth}|p{0.58\textwidth}|}
		\caption{Beschreibung der Felder in eine JSON-Objekt welches in der Liste aller Schüler eines Unterrichtsfaches.}
\endfoot
		\caption{Beschreibung der Felder in eine JSON-Objekt welches in der Liste aller Schüler eines Unterrichtsfaches.}
		\label{tab:rest:api:subjects:id:students:read:ret:json}
\endlastfoot 
\hline
			\textbf{Feldname} & \textbf{Datentyp} & \textbf{Beschreibung} \\ \hline
\endhead
subject & STRING & Zeichenkette mit der das Unterrichtsfach unter /api/subjects/\$id abgefragt werden kann\\ \hline
user & STRING &  Zeichenkette mit der User unter /api/users/\$id abfragt werden kann \\ \hline
start & CALENDARDATE & Optional; wenn der Schüler zu einen Späteren Zeitpunkt als ab wan das Unterrichtsfach Startet den Unterrichtsfach hinzugefügt wurde.\\ \hline
end & CALENDARDATE & Optional; wenn ein Schüler vor Ablauf des Zeitraumes eines Unterrichtsfaches aus dem Unterrichtsfach entfernt wurde.\\ \hline
\end{longtable}