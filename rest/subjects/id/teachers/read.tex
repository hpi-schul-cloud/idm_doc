\subsubsection{READ}
\label{secrest:api:subjects:id:teachers:read}
Es sind nur Anfragen mit der HTTP-GET-Methode für ein READ auf die Daten zugelassen.
Bei einer READ-Abnfrage auf diese Route wird eine Liste von Objekten mit Metainformationen zu den unterrichtenden Lehrkräften des per ID ausgewählten Unterrichtsfaches zurückgegeben.
Für die Daten gilt, dasd diese anhand des Kontextes des anfragenden Users Einschränkungen unterliegen.

Die Antwort erfolgt mit dem JSON-Objekt \reflisting{lst:code:rest:api:subjects:id:teachers:read:ret}. 
Die einzelnen Felder der Antwort werden in Tabelle \reftabllec{tab:rest:api:subjects:id:teachers:read:ret:json} beschrieben.
Die Berechtigungen auf den Endpoint können \reftabllec{tab:rest:api:subjects:id:teachers:read:right} entnommen werden.
\lstset{
  language=json,
  tabsize=2,
  captionpos=b,
  numbers=left,
  commentstyle=\color{green},
  backgroundcolor=\color{white},
  numberstyle=\color{gray},
  keywordstyle=\color{blue} \textbf,%otherkeywords={xdata},
  keywordstyle=[2]\color{red}\textbf,
  identifierstyle=\color{black},
  stringstyle=\color{red}\ttfamily,
  basicstyle = \ttfamily \color{black} \footnotesize,
  showstringspaces=false,
	breakatwhitespace=false,         % sets if automatic breaks should only happen at whitespace
  breaklines=true, 
}
\begin{lstlisting}[caption={JSON-Antwort für einen GET-Aufruf der Route /api/subjects/\$id/teachers},label={lst:code:rest:api:subjects:id:teachers:read:ret},frame=tlrb]
[
 { 
  subject: "<STRING>",
  user: "<STRING>",
	start: "<DATE>",
	end: "<DATE>",
 },
 ...
]
\end{lstlisting}
\begin{longtable}{|p{0.2\textwidth}|p{0.2\textwidth}|p{0.58\textwidth}|}
		\caption{Beschreibung der Felder in einem JSON-Objekt mit der Liste aller unterrichtenden Lehrkräfte eines Unterrichtsfaches.}
\endfoot
		\caption{Beschreibung der Felder in einem JSON-Objekt mit der Liste aller unterrichtenden Lehrkräfte eines Unterrichtsfaches.}
		\label{tab:rest:api:subjects:id:teachers:read:ret:json}
\endlastfoot 
\hline
			\textbf{Feldname} & \textbf{Datentyp} & \textbf{Beschreibung} \\ \hline
\endhead
subject & STRING & Zeichenkette, mit der das Unterrichtsfach unter /api/subjects/\$id abgefragt werden kann. \\ \hline
user & STRING &  Zeichenkette, mit der die Lehrkraft unter /api/users/\$id abfragt werden kann. \\ \hline
start & DATE & Optional; wird gesetzt, falls die Lehrkraft ach Start des Zeitraumes eines Unterrichtsfaches dem Unterrichtsfach hinzugefügt wurde. \\ \hline
end & DATE & Optional; wird gesetzt, falls die Lehrkräften vor Ablauf des Zeitraumes eines Unterrichtsfaches aus dem Unterrichtsfach entfernt wurde. \\ \hline
\end{longtable}
