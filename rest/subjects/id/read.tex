\subsubsection{READ}
\label{sec:rest:api:subjects:id:read}
Es sind nur Anfragen mit der HTTP-GET-Methode für ein READ auf die Daten zugelassen.
Dieser Pfad gibt bei einer READ-Anfrage für das per ID ausgewählte Unterrichtsfach die allgemeinen Informationen davon wieder.
Die Daten, welche wiedergegeben werden, sind durch den Kontext des anfragenden Benutzers eingeschränkt.

Die Antwort erfolgt mit dem JSON-Objekt \reflisting{lst:code:rest:api:subjects:id:read:ret}. 
Die einzelnen Felder der Antwort werden in \reftabllec{tab:rest:api:subjects:id:read:ret} beschrieben.

\lstset{
  language=json,
  tabsize=2,
  captionpos=b,
  numbers=left,
  commentstyle=\color{green},
  backgroundcolor=\color{white},
  numberstyle=\color{gray},
  keywordstyle=\color{blue} \textbf,%otherkeywords={xdata},
  keywordstyle=[2]\color{red}\textbf,
  identifierstyle=\color{black},
  stringstyle=\color{red}\ttfamily,
  basicstyle = \ttfamily \color{black} \footnotesize,
  showstringspaces=false,
	breakatwhitespace=false,         % sets if automatic breaks should only happen at whitespace
  breaklines=true, 
}
\begin{lstlisting}[caption={JSON-Antwort für einen GET-Aufruf des Pfads /api/subjects/\$id},label={lst:code:rest:api:subjects:id:read:ret},frame=tlrb]
{
    id: "<STRING>",
    displayName: "<STRING>",
    schoolSubject: [
    {
        id: "<STRING>",
        displayName: "<STRING>"
    },
    ...
    ],
    school: {
        id: "<STRING>",
        displayName: "<STRING>"
    },
    schoolYear: {
            id: "<STRING>",
            displayName: "<STRING>"
    },
    grade: [
        "STRING",
        ...
    ],
    users: [
    {
        id: "<STRING>",
        displayName: "<STRING>",
        roleWithinSubject: <STRING>",
        start: "<DATE>",
        end: "<DATE>"    
    },
    ...
    ]
 }
\end{lstlisting}

\begin{longtable}{|p{0.16\textwidth}|p{0.16\textwidth}|p{0.58\textwidth}|}
		\caption{Beschreibung der Felder in einem JSON-Objekt für ein Unterrichtsfach}
\endfoot
		\caption{Beschreibung der Felder in einem JSON-Objekt für ein Unterrichtsfach}
		\label{tab:rest:api:subjects:id:read:ret}
\endlastfoot 
\hline
			\textbf{Feldname} & \textbf{Datentyp} & \textbf{Beschreibung} \\ \hline
\endhead
id & STRING & Eindeutige Zeichenkette, die das Unterrichtsfach an einer Schule repräsentiert.  \\ \hline
displayName & STRING & Name des Unterrichtsfaches. \\ \hline
schoolSubject & Liste von STRING & Für jede Zeichenkette aus dieser Liste kann unter /api/school-subjects/\$id die genauen Informationen abgefragt werden. \\ \hline
school & STRING & ID der Schule, zu der das Fach gehört. Die Daten der Schule können damit unter /api/schools/\$id abgefragt werden. \\ \hline
schoolYear & STRING & ID des Schuljahres, in dem das Unterrichtsfach angeboten wurde. Das Schuljahr kann damit unter /api/school-years/\$id abgefragt werden. \\ \hline
grade & List of STRINGs & Enthält eine Liste von Klassenstufen, die dieser Klasse zugeordnet sind. \\ \hline
users & Array of Objects & User-Rollen-Informationen in Bezug auf das Unterrichtsfach.\\ \hline
% roleWithinSubject & STRING & Rolle in Bezug auf ein Schulfach. \\hline
% start & DATE & Eine Datumsangabe nach ISO-8601 im Format YYYY-MM-DD, ab dem das Unterrichtsfach angeboten wird.  \\ \hline
% end & DATE & Eine Datumsangabe nach ISO-8601 im Format YYYY-MM-DD, bis zu dem das Unterrichtsfach angeboten wird. \\ \hline
\end{longtable}
