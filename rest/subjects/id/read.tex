\subsubsection{READ}
\label{sec:rest:api:subjects:id:read}
Es sind nur Anfragen mit der HTTP-GET-Methode für ein READ auf die Daten zugelassen.
Dieser Pfad gibt bei einer READ-Anfrage für das per ID ausgewählte Unterrichtsfach die allgemeinen Informationen davon wieder.
Die Daten, welche wiedergegeben werden, sind durch den Kontext des anfragenden Benutzers eingeschränkt.

Die Antwort erfolgt mit dem JSON-Objekt \reflisting{lst:code:rest:api:subjects:id:read:ret}. 
Die einzelnen Felder der Antwort werden in \reftabllec{tab:rest:api:subjects:id:read:ret:json} beschrieben.

\lstset{
  language=json,
  tabsize=2,
  captionpos=b,
  numbers=left,
  commentstyle=\color{green},
  backgroundcolor=\color{white},
  numberstyle=\color{gray},
  keywordstyle=\color{blue} \textbf,%otherkeywords={xdata},
  keywordstyle=[2]\color{red}\textbf,
  identifierstyle=\color{black},
  stringstyle=\color{red}\ttfamily,
  basicstyle = \ttfamily \color{black} \footnotesize,
  showstringspaces=false,
	breakatwhitespace=false,         % sets if automatic breaks should only happen at whitespace
  breaklines=true, 
}
\begin{lstlisting}[caption={JSON-Antwort für einen GET-Aufruf des Pfads /api/subjects/\$id},label={lst:code:rest:api:subjects:id:read:ret},frame=tlrb]
{
    id: "<STRING>",
    displayName: "<STRING>",
    referenceSubject: [
    {
        id: "<STRING>",
        displayName: "<STRING>"
    },
    ...
    ],
    school: {
        id: "<STRING>",
        displayName: "<STRING>"
    },
    schoolPeriod: {
            id: "<STRING>",
            displayName: "<STRING>"
    },
    grade: [
        "STRING",
        ...
    ],
    users: [
    {
        id: "<STRING>",
        roleWithinSubject: <STRING>",
        start: "<DATE>",
        end: "<DATE>"    
    },
    ...
    ]
 }
\end{lstlisting}

\begin{longtable}{|p{0.2\textwidth}|p{0.2\textwidth}|p{0.58\textwidth}|}
		\caption{Beschreibung der Felder in einem JSON-Objekt für ein Unterrichtsfach}
\endfoot
		\caption{Beschreibung der Felder in einem JSON-Objekt für ein Unterrichtsfach}
		\label{tab:rest:api:subjects:id:read:ret:json}
\endlastfoot 
\hline
			\textbf{Feldname} & \textbf{Datentyp} & \textbf{Beschreibung} \\ \hline
\endhead
id & STRING & Eindeutige Zeichenkette, die das Unterrichtsfach an einer Schule repräsentiert.  \\ \hline
displayName & STRING & Name des Unterrichtsfaches \\ \hline
referenceSubject & Array of Objects & OPTIONAL; Liste von Unterrichtsfächern und den Rollen, welcher der Benutzer in diesen innehat. Der Aufbau der Objekte kann \reftabllec{tab:rest:api:subjects:id:reference-subjects} entnommen werden. \\ \hline
school & Object & Objekt mit Informationen zu Schule zu der die Klasse gehört. Der Aufbau des Objektes kann \reftabllec{tab:rest:api:subjects:id:school} entnommen werden.  \\ \hline
schoolPeriod & Object & Objekt mit Informationen zum Schuljahr, in welchem der Benutzer die Klasse besucht hat. Der Aufbau des Objektes kann \reftabllec{tab:rest:api:subjects:id:schoolperiod} entnommen werden. \\ \hline
grade & List of STRINGs & Enthält eine Liste von Klassenstufen, die dieser Klasse zugeordnet sind. \\ \hline
users & Array of Objects & Liste von Rollen in Bezug auf ein Unterrichtsfach. Der Aufbau der Objekte kann \reftabllec{tab:rest:api:subjects:id:users:read:ret:json} entnommen werden. \\ \hline
\end{longtable}

\begin{longtable}{|p{0.2\textwidth}|p{0.2\textwidth}|p{0.58\textwidth}|}
        \caption{Beschreibung der Felder in einem JSON-Objekt für das Zuordnen von Referenzschulfächern zu einem Unterrichtsfach}
\endfoot
        \caption{Beschreibung der Felder in einem JSON-Objekt für das Zuordnen von Referenzschulfächern zu einem Unterrichtsfach}
        \label{tab:rest:api:subjects:id:reference-subjects}
\endlastfoot 
\hline
            \textbf{Feldname} & \textbf{Datentyp} & \textbf{Beschreibung} \\ \hline
\endhead
id & STRING & ID des Referenzschulfachs. \\ \hline
displayName & STRING & OPTIONAL; Anzeigename des Referenzschulfachs \\ \hline
\end{longtable}

\begin{longtable}{|p{0.2\textwidth}|p{0.2\textwidth}|p{0.58\textwidth}|}
        \caption{Beschreibung der Felder im JSON-Objekt für die Schule }
\endfoot
        \caption{Beschreibung der Felder im JSON-Objekt für die Schule }
        \label{tab:rest:api:subjects:id:school}
\endlastfoot 
\hline
            \textbf{Feldname} & \textbf{Datentyp} & \textbf{Beschreibung} \\ \hline
\endhead
id & STRING & ID der Schule, mit der unter /api/schools/\$id weitere Informationen abgefragt werden können. \\ \hline
displayName & STRING & OPTIONAL; Anzeigename der Schule \\ \hline
\end{longtable}

\begin{longtable}{|p{0.2\textwidth}|p{0.2\textwidth}|p{0.58\textwidth}|}
        \caption{Beschreibung der Felder im JSON-Objekt für das Schuljahr }
\endfoot
        \caption{Beschreibung der Felder im JSON-Objekt für das Schuljahr }
        \label{tab:rest:api:subjects:id:schoolperiod}
\endlastfoot 
\hline
            \textbf{Feldname} & \textbf{Datentyp} & \textbf{Beschreibung} \\ \hline
\endhead
id & STRING & ID des Schuljahres, mit der unter /api/school-periods/\$id weitere Informationen abgefragt werden können. \\ \hline
displayName & STRING & OPTIONAL; Anzeigename des Schuljahres \\\hline
\end{longtable}
