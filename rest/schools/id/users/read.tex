\subsubsection{READ}
\label{sec:end:rest:api:schools:id:users:read}
Es sind nur Anfragen mit der HTTP-GET-Methode für ein READ auf die Daten zugelassen.
Bei diesem Aufruf wird eine Liste mit von Objekten übermittelt, auf welche der aktuelle Benutzer Zugriff hat, eingeschränkt auf die Schule, welche über die ID ausgewählt wurde.
Ein Objekt aus der Liste enthält immer für genau einen Benutzer die Information der Benutzer-ID, der Schul-ID, der Rollen-ID, eines Zeitpunkts, ab wann das Objekt gültig war, und optional eines Zeitpunkts, bis wann das Objekt gültig ist.
Gibt es Unterbrechungen in den Zeiträumen, in denen ein Benutzer eine Rolle in einer Schule eingenommen hat, so muss es für jeden Zeitraum ein eigenes Objekt geben.
Zum Abrufen von Daten mit IDs müssen die Anwendungen über die entsprechenden Spezialrouten gehen.
Die Antwort erfolgt mit dem JSON-Objekt \reflisting{lst:code:rest:api:schools:id:users:read:ret}.
Die einzelnen Felder der Antwort werden in \reftabllec{tab:rest:api:schools:id:users:read:ret:json} beschrieben.
Die Berechtigungen auf den Endpunkt können aus \reftabllec{tab:rest:api:schools:id:users:read:right} entnommen werden.

\lstset{
  language=json,
  tabsize=2,
  captionpos=b,
  numbers=left,
  commentstyle=\color{green},
  backgroundcolor=\color{white},
  numberstyle=\color{gray},
  keywordstyle=\color{blue} \textbf,%otherkeywords={xdata},
  keywordstyle=[2]\color{red}\textbf,
  identifierstyle=\color{black},
  stringstyle=\color{red}\ttfamily,
  basicstyle = \ttfamily \color{black} \footnotesize,
  showstringspaces=false,
	breakatwhitespace=false,         % sets if automatic breaks should only happen at whitespace
  breaklines=true, 
}
\begin{lstlisting}[caption={JSON-Antwort für einen GET-Aufruf der Route /api/schools/\$id/users},label={lst:code:rest:api:schools:id:users:read:ret},frame=tlrb]
[
 {
  school_id: "<STRING>",
  user_id: "<STRING>",
	role: "<STRING>",
	start: "<TIMESTAMP>",
	end: "<TIMESTAMP>",
	school-years: ["<STRING>",...,"<STRING>"]
 },
...
 {
  school_id: "<STRING>",
  user_id: "<STRING>",
	role: "<STRING>",
	start: "<TIMESTAMP>",
	end: "<TIMESTAMP>",
	school-years: ["<STRING>",...,"<STRING>"]
 },
]
\end{lstlisting}

\begin{table}[htb]
	\begin{tabularx}{\textwidth}{|c|c|X|}
		\hline
			\textbf{Feldname} & \textbf{Datentyp} & \textbf{Beschreibung} \\ \hline
			school\_id & STRING & ID der Schule \\ \hline
			user\_id & STRING & ID des Benutzers \\ \hline
			role & STRING & Rolle des Benutzers an der Schule \\ \hline
			start & TIMESTAMP & Zeitpunkt, ab wann der Benutzer diese Rolle innehat \\ \hline
			end & TIMESTAMP & Optional; gibt an, bis wann ein Benutzer diese Rolle innehatte \\ \hline
			school-years & LIST of STRINGS & Optional; das Feld gibt es nur, wenn die Rolle \textit{students} oder \textit{external-students} ist; enthält eine Liste von Schuljahres-IDs \\ \hline 
	\end{tabularx}

		\caption{Beschreibung der Felder in einem JSON-Objekt für einen Benutzer an einer Schule}
		\label{tab:rest:api:schools:id:users:read:ret:json}
\end{table}

\begin{longtable}{|c|p{0.7\textwidth}|}
\caption{Berechtigungen auf dem Endpunkt}
\endfoot
		\caption{Berechtigungen auf dem Endpunkt}
		\label{tab:rest:api:schools:id:users:read:right}
\endlastfoot
\hline
\textbf{Benutzergruppen} & \textbf{Zugelassene Daten} \\ \hline
\endhead
guest & Darf den Endpunkt nicht aufrufen und keine Daten vom Endpunkt erhalten. \\ \hline
user & Bekommt eine Liste mit ihren Daten von der ausgewählten Schule wieder. \\ \hline 
students & Bekommt die Daten aller ihrer Mitschüler von der ausgewählten Schule, mit denen sie gemeinsame Klassen oder Kurse hat, (nur Rollen: \textit{students} und \textit{external-students}),
           den Daten ihrer Erziehungsberechtigten (nur Rolle \textit{guardians}), 
					 den Daten ihrer Lehrer von der ausgewählten Schule (nur Rolle \textit{teacher}) und 
					 den Daten ihrer Schulleiter von der ausgewählten Schule (nur Rolle \textit{principal}) wieder.\\ \hline
external-students &  Bekommt eine Liste mit den Daten aller ihrer Mitschüler von der ausgewählten Schule, mit denen sie gemeinsame Klassen oder Kurse hat,  (nur Rollen: \textit{students} und \textit{external-students}),
           den Daten ihrer Lehrer von der ausgewählten Schule (nur Rolle \textit{teacher}) und 
					 den Daten ihres Schulleiters von der ausgewählten Schule (nur Rolle \textit{principal}) wieder.\\ \hline
guardians & Bekommt eine Liste mit den Daten ihrer Kinder unter 18 von der ausgewählten Schule (nur Rollen: \textit{students} und \textit{external-students}), 
					den Daten der Personen von der ausgewählten Schule, für die Sie Vormund ist, (nur Rollen: \textit{students} und \textit{external-students}), 
					den Daten der Lehrer ihrer Kinder unter 18 von der ausgewählten Schule (nur Rolle \textit{teacher}),
					den Daten der Lehrer von Personen von der ausgewählten Schule, für die sie Vormund ist, (nur Rolle \textit{teacher}),
					den Daten der Schulleiter ihrer Kinder unter 18 von der ausgewählten Schule, (nur Rolle \textit{principal}) und
					die Daten der Schulleiter von Personen von der ausgewählten Schule, für die sie die Vormundschaft hat, (nur Rolle \textit{principal})
					wieder.\\ \hline
teacher & Bekommt eine Liste mit den Daten aller Schüler von der ausgewählten Schule, die sie unterrichten, (nur Rollen: \textit{students} und \textit{external-students}),
					den Daten aller Erziehungsberechtigten von Schülern unter 18 von der ausgewählten Schule, die sie unterrichten, (nur Rolle \textit{guardians}),
					den Daten aller Erziehungsberechtigten von Schülern mit einem Vormund von der ausgewählten Schule, die sie unterrichten, (nur Rolle \textit{guardians}) und
					den Daten aller Kollegen von der ausgewählten Schule (nur mit den Rollen \textit{teacher}, \textit{principal}, \textit{school-admin}) 
					wieder.\\ \hline
principal & Bekommt eine Liste mit ihren Daten,
            den Daten aller Schüler von der ausgewählten Schule, an der sie Schulleiter ist, (nur Rollen: \textit{students} und \textit{external-students}),
						den Daten aller Erziehungsberechtigten von Schülern von der ausgewählten Schule, an der sie Schulleiter ist, (nur Rolle \textit{guardians}) und
					  den Daten aller Kollegen von der ausgewählten Schule (nur mit den Rollen \textit{teacher}, \textit{principal}, \textit{school-admin})
						wieder.\\ \hline
school-admin & Bekommt eine Liste mit ihren Daten und
               den Daten aller Personen von der ausgewählten Schule, an der sie die Rolle \textit{school-admin} hat, mit den Rollen \textit{students}, \textit{external-students}, \textit{guardians}, \textit{teacher}, \textit{principal} und \textit{school-admin} wieder.  \\ \hline
school-board & \textcolor[rgb]{1,0.41,0.13}{\textbf{Muss geklärt werden, welche Einschränkungen auf den Daten es hier gibt}} \\ \hline
fed-school-board & \textcolor[rgb]{1,0.41,0.13}{\textbf{Muss geklärt werden, welche Einschränkungen auf den Daten es hier gibt}} \\ \hline
sync-systems & Liste mit allen Personen, in den Rollen der Personen an den Schulen, von den es syncen darf. \\ \hline
	\end{longtable}