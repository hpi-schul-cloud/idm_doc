\subsubsection{READ}
\label{sec:rest:api:schools:users:read}
Es sind nur Anfragen mit der HTTP-GET-Methode für ein READ auf die Daten zugelassen.
Dieser Endpunkt listet die Informationen über Benutzer und ihre Rollen an der Schule auf.
Für jede Kombination aus Benutzer, Rolle und Startdatum gibt es einen Eintrag in der Liste, die zurückgegeben wird.
Bei manchen Kombinationen gibt es zudem noch ein Datum, bis zu dem der Benutzer die Rolle innenhatte.
Zudem kann es bei einigen Rollen noch die Zusatzinformation geben, in welche Schuljahren der User die Rolle innen hatte.

Die Antwort erfolgt mit dem JSON-Objekt \reflisting{lst:code:rest:api:schools:users:read:ret}. 
Die einzelnen Felder der Antwort werden in \reftabllec{tab:rest:api:schools:users:read:ret:json} beschrieben.
Die Berechtigungen auf den Endpunkt können \reftabllec{tab:rest:api:schools:users:read:right} entnommen werden.

\lstset{
  language=json,
  tabsize=2,
  captionpos=b,
  numbers=left,
  commentstyle=\color{green},
  backgroundcolor=\color{white},
  numberstyle=\color{gray},
  keywordstyle=\color{blue} \textbf,%otherkeywords={xdata},
  keywordstyle=[2]\color{red}\textbf,
  identifierstyle=\color{black},
  stringstyle=\color{red}\ttfamily,
  basicstyle = \ttfamily \color{black} \footnotesize,
  showstringspaces=false,
	breakatwhitespace=false,         % sets if automatic breaks should only happen at whitespace
  breaklines=true, 
}
\begin{lstlisting}[caption={JSON-Antwort für einen GET-Aufruf des Pfads /api/schools/\$id/users},label={lst:code:rest:api:schools:users:read:ret},frame=tlrb]
[
 {
  user: "<STRING>",
  role: "<STRING>",
	start: "<DATE>",
	end: "<DATE>",
	school-years: ["<STRING>"',...],
 },
 ...
]
\end{lstlisting}
\begin{longtable}{|p{0.16\textwidth}|p{0.16\textwidth}|p{0.58\textwidth}|}
		\caption{Beschreibung der Felder in einem JSON-Objekt für das Zuordnen eines Benutzer in einer Rolle zu einer Schule}
\endfoot
		\caption{Beschreibung der Felder in einem JSON-Objekt für das Zuordnen eines Benutzer in einer Rolle zu einer Schule}
		\label{tab:rest:api:schools:users:read:ret:json}
\endlastfoot 
\hline
			\textbf{Feldname} & \textbf{Datentyp} & \textbf{Beschreibung} \\ \hline
\endhead
user & STRING & Zeichenkette, welche eine ID von einem Benutzer ist, der unter /api/subjects/\$id abgerufen werden kann. \\ \hline
role & STRING & Rolle, die der Benutzer innehat. \\ \hline
start & DATE & Datum, ab dem der Benutzer die Rolle innehat \\ \hline
end & DATE & Optional; Gibt an, bis zu welchen Tag der Benutzer die Rolle innehatte. \\ \hline
school-years & Liste & Optional; Liste von Zeichenketten, welche Referenzen auf Schuljahre sind, die unter /api/school-years/\$id abgerufen werden können. \\ \hline 
\end{longtable}
