\subsubsection{READ}
\label{sec:rest:api:school:id:read}
Es sind nur Anfragen mit der HTTP-GET-Methode für ein READ auf die Daten zugelassen.
Diese Route gibt für die per ID ausgewählte Schule die Stammdaten zurück.

Die Antwort erfolgt mit dem JSON-Objekt \reflisting{lst:code:rest:api:school:id:read:ret}. 
Die einzelnen Felder der Antwort werden in \reftabllec{tab:rest:api:school:id:read:ret} beschrieben.
Die Berechtigungen auf den Endpunkt können \reftabllec{tab:rest:api:school:id:read:right} entnommen werden.
\lstset{
  language=json,
  tabsize=2,
  captionpos=b,
  numbers=left,
  commentstyle=\color{green},
  backgroundcolor=\color{white},
  numberstyle=\color{gray},
  keywordstyle=\color{blue} \textbf,%otherkeywords={xdata},
  keywordstyle=[2]\color{red}\textbf,
  identifierstyle=\color{black},
  stringstyle=\color{red}\ttfamily,
  basicstyle = \ttfamily \color{black} \footnotesize,
  showstringspaces=false,
	breakatwhitespace=false,         % sets if automatic breaks should only happen at whitespace
  breaklines=true, 
}
\begin{lstlisting}[caption={JSON-Antwort für einen GET-Aufruf des Pfads /api/school/\$id},label={lst:code:rest:api:school:id:read:ret},frame=tlrb]
{
 school: "<STRING>",
 name: "<STRING>",
}
\end{lstlisting}

\begin{longtable}{|p{0.16\textwidth}|p{0.16\textwidth}|p{0.58\textwidth}|}
		\caption{Beschreibung der Felder in einem JSON-Objekt für das Zuordnen eines Benutzer in einer Rolle zu einer Schule}
\endfoot
		\caption{Beschreibung der Felder in einem JSON-Objekt für das Zuordnen eines Benutzer in einer Rolle zu einer Schule}
		\label{tab:rest:api:school:id:read:ret}
\endlastfoot 
\hline
			\textbf{Feldname} & \textbf{Datentyp} & \textbf{Beschreibung} \\ \hline
\endhead
school & STRING & Eindeutiger Identifikations-String der Schule im IDM \\ \hline
name & STRING & Name der Schule \\ \hline
\end{longtable}

\begin{longtable}{|c|p{0.7\textwidth}|}
\caption{Berechtigungen auf dem Endpunkt}
\endfoot
		\caption{Berechtigungen auf dem Endpunkt}
		\label{tab:rest:api:school:id:read:right}
\endlastfoot
\hline
\textbf{Benutzergruppen} & \textbf{Zugelassene Daten} \\ \hline
\endhead
guest & Darf den Endpunkt nicht aufrufen und keine Daten vom Endpunkt erhalten. \\ \hline
user &  \\ \hline 
students & \\ \hline
external-students & \\ \hline
guardians & \\ \hline
teacher & \\ \hline
principal & \\ \hline
school-admin & \\ \hline
school-board & \\ \hline
fed-school-board & \\ \hline
sync-systems & \\ \hline
	\end{longtable}