\subsubsection{READ}
\label{sec:rest:api:schools:id:read}
Es sind nur Anfragen mit der HTTP-GET-Methode für ein READ auf die Daten zugelassen.
Dieser Pfad gibt für die per ID ausgewählte Schule die Stammdaten zurück.

Die Antwort erfolgt mit dem JSON-Objekt \reflisting{lst:code:rest:api:schools:id:read:ret}. 
Die einzelnen Felder der Antwort werden in \reftabllec{tab:rest:api:schools:id:read:ret} beschrieben.
\lstset{
  language=json,
  tabsize=2,
  captionpos=b,
  numbers=left,
  commentstyle=\color{green},
  backgroundcolor=\color{white},
  numberstyle=\color{gray},
  keywordstyle=\color{blue} \textbf,%otherkeywords={xdata},
  keywordstyle=[2]\color{red}\textbf,
  identifierstyle=\color{black},
  stringstyle=\color{red}\ttfamily,
  basicstyle = \ttfamily \color{black} \footnotesize,
  showstringspaces=false,
	breakatwhitespace=false,         % sets if automatic breaks should only happen at whitespace
  breaklines=true, 
}
\begin{lstlisting}[caption={JSON-Antwort für einen GET-Aufruf des Pfads /api/school/\$id},label={lst:code:rest:api:schools:id:read:ret},frame=tlrb]
{
	id: "<STRING>",
	displayName: "<STRING>",
	schoolNumber: "<STRING>",
	schoolType: "<STRING>",
	district: "<STRING>",
	classes: [
	{
		id: "<STRING>",
		displayName: "<STRING>"
	},
	...
	],
	subjects: [
	{
		id: "<STRING>",
		displayName: "<STRING>"
	},
	...
	],
	users: [
	{
		id: "<STRING>",
		role: "<STRING>",
		start: "<DATE>",
		end: "<DATE>"
	},
	...
	],
	status: "<ENUM>",
	deleteOn: "<DATE>"

}
\end{lstlisting}

\begin{longtable}{|p{0.18\textwidth}|p{0.18\textwidth}|p{0.52\textwidth}|}
		\caption{Beschreibung der Felder in einem JSON-Objekt für das Zuordnen eines Benutzer in einer Rolle zu einer Schule}
\endfoot
		\caption{Beschreibung der Felder in einem JSON-Objekt für das Zuordnen eines Benutzer in einer Rolle zu einer Schule}
		\label{tab:rest:api:schools:id:read:ret}
\endlastfoot 
\hline
			\textbf{Feldname} & \textbf{Datentyp} & \textbf{Beschreibung} \\ \hline
\endhead
id & STRING & Eindeutiger Identifikations-String der Schule im IdM \\ \hline
displayName & STRING & Name der Schule \\ \hline
schoolNumber & STRING & OPTIONAL; Schulnummer \\ \hline
schoolType & STRING & OPTIONAL; Schultyp \\ \hline
district & STRING & OPTIONAL; Schulbezirk \\ \hline
classes & Array of Objects & Liste von Klassen an der Schule. Der Aufbau der Objekte kann \reftabllec{tab:rest:api:schools:id:read:ret:subjects} entnommen werden. \\ \hline
subjects & Array of Objects & Liste von Unterrichtsfächern an der Schule. Der Aufbau der Objekte kann \reftabllec{tab:rest:api:schools:id:read:ret:classes} entnommen werden. \\ \hline
users & Array of Objects & Liste von Benutzern an der Schule. Der Aufbau der Objekte kann \reftabllec{tab:rest:api:schools:id:read:ret:users} entnommen werden. \\ \hline
\end{longtable}

\begin{longtable}{|p{0.18\textwidth}|p{0.18\textwidth}|p{0.52\textwidth}|}
		\caption{Beschreibung der Felder in JSON-Objekt classes[]}
\endfoot
		\caption{Beschreibung der Felder in JSON-Objekt classes[]}
		\label{tab:rest:api:schools:id:read:ret:classes}
\endlastfoot 
\hline
			\textbf{Feldname} & \textbf{Datentyp} & \textbf{Beschreibung} \\ \hline
\endhead
id & STRING & ID der Klasse \\ \hline
displayName & STRING & OPTIONAL; Anzeigename der Klasse \\ \hline
\end{longtable}

\begin{longtable}{|p{0.18\textwidth}|p{0.18\textwidth}|p{0.52\textwidth}|}
		\caption{Beschreibung der Felder in JSON-Objekt subjects[]}
\endfoot
		\caption{Beschreibung der Felder in JSON-Objekt subjects[]}
		\label{tab:rest:api:schools:id:read:ret:subjects}
\endlastfoot 
\hline
			\textbf{Feldname} & \textbf{Datentyp} & \textbf{Beschreibung} \\ \hline
\endhead
id & STRING & ID des Unterrichtsfaches \\ \hline
displayName & STRING & OPTIONAL; Anzeigename des Unterrichtsfaches \\ \hline
\end{longtable}

\begin{longtable}{|p{0.18\textwidth}|p{0.18\textwidth}|p{0.52\textwidth}|}
		\caption{Beschreibung der Felder in JSON-Objekt users[]}
\endfoot
		\caption{Beschreibung der Felder in JSON-Objekt users[]}
		\label{tab:rest:api:schools:id:read:ret:users}
\endlastfoot 
\hline
			\textbf{Feldname} & \textbf{Datentyp} & \textbf{Beschreibung} \\ \hline
\endhead
id & STRING & ID des Benutzers \\ \hline
role & STRING & Rolle, welche der Benutzer an der Schule innehat. Werte dafür können \reftabllec{tab:intro:rolesschool} entnommen werden. \\ \hline
start & DATE & OPTIONAL; eine Datumsangabe nach ISO-8601 in dem Format YYYY-MM-DD. Gibt Datum an, ab dem der Benutzer die Rolle an der Schule innehat. \\ \hline
end & DATE & OPTIONAL; eine Datumsangabe nach ISO-8601 in dem Format YYYY-MM-DD. Gibt Datum an, bis zu dem der Benutzer die Rolle an der Schule innehat. \\ \hline
status & ENUM & 
\begin{itemize}
	\item active steht für aktiver Benutzer
	\item inactive steht für inaktiver Benutzer
\end{itemize}
 \\ \hline
deleteOn & DATE & OPTIONAL; eine Datumsangabe nach ISO-8601 in dem Format YYYY-MM-DD. Datum, an dem die Daten des Benutzers gelöscht werden sollen. \\ \hline
\end{longtable}
