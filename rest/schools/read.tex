\subsubsection{READ}
\label{sec:rest:api:school:read}
Es sind nur Anfragen mit der HTTP-GET-Methode für ein READ auf die Daten zugelassen.
Der Endpunkt gib eine Liste mit allen Schul-IDs aus, die es im System gibt und für die Daten unter /api/schools/\$id abgerufen werden können.

Die Antwort erfolgt mit dem JSON-Objekt \reflisting{lst:code:rest:api:school:read:ret}. 
Die einzelnen Felder der Antwort werden in \reftabllec{tab:rest:api:school:read:ret} beschrieben.
Die Berechtigungen auf den Endpunkt können \reftabllec{tab:rest:api:school:read:right} entnommen werden.
\lstset{
  language=json,
  tabsize=2,
  captionpos=b,
  numbers=left,
  commentstyle=\color{green},
  backgroundcolor=\color{white},
  numberstyle=\color{gray},
  keywordstyle=\color{blue} \textbf,%otherkeywords={xdata},
  keywordstyle=[2]\color{red}\textbf,
  identifierstyle=\color{black},
  stringstyle=\color{red}\ttfamily,
  basicstyle = \ttfamily \color{black} \footnotesize,
  showstringspaces=false,
	breakatwhitespace=false,         % sets if automatic breaks should only happen at whitespace
  breaklines=true, 
}
\begin{lstlisting}[caption={JSON-Antwort für einen GET-Aufruf des Pfads /api/schools},label={lst:code:rest:api:school:read:ret},frame=tlrb]
[
 "<STRING>",
 ...
]
\end{lstlisting}

\begin{longtable}{|p{0.16\textwidth}|p{0.16\textwidth}|p{0.58\textwidth}|}
		\caption{Beschreibung des Felder in einem JSON-Objekt, was die Schul-IDs wiedergibt}
\endfoot
		\caption{Beschreibung des Felder in einem JSON-Objekt, was die Schul-IDs wiedergibt}
		\label{tab:rest:api:school:read:ret}
\endlastfoot 
\hline
			\textbf{Feldname} & \textbf{Datentyp} & \textbf{Beschreibung} \\ \hline
\endhead
 & STRING & Eindeutiger Identifikations-String der Schule im IDM, mit dem unter /api/schools/\$id die Schule abgerufen werden kann.\\ \hline
\end{longtable}