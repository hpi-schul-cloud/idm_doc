\subsubsection{READ}
\label{sec:rest:api:classes:id:read}
Es sind nur Anfragen mit der HTTP-GET-Methode für ein READ auf die Daten zugelassen.
Bei Anfragen an diesen Endpunkt werden die allgemeinen Daten einer Klasse ausgegeben.

Die Antwort erfolgt mit dem JSON-Objekt \reflisting{lst:code:rest:api:classes:id:read:ret}. 
Die einzelnen Felder der Antwort werden in \reftabllec{tab:rest:api:classes:id:read:ret} beschrieben.
\lstset{
  language=json,
  tabsize=2,
  captionpos=b,
  numbers=left,
  commentstyle=\color{green},
  backgroundcolor=\color{white},
  numberstyle=\color{gray},
  keywordstyle=\color{blue} \textbf,%otherkeywords={xdata},
  keywordstyle=[2]\color{red}\textbf,
  identifierstyle=\color{black},
  stringstyle=\color{red}\ttfamily,
  basicstyle = \ttfamily \color{black} \footnotesize,
  showstringspaces=false,
	breakatwhitespace=false,         % sets if automatic breaks should only happen at whitespace
  breaklines=true, 
}
\begin{lstlisting}[caption={JSON-Antwort für einen GET-Aufruf des Pfads /api/classes/\$id},label={lst:code:rest:api:classes:id:read:ret},frame=tlrb]
[
{
    id: "<STRING>",
    displayName: "<STRING>",
    summary: "<STRING>",
    school: {
        id: "<STRING>",
        displayName: "<STRING>"
    },
    schoolYear: {
        id: "<STRING>",
        displayName: "<STRING>"
    },
    grade: [
        "STRING",
        ...
    ],
    users: [
    {
        id: "<STRING>",
        displayName: "<STRING>",
        roleWithinClass: <STRING>",
        start: "<DATE>",
        end: "<DATE>"    
    },
    ...
    ]
}
]
\end{lstlisting}

\begin{longtable}{|p{0.2\textwidth}|p{0.2\textwidth}|p{0.58\textwidth}|}
		\caption{Beschreibung der Felder in einem JSON-Objekt für das Zuordnen eines Benutzer in einer Rolle zu einer Schule}
\endfoot
		\caption{Beschreibung der Felder in einem JSON-Objekt für das Zuordnen eines Benutzer in einer Rolle zu einer Schule}
		\label{tab:rest:api:classes:id:read:ret}
\endlastfoot 
\hline
			\textbf{Feldname} & \textbf{Datentyp} & \textbf{Beschreibung} \\ \hline
\endhead
id & STRING & ID der Klasse im System. \\ \hline
displayName & STRING & Name der Klasse, welche zum Anzeigen verwendet wird. \\ \hline
summary & STRING & Zusammenfassung der Klasse. \\ \hline
school & Object & Objekt mit Informationen zu Schule zu der die Klasse gehört \\ \hline
school.id & STRING & ID der Schule, mit der unter /api/schools/\$id weitere Informationen abgefragt werden können. \\ \hline
school.displayName & STRING & Anzeigename der Schule \\ \hline
schoolYear & Object & Objekt mit Informationen zum Schuljahr, in welchem der Benutzer die Klasse besucht hat. \\ \hline
schoolYear.id & STRING & ID des Schuljahres, mit der unter /api/school-years/\$id weitere Informationen abgefragt werden können. \\ \hline
schoolYear.displayName & STRING & Anzeigename des Schuljahres. \\\hline
grade & List of STRINGs & Enthält eine Liste von Klassenstufen, die dieser Klasse zugeordnet sind. \\ \hline
users & Array of Objects & Liste von Rollen in Bezug auf eine Klasse. Der Aufbau der Objekte kann \reftabllec{tab:rest:api:classes:id:users:read:ret} entnommen werden. \\ \hline
\end{longtable}
