\subsubsection{READ}
\label{sec:rest:api:classes:id:read}
Es sind nur Anfragen mit der HTTP-GET-Methode für ein READ auf die Daten zugelassen.
Bei Anfragen an diesen Endpunkt werden die allgemeinen Daten einer Klasse ausgegeben.

Die Antwort erfolgt mit dem JSON-Objekt \reflisting{lst:code:rest:api:classes:id:read:ret}. 
Die einzelnen Felder der Antwort werden in \reftabllec{tab:rest:api:classes:id:read:ret} beschrieben.
Die Berechtigungen auf den Endpunkt können \reftabllec{tab:rest:api:classes:id:read:right} entnommen werden.
\lstset{
  language=json,
  tabsize=2,
  captionpos=b,
  numbers=left,
  commentstyle=\color{green},
  backgroundcolor=\color{white},
  numberstyle=\color{gray},
  keywordstyle=\color{blue} \textbf,%otherkeywords={xdata},
  keywordstyle=[2]\color{red}\textbf,
  identifierstyle=\color{black},
  stringstyle=\color{red}\ttfamily,
  basicstyle = \ttfamily \color{black} \footnotesize,
  showstringspaces=false,
	breakatwhitespace=false,         % sets if automatic breaks should only happen at whitespace
  breaklines=true, 
}
\begin{lstlisting}[caption={JSON-Antwort für einen GET-Aufruf der Route /api/classes/\$id},label={lst:code:rest:api:classes:id:read:ret},frame=tlrb]
{
 class: "<STRING>",
 name: "<STRING>",
 school: "<STRING>",
 school-year: "<STRING>",
 start: "<DATE>",
 end: "<DATE>",
 grade: [ "<STRING>", ... ],
}
\end{lstlisting}

\begin{longtable}{|p{0.2\textwidth}|p{0.2\textwidth}|p{0.58\textwidth}|}
		\caption{Beschreibung der Felder in einem JSON-Objekt für das Zuordnen eines Benutzer in einer Rolle zu einer Schule}
\endfoot
		\caption{Beschreibung der Felder in einem JSON-Objekt für das Zuordnen eines Benutzer in einer Rolle zu einer Schule}
		\label{tab:rest:api:classes:id:read:ret}
\endlastfoot 
\hline
			\textbf{Feldname} & \textbf{Datentyp} & \textbf{Beschreibung} \\ \hline
\endhead
class & STRING & ID der Klasse im System. \\ \hline
name & STRING & Name der Klasse, welche zum Anzeigen verwendet wird. \\ \hline
school & STRING & ID, zu welcher Schule die Klasse gehört. Die Schule kann damit unter /api/schools/\$id abgefragt werden. \\ \hline
school-year & STRING & Schuljahr, in welchem die Klasse existiert. Die Daten des Schuljahres können unter /api/school-years/\$id abgrufen werden. \\ \hline
start & DATE & Zeitpunkt, ab wann das Klassenobjekt existiert, falls dies abweichend zum Startzeitpunkt des Schuljahres ist. \\ \hline 
end & DATE & Zeitpunkt, bis wann das Klassenobjekt existiert, falls dies abweichend zum Endzeitpunkt des Schuljahres ist. \\ \hline 
grade & List of STRINGs & Enthält eine Liste von Klassenstufen von Schülern in dieser Klasse. \\ \hline
\end{longtable}
