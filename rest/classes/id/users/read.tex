\subsubsection{READ}
\label{sec:rest:api:classes:id:users:read}
Es sind nur Anfragen mit der HTTP-GET-Methode für ein READ auf die Daten zugelassen.
Bei Anfragen an diesen Endpunkt wird eine Liste von IDs von Benutzern zurückgegeben, die der Klasse zugeordnet sind.
Optional ist der Eintritts- sowie Austrittszeitpunkt. Diese Felder werden gesetzt, falls der Benutzer dies nicht über den kompletten Zeitraum der Klasse war.

Die Antwort erfolgt mit dem JSON-Objekt \reflisting{lst:code:rest:api:classes:id:users:read:ret}. 
Die einzelnen Felder der Antwort werden in \reftabllec{tab:rest:api:classes:id:users:read:ret} beschrieben.
\lstset{
  language=json,
  tabsize=2,
  captionpos=b,
  numbers=left,
  commentstyle=\color{green},
  backgroundcolor=\color{white},
  numberstyle=\color{gray},
  keywordstyle=\color{blue} \textbf,%otherkeywords={xdata},
  keywordstyle=[2]\color{red}\textbf,
  identifierstyle=\color{black},
  stringstyle=\color{red}\ttfamily,
  basicstyle = \ttfamily \color{black} \footnotesize,
  showstringspaces=false,
	breakatwhitespace=false,         % sets if automatic breaks should only happen at whitespace
  breaklines=true, 
}
\begin{lstlisting}[caption={JSON-Antwort für einen GET-Aufruf des Pfads /api/classes/\$id/users},label={lst:code:rest:api:classes:id:users:read:ret},frame=tlrb]
[
{
    id: "<STRING>",
    roleWithinClass: <STRING>",
    substitute: "<BOOLEAN>",
    start: "<DATE>",
    end: "<DATE>"    
},
...
]
\end{lstlisting}

\begin{longtable}{|p{0.2\textwidth}|p{0.2\textwidth}|p{0.58\textwidth}|}
		\caption{Beschreibung der Felder in einem JSON-Objekt aus der Liste der Benutzer}
\endfoot
		\caption{Beschreibung der Felder in einem JSON-Objekt aus der Liste der Benutzer}
		\label{tab:rest:api:classes:id:users:read:ret}
\endlastfoot 
\hline
			\textbf{Feldname} & \textbf{Datentyp} & \textbf{Beschreibung} \\ \hline
\endhead
id & STRING & Zeichenkette, mit der der Benutzer unter /api/users/\$id abfragt werden kann. \\ \hline
roleWithinClass & STRING & Rolle des Benutzer innerhalb der Klasse. \\ \hline
substitute & BOOLEAN & Gibt an, ob Rolle in Vertretung wahrgenommen wird. \\ \hline
start & DATE & OPTIONAL; Eine Datumsangabe nach ISO-8601 im Format YYYY-MM-DD. Wird gesetzt, falls der Benutzer nach Start des Zeitraumes einer Klasse der Klasse hinzugefügt wurde. \\ \hline
end & DATE & OPTIONAL; Eine Datumsangabe nach ISO-8601 im Format YYYY-MM-DD. Wird gesetzt, falls der Benutzer vor Ablauf des Zeitraumes einer Klasse aus der Klasse entfernt wurde. \\ \hline
\end{longtable}
