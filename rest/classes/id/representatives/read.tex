\subsubsection{READ}
\label{sec:rest:api:classes:id:representatives:read}
Es sind nur Anfragen mit der HTTP-GET-Methode für ein READ auf die Daten zugelassen.
Bei Anfragen an diesen Endpunkt wird eine Liste von Schülern oder Vormündern, welche in einer Klasse oder Jahrgangsstufe die Statusgruppe repräsentieren, zurückgegeben.
Die Position wird durch eine Ordnungsnummer widergespiegelt. Je kleiner die Nummer, desto wichtiger der Repräsentant der Statusgruppe.
Es können auch mehrere Repräsentanten einer Statusgruppe die gleiche Ordnungsnummer haben.
Zusätzlich kann Eintrittszeitpunkt sowie der Austrittszeitpunkt vermerkt werden, falls dieser abweichend zum Zeitraum der Klasse ist.

Die Antwort erfolgt mit dem JSON-Objekt \reflisting{lst:code:rest:api:classes:id:representatives:read:ret}. 
Die einzelnen Felder der Antwort werden in \reftabllec{tab:rest:api:classes:id:representatives:read:ret} beschrieben.
\lstset{
  language=json,
  tabsize=2,
  captionpos=b,
  numbers=left,
  commentstyle=\color{green},
  backgroundcolor=\color{white},
  numberstyle=\color{gray},
  keywordstyle=\color{blue} \textbf,%otherkeywords={xdata},
  keywordstyle=[2]\color{red}\textbf,
  identifierstyle=\color{black},
  stringstyle=\color{red}\ttfamily,
  basicstyle = \ttfamily \color{black} \footnotesize,
  showstringspaces=false,
	breakatwhitespace=false,         % sets if automatic breaks should only happen at whitespace
  breaklines=true, 
}
\begin{lstlisting}[caption={JSON-Antwort für einen GET-Aufruf des Pfads /api/classes/\$id/representatives},label={lst:code:rest:api:classes:id:representatives:read:ret},frame=tlrb]
[ 
{ 
	user: "<STRING>",
	role: "<STRING>",
	order: "<NUMBER>",
  	start: "<DATE>",
  	end: "<DATE>",
},
 ... 
]
\end{lstlisting}

\begin{longtable}{|p{0.2\textwidth}|p{0.2\textwidth}|p{0.58\textwidth}|}
		\caption{Beschreibung der Felder in einem JSON-Objekt aus der Liste der Repräsentanten von Statusgruppen in einer Klasse}
\endfoot
		\caption{Beschreibung der Felder in einem JSON-Objekt aus der Liste der Repräsentanten von Statusgruppen in einer Klasse}
		\label{tab:rest:api:classes:id:representatives:read:ret}
\endlastfoot 
\hline
			\textbf{Feldname} & \textbf{Datentyp} & \textbf{Beschreibung} \\ \hline
\endhead
user & STRING & Die User-ID des Repräsentanten der Statusgruppe, die Daten des Benutzers können unter /api/users/\$id abgerufen werden \\ \hline
role & STRING & Die Rolle, welcher der Repräsentant der Statusgruppe im System hat (zugelassen sind student oder guardian). \\ \hline
order & NUMBER & Die Ordnungsnummer, die angibt, welche Position der Repräsentant der Statusgruppe in einer Klasse innehat. \\ \hline
start & DATE & Optional; wird benötigt, wenn der Repräsentant der Statusgruppe zu einem abweichenden Zeitpunkt das Amt bekommt. \\ \hline
end & DATE & Optional; wird benötigt, wenn der Repräsentant der Statusgruppe vorzeitig vor Ende des Zeitraums der Klasse nicht mehr als Repräsentant der Statusgruppe fungiert. \\ \hline
\end{longtable}
