\subsubsection{READ}
\label{sec:rest:api:classes:id:teachers:read}
Es sind nur Anfragen mit der HTTP-GET-Methode für ein READ auf die Daten zugelassen.
Bei anfragen an diesen Endpunkt wird eine Liste von IDs von Klassenlehrer die der Klasse als Klassenlehrer zugeordnet sind.
Optional ist der Eintitszeitpunkt, wie der Austritszeitpunkt wenn die Klasse dieser für ein Klassenlehrer zur dauer der Klasse abweicht.

Die Antwort erfolgt mit dem JSON-Objekt \reflisting{lst:code:rest:api:classes:id:teachers:read:ret}. 
Die einzelnen Felder der Antwort werden in \reftabllec{tab:rest:api:classes:id:teachers:read:ret} beschrieben.
Die Berechtigungen auf den Endpunkt können \reftabllec{tab:rest:api:classes:id:teachers:read:right} entnommen werden.
\lstset{
  language=json,
  tabsize=2,
  captionpos=b,
  numbers=left,
  commentstyle=\color{green},
  backgroundcolor=\color{white},
  numberstyle=\color{gray},
  keywordstyle=\color{blue} \textbf,%otherkeywords={xdata},
  keywordstyle=[2]\color{red}\textbf,
  identifierstyle=\color{black},
  stringstyle=\color{red}\ttfamily,
  basicstyle = \ttfamily \color{black} \footnotesize,
  showstringspaces=false,
	breakatwhitespace=false,         % sets if automatic breaks should only happen at whitespace
  breaklines=true, 
}
\begin{lstlisting}[caption={JSON-Antwort für einen GET-Aufruf der Route /api/classes/\$id/teachers},label={lst:code:rest:api:classes:id:teachers:read:ret},frame=tlrb]
[ 
 { 
	user: "<STRING>",
  start: "<CALENDARDATE>",
  end: "<CALENDARDATE>",
	order: [
	 {
		order: "<NUMBER>",
    start: "<CALENDARDATE>",
    end: "<CALENDARDATE>",
	 },
	 ...
	],
 },
 ... 
]
\end{lstlisting}

\begin{longtable}{|p{0.2\textwidth}|p{0.2\textwidth}|p{0.58\textwidth}|}
		\caption{Beschreibung der Felder in einem JSON-Objekt aus der Liste der Lehrer}
\endfoot
		\caption{Beschreibung der Felder in einem JSON-Objekt aus der Liste der Lehrer}
		\label{tab:rest:api:classes:id:teachers:read:ret}
\endlastfoot 
\hline
			\textbf{Feldname} & \textbf{Datentyp} & \textbf{Beschreibung} \\ \hline
\endhead
user & STRING & Die User id des Lehrers \\ \hline
start & CALENDARDATE & Optional, wir benötigt wenn der Lehrer später zu eine Klasse als Klassenlehrer zugewiesen wird. \\ \hline
end & CALENDARDATE & Optional, wir benötigt wenn der Lehrer vorzeitig vor ende des Zeitraums der Klasse nicht mehr der Klasse als Klassenlehrer zugewiesen ist. \\ \hline
order & Object List & Eine liste mit Objekten die angibt welche Position der Lehrer von wann bis wann in der Klasse inne hatte, niedriger Nummer heißt höhere Position. \\ \hline
order.order & NUMBER & Die Ordnungsnummer welche angibt welche Position der Lehre in einer Klasse inne hat, beginnt immer bei 1. Es können auch mehre Lehrer die selbe Nummer haben wenn die Klasse zum Beispiel mehre Gleichberechtigte Klassenlehrer hat. \\ \hline
order.start & CALENDARDATE & Optional, Zeitpunkt ab wann der Lehrer die Position Innen hat. \\ \hline
order.end & CALENDARDATE & Optional, Zeitpunkt bis wann der Lehrer die Position Innen hatte. \\ \hline
\end{longtable}


\begin{longtable}{|c|p{0.7\textwidth}|}
\caption{Berechtigungen auf dem Endpunkt}
\endfoot
		\caption{Berechtigungen auf dem Endpunkt}
		\label{tab:rest:api:classes:id:teachers:read:right}
\endlastfoot
\hline
\textbf{Benutzergruppen} & \textbf{Zugelassene Daten} \\ \hline
\endhead
guest & Darf den Endpunkt nicht aufrufen und keine Daten vom Endpunkt erhalten. \\ \hline
user &  \\ \hline 
teachers & \\ \hline
external-teachers & \\ \hline
guardians & \\ \hline
teacher & \\ \hline
principal & \\ \hline
school-admin & \\ \hline
school-board & \\ \hline
fed-school-board & \\ \hline
sync-systems & \\ \hline
	\end{longtable}