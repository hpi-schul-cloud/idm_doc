\subsubsection{READ}
\label{sec:rest:api:classes:id:teachers:read}
Es sind nur Anfragen mit der HTTP-GET-Methode für ein READ auf die Daten zugelassen.
Bei Anfragen an diesen Endpunkt wird eine Liste von IDs von Klassenlehrern zurückgegeben, die der Klasse als Klassenlehrer zugeordnet sind.
Optional ist der Eintritts- sowie Austrittszeitpunkt. Diese Felder werden gesetzt, falls der Klassenlehrer dies nicht über den kompletten Zeitraum der Klasse war.

Die Antwort erfolgt mit dem JSON-Objekt \reflisting{lst:code:rest:api:classes:id:teachers:read:ret}. 
Die einzelnen Felder der Antwort werden in \reftabllec{tab:rest:api:classes:id:teachers:read:ret} beschrieben.
\lstset{
  language=json,
  tabsize=2,
  captionpos=b,
  numbers=left,
  commentstyle=\color{green},
  backgroundcolor=\color{white},
  numberstyle=\color{gray},
  keywordstyle=\color{blue} \textbf,%otherkeywords={xdata},
  keywordstyle=[2]\color{red}\textbf,
  identifierstyle=\color{black},
  stringstyle=\color{red}\ttfamily,
  basicstyle = \ttfamily \color{black} \footnotesize,
  showstringspaces=false,
	breakatwhitespace=false,         % sets if automatic breaks should only happen at whitespace
  breaklines=true, 
}
\begin{lstlisting}[caption={JSON-Antwort für einen GET-Aufruf des Pfads /api/classes/\$id/teachers},label={lst:code:rest:api:classes:id:teachers:read:ret},frame=tlrb]
[ 
{ 
	user: "<STRING>",
  	start: "<DATE>",
  	end: "<DATE>",
	order: [
	{
		order: "<NUMBER>",
    	start: "<DATE>",
    	end: "<DATE>"
	},
	 ...
	],
},
 ... 
]
\end{lstlisting}

\begin{longtable}{|p{0.2\textwidth}|p{0.2\textwidth}|p{0.58\textwidth}|}
		\caption{Beschreibung der Felder in einem JSON-Objekt aus der Liste der Lehrer}
\endfoot
		\caption{Beschreibung der Felder in einem JSON-Objekt aus der Liste der Lehrer}
		\label{tab:rest:api:classes:id:teachers:read:ret}
\endlastfoot 
\hline
			\textbf{Feldname} & \textbf{Datentyp} & \textbf{Beschreibung} \\ \hline
\endhead
user & STRING & Die User-ID des Lehrers, die Daten des Benutzers können unter /api/users/\$id abgerufen werden  \\ \hline
start & DATE & Optional; Eine Datumsangabe nach ISO-8601 im Format YYYY-MM-DD. Wird gesetzt, falls der Lehrer erst während des Zeitraums einer Klasse als Klassenlehrer zugewiesen wird. \\ \hline
end & DATE & Optional; Eine Datumsangabe nach ISO-8601 im Format YYYY-MM-DD. Wird gesetzt, falls der Lehrer vor Ende des Zeitraums der Klasse nicht mehr der Klasse als Klassenlehrer zugewiesen ist. \\ \hline
order & Object List & Eine Liste mit Objekten, die angibt, welche Position der Lehrer von wann bis wann in der Klasse innehatte. Niedrigere Nummer heißt höhere Position. \\ \hline
order.order & NUMBER & Die Ordnungsnummer, welche angibt welche Position der Lehrer in einer Klasse inne hat. Beginnt immer bei 1. Es können auch mehre Lehrer dieselbe Nummer haben, falls die Klasse zum Beispiel mehrere gleichberechtigte Klassenlehrer hat. \\ \hline
order.start & DATE & Optional; Zeitpunkt, ab dem der Lehrer die Position innehatte. \\ \hline
order.end & DATE & Optional; Zeitpunkt, bis zu dem der Lehrer die Position innehatte. \\ \hline
\end{longtable}
