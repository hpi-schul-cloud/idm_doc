\subsubsection{READ}
\label{sec:rest:api:users:id:assignments:read}
Es sind nur Anfragen mit der HTTP-GET-Methode für ein READ auf die Daten zugelassen.
Sie gibt für den per ID ausgewählten Benutzer seine Zuordnungen im System wieder.
Für diese Daten gelten die Einschränkungen die durch den Kontext des anfragenden Benutzers gegeben sind.

Die Antwort erfolgt mit dem JSON-Objekt \reflisting{lst:code:rest:api:users:id:assignments:read:ret}. 
Die einzelnen Felder der Antwort werden in \reftabllec{tab:rest:api:users:id:assignments:read:ret} beschrieben.
\lstset{
  language=json,
  tabsize=2,
  captionpos=b,
  numbers=left,
  commentstyle=\color{green},
  backgroundcolor=\color{white},
  numberstyle=\color{gray},
  keywordstyle=\color{blue} \textbf,%otherkeywords={xdata},
  keywordstyle=[2]\color{red}\textbf,
  identifierstyle=\color{black},
  stringstyle=\color{red}\ttfamily,
  basicstyle = \ttfamily \color{black} \footnotesize,
  showstringspaces=false,
	breakatwhitespace=false,         % sets if automatic breaks should only happen at whitespace
  breaklines=true, 
}
\begin{lstlisting}[caption={JSON-Antwort für einen GET-Aufruf des Pfads /api/users/\$id/assignments},label={lst:code:rest:api:users:id:assignments:read:ret},frame=tlrb]
 {
  school: "<STRING>",
	role: "<STRING>",
	start: "<CALENDARDATE>",
	end: "<CALENDARDATE>",
	school-years: ["<STRING>",...]
 },
 ...
\end{lstlisting}

\begin{longtable}{|p{0.2\textwidth}|p{0.2\textwidth}|p{0.58\textwidth}|}
		\caption{Beschreibung der Felder in einem JSON-Objekt für das Zuordnen eines Benutzer in einer Rolle zu einer Schule}
\endfoot
		\caption{Beschreibung der Felder in einem JSON-Objekt für das Zuordnen eines Benutzer in einer Rolle zu einer Schule}
		\label{tab:rest:api:users:id:assignments:read:ret}
\endlastfoot 
\hline
			\textbf{Feldname} & \textbf{Datentyp} & \textbf{Beschreibung} \\ \hline
\endhead
school & STRING & Zeichenkette, welche eine ID von einer Schule ist, mit der diese unter /api/schools/\$id abgerufen werden kann. \\ \hline
role & STRING & Rolle, die der Benutzer innehat. \\ \hline
start & CALENDARDATE & Datum, ab dem der Benutzer die Rolle innehat. \\ \hline
end & CALENDARDATE & Optional; Gibt an, bis zu welchen Datum der Benutzer die Rolle innehatte. \\ \hline
school-years & Liste & Optional; Liste von Zeichenketten, welche Referenzen auf Schuljahre sind, die unter /api/school-years/\$id abgerufen werden können. \\ \hline 
\end{longtable}
