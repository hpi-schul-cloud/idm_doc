\subsection{Endpunkt in der REST-API: /api/users/\$id}
label{sec:end:rest:api:users}
An diesem Endpunkt werden Informationen zu Benutzern verarbeitet. Eine Benutzer beschreibt eine Person, die innerhalb einer Schule (s. \refabschnitt{sec:end:rest:api:schools}) in einem definierten Zeitraum eine bestimmte Rolle innehat, Teil einer Klasse/Klassenstufe ist (s. \refabschnitt{sec:end:rest:api:classes}) oder auch an der Unterrichtsdurchführung teilnimmt (s. \refabschnitt{sec:end:rest:api:subjects}). Die Person kann zudem in einer bestimmten Relation zu einer anderen im Kontext 'Schule' agierenden Person stehen.

Die \reftabllec{tab:rest:api:users:id:meth} listet auf, welche Operationen zugelassen sind und welche HTTP-Methoden dabei verwendet werden. 

\begin{table}[!htbp]
	\begin{tabular}{|c|c|c|}
		\hline
			\textbf{Operation} & \textbf{Zugelassen?} & \textbf{HTTP-Methode} \\ \hline
			CREATE & Nein &  \\ \hline 
			READ & Ja & GET \\ \hline
			UPDATE & Nein & \\ \hline 
			DELETE & Nein & \\ \hline
	\end{tabular}

		\caption{Zugelassene Operationen auf /api/users/\$id}
		\label{tab:rest:api:users:id:meth}
\end{table}

\subsubsection{READ}
\label{sec:rest:api:users:id:read}
Es sind nur Anfragen mit der HTTP-GET-Methode für ein READ auf die Daten zugelassen.
Die Route gibt für den per ID ausgewählten Benutzer seine persönlichen Daten wieder.
Diese können eingeschränkt werden durch den Kontext, in dem sich der anfragende Benutzer befinden.

Die Antwort erfolgt mit dem JSON-Objekt \reflisting{lst:code:rest:api:users:id:read:ret}. 
Die einzelnen Felder der Antwort werden in \reftabllec{tab:rest:api:users:id:read:ret} beschrieben.
Die Berechtigungen auf den Endpunkt können \reftabllec{tab:rest:api:users:id:read:right} entnommen werden.
\lstset{
  language=json,
  tabsize=2,
  captionpos=b,
  numbers=left,
  commentstyle=\color{green},
  backgroundcolor=\color{white},
  numberstyle=\color{gray},
  keywordstyle=\color{blue} \textbf,%otherkeywords={xdata},
  keywordstyle=[2]\color{red}\textbf,
  identifierstyle=\color{black},
  stringstyle=\color{red}\ttfamily,
  basicstyle = \ttfamily \color{black} \footnotesize,
  showstringspaces=false,
	breakatwhitespace=false,         % sets if automatic breaks should only happen at whitespace
  breaklines=true, 
}
\begin{lstlisting}[caption={JSON-Antwort für einen GET-Aufruf der Route /api/users/\$id},label={lst:code:rest:api:users:id:read:ret},frame=tlrb]
{
 id: "<STRING>",
 name: "<STRING>",
 surename: "<STRING>",
 dateofbirth: "<CALENDARDATE>",
 sex: "<ENUM>",
}
\end{lstlisting}

\begin{longtable}{|p{0.2\textwidth}|p{0.2\textwidth}|p{0.58\textwidth}|}
		\caption{Beschreibung der Felder in einem JSON-Objekt von ein Benutzer}
\endfoot
		\caption{Beschreibung der Felder in einem JSON-Objekt von ein Benutzer}
		\label{tab:rest:api:users:id:read:ret}
\endlastfoot 
\hline
			\textbf{Feldname} & \textbf{Datentyp} & \textbf{Beschreibung} \\ \hline
\endhead
id & STRING & ID des Benutzers mit im IDM, damit können unter /api/users/\$id weiter Daten Abgefragt werden \\ \hline
name & STRING & Vorname des Benutzers \\ \hline
surname & STRING & Nachname des Benutzers \\ \hline
dateofbirth & CALENDARDATE & Geburtstag des Benutzers \\ \hline
sex & ENUM & Geschlecht des Benutzers 
\begin{itemize}
	\item 0 steht für Divers
	\item 1 steht für Weiblich
	\item 2 steht für Männlich
\end{itemize}
 \\ \hline
\end{longtable}