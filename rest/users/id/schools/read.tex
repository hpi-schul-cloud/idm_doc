\subsubsection{READ}
\label{sec:rest:api:users:id:schools:read}
Es sind nur Anfragen mit der HTTP-GET-Methode für ein READ auf die Daten zugelassen.
Sie gibt für den per ID ausgewählten Benutzer seine Zuordnungen im System wieder.
Für diese Daten gelten die Einschränkungen die durch den Kontext des anfragenden Benutzers gegeben sind.

Die Antwort erfolgt mit dem JSON-Objekt \reflisting{lst:code:rest:api:users:id:schools:read:ret}. 
Die einzelnen Felder der Antwort werden in \reftabllec{tab:rest:api:users:id:schools:read:ret} beschrieben.
\lstset{
  language=json,
  tabsize=2,
  captionpos=b,
  numbers=left,
  commentstyle=\color{green},
  backgroundcolor=\color{white},
  numberstyle=\color{gray},
  keywordstyle=\color{blue} \textbf,%otherkeywords={xdata},
  keywordstyle=[2]\color{red}\textbf,
  identifierstyle=\color{black},
  stringstyle=\color{red}\ttfamily,
  basicstyle = \ttfamily \color{black} \footnotesize,
  showstringspaces=false,
	breakatwhitespace=false,         % sets if automatic breaks should only happen at whitespace
  breaklines=true, 
}
\begin{lstlisting}[caption={JSON-Antwort für einen GET-Aufruf des Pfads /api/users/\$id/schools},label={lst:code:rest:api:users:id:schools:read:ret},frame=tlrb]
[
{
 		school: {
 			id: "<STRING>",
 			displayName: "<STRING>"
 		},
 		role: "<STRING>",
 		start: "<DATE>",
 		end: "<DATE>"
},
...
]
\end{lstlisting}

\begin{longtable}{|p{0.2\textwidth}|p{0.2\textwidth}|p{0.58\textwidth}|}
		\caption{Beschreibung der Felder in einem JSON-Objekt für das Zuordnen eines Benutzer in einer Rolle zu einer Schule}
\endfoot
		\caption{Beschreibung der Felder in einem JSON-Objekt für das Zuordnen eines Benutzer in einer Rolle zu einer Schule}
		\label{tab:rest:api:users:id:schools:read:ret}
\endlastfoot 
\hline
			\textbf{Feldname} & \textbf{Datentyp} & \textbf{Beschreibung} \\ \hline
\endhead
school & Object & Objekt mit Informationen zur Schule, zu der die Klasse gehört. Der Aufbau des Objektes kann \reftabllec{tab:rest:api:users:id:schools:school} entnommen werden. \\ \hline
role & STRING & Rolle, welche der Benutzer an der Schule innehat. Werte dafür können \reftabllec{tab:intro:rolesschool} entnommen werden. \\ \hline
start & DATE & OPTIONAL; eine Datumsangabe nach ISO-8601 in dem Format YYYY-MM-DD. Gibt Datum an, ab dem der Benutzer die Rolle an der Schule innehat. \\ \hline
end & DATE & OPTIONAL; eine Datumsangabe nach ISO-8601 in dem Format YYYY-MM-DD. Gibt Datum an, bis zu dem der Benutzer die Rolle an der Schule innehat. \\ \hline
\end{longtable}

\begin{longtable}{|p{0.2\textwidth}|p{0.2\textwidth}|p{0.58\textwidth}|}
        \caption{Beschreibung der Felder im JSON-Objekt für die Schule }
\endfoot
        \caption{Beschreibung der Felder im JSON-Objekt für die Schule }
        \label{tab:rest:api:users:id:schools:school}
\endlastfoot 
\hline
            \textbf{Feldname} & \textbf{Datentyp} & \textbf{Beschreibung} \\ \hline
\endhead
id & STRING & ID der Schule, mit der unter /api/schools/\$id weitere Informationen abgefragt werden können. \\ \hline
displayName & STRING & OPTIONAL; Anzeigename der Schule \\ \hline
\end{longtable}