\subsubsection{READ}
\label{sec:rest:api:users:id:guardians:read}
Es sind nur Anfragen mit der HTTP-GET-Methode für ein READ auf die Daten zugelassen.
Die Route gibt für den per ID ausgewählten Benutzer die Liste mit den IDs seiner Eltern, Erziehungsberechtigen und Vormünder wieder.
Dies geschieht im Kontext der Informationen die der anfragende Nutzer sehen darf.

Die Antwort erfolgt mit dem JSON-Objekt \reflisting{lst:code:rest:api:users:id:guardians:read:ret}. 
Die einzelnen Felder der Antwort werden in \reftabllec{tab:rest:api:users:id:guardians:read:ret} beschrieben.
Die Berechtigungen auf den Endpunkt können \reftabllec{tab:rest:api:users:id:guardians:read:right} entnommen werden.
\lstset{
  language=json,
  tabsize=2,
  captionpos=b,
  numbers=left,
  commentstyle=\color{green},
  backgroundcolor=\color{white},
  numberstyle=\color{gray},
  keywordstyle=\color{blue} \textbf,%otherkeywords={xdata},
  keywordstyle=[2]\color{red}\textbf,
  identifierstyle=\color{black},
  stringstyle=\color{red}\ttfamily,
  basicstyle = \ttfamily \color{black} \footnotesize,
  showstringspaces=false,
	breakatwhitespace=false,         % sets if automatic breaks should only happen at whitespace
  breaklines=true, 
}
\begin{lstlisting}[caption={JSON-Antwort für einen GET-Aufruf der Route /api/user/\$id/guardians},label={lst:code:rest:api:users:id:guardians:read:ret},frame=tlrb]
[
 id: "<STRING>",
...
]
\end{lstlisting}


\begin{longtable}{|p{0.2\textwidth}|p{0.2\textwidth}|p{0.58\textwidth}|}
		\caption{Beschreibung der Felder in einem JSON-Objekt für das Zuordnen eines Benutzer in einer Rolle zu einer Schule}
\endfoot
		\caption{Beschreibung der Felder in einem JSON-Objekt für das Zuordnen eines Benutzer in einer Rolle zu einer Schule}
		\label{tab:rest:api:users:id:guardians:read:ret}
\endlastfoot 
\hline
			\textbf{Feldname} & \textbf{Datentyp} & \textbf{Beschreibung} \\ \hline
\endhead
 & STRING & ID des Erziehungsberechtigten, mit der unter /api/users/\$id weitere Informationen abgefragt werden können.	\\ \hline
\end{longtable}