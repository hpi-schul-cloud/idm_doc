\subsubsection{READ}
\label{sec:rest:api:users:id:subjects:read}
Es sind nur Anfragen mit der HTTP-GET-Methode für ein READ auf die Daten zugelassen.
Gibt für den per ID ausgewählten Benutzer im Kontext des anfragenden Benutzers eine Liste mit Schulfächern für den ausgewählten Benutzer zurück.
Die Antwort erfolgt mit dem JSON-Objekt \reflisting{lst:code:rest:api:users:id:subjects:read:ret}. 
Die einzelnen Felder der Antwort werden in \reftabllec{tab:rest:api:users:id:subjects:read:ret} beschrieben.
\lstset{
  language=json,
  tabsize=2,
  captionpos=b,
  numbers=left,
  commentstyle=\color{green},
  backgroundcolor=\color{white},
  numberstyle=\color{gray},
  keywordstyle=\color{blue} \textbf,%otherkeywords={xdata},
  keywordstyle=[2]\color{red}\textbf,
  identifierstyle=\color{black},
  stringstyle=\color{red}\ttfamily,
  basicstyle = \ttfamily \color{black} \footnotesize,
  showstringspaces=false,
	breakatwhitespace=false,         % sets if automatic breaks should only happen at whitespace
  breaklines=true, 
}
\begin{lstlisting}[caption={JSON-Antwort für einen GET-Aufruf des Pfads /api/users/\$id/subjects},label={lst:code:rest:api:users:id:subjects:read:ret},frame=tlrb]
[
{
	id: "<STRING>",
	displayName: "<STRING>",
	referenceSubject: [
	{
		id: "<STRING>",
		displayName: "<STRING>"
	},
	...
	],
	school: {
		id: "<STRING>",
		displayName: "<STRING>"
	},
	schoolPeriod: {
		id: "<STRING>",
		displayName: "<STRING>"
	},
	roleWithinSubject: "<STRING>",
	start: "<DATE>",
	end: "<DATE>" 		
},
...
]
\end{lstlisting}

\begin{longtable}{|p{0.2\textwidth}|p{0.2\textwidth}|p{0.58\textwidth}|}
		\caption{Beschreibung der Zeichenkette in der JSON-Liste für Unterrichtsfächer eines Benutzers}
\endfoot
		\caption{Beschreibung der Zeichenkette in der JSON-Liste für Unterrichtsfächer eines Benutzers}
		\label{tab:rest:api:users:id:subjects:read:ret}
\endlastfoot 
\hline
			\textbf{Feldname} & \textbf{Datentyp} & \textbf{Beschreibung} \\ \hline
\endhead
id & STRING & ID des Unterrichtsfaches, mit der unter /api/subjects/\$id weiter Informationen abgefragt werden können. \\ \hline
displayName & STRING & OPTIONAL; Anzeigenamen des Unterrichtsfaches \\ \hline
referenceSubject & Array of Objects & Array mit Objekten zu Referenzschulfächern. Der Aufbau der Objekte kann \reftabllec{tab:rest:api:subjects:id:reference-subjects} entnommen werden. \\ \hline
school & Object & Objekt mit Informationen zu der Schule, an der das Unterrichtsfach angeboten wird. \\ \hline
school.id & STRING & ID der Schule, mit der unter /api/schools/\$id weitere Informationen abgefragt werden können. \\ \hline
school.displayName & STRING & OPTIONAL; Anzeigename für die Schule \\ \hline
schoolPeriod & Object & Objekt mit Informationen zum Schuljahr, in welche der Benutzer das Unterrichtsfach besucht hat. \\ \hline
schoolPeriod.id & STRING & ID des Schuljahres, mit der unter /api/school-periods/\$id weitere Informationen abgefragt werden können. \\ \hline
schoolPeriod.displayName & STRING & OPTIONAL; Anzeigename des Schuljahres \\\hline
roleWithinSubject & STRING & Rolle, welche der Benutzer in dem Unterrichtsfach innehat. Werte dafür können \reftabllec{tab:intro:rolessubject} entnommen werden. \\ \hline
start & DATE & OPTIONAL; eine Datumsangabe nach ISO-8601 in dem Format YYYY-MM-DD. Gibt Datum an, ab dem der Benutzer das Unterrichtsfach hat.\\ \hline
end & DATE & OPTIONAL; eine Datumsangabe nach ISO-8601 in dem Format YYYY-MM-DD. Gibt Datum an, bis zu dem der Benutzer das Unterrichtsfach hat. \\ \hline
\end{longtable}
