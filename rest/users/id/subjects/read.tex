\subsubsection{READ}
\label{sec:rest:api:users:id:subjects:read}
Es sind nur Anfragen mit der HTTP-GET-Methode für ein READ auf die Daten zugelassen.
Gibt für den per ID ausgewählten Benutzer im Kontext des anfragenden Benutzers eine Liste mit Schulfächern für den ausgewählten Benutzer zurück.
Die Antwort erfolgt mit dem JSON-Objekt \reflisting{lst:code:rest:api:users:id:subjects:read:ret}. 
Die einzelnen Felder der Antwort werden in \reftabllec{tab:rest:api:users:id:subjects:read:ret} beschrieben.
Die Berechtigungen auf den Endpunkt können \reftabllec{tab:rest:api:users:id:subjects:read:right} entnommen werden.
\lstset{
  language=json,
  tabsize=2,
  captionpos=b,
  numbers=left,
  commentstyle=\color{green},
  backgroundcolor=\color{white},
  numberstyle=\color{gray},
  keywordstyle=\color{blue} \textbf,%otherkeywords={xdata},
  keywordstyle=[2]\color{red}\textbf,
  identifierstyle=\color{black},
  stringstyle=\color{red}\ttfamily,
  basicstyle = \ttfamily \color{black} \footnotesize,
  showstringspaces=false,
	breakatwhitespace=false,         % sets if automatic breaks should only happen at whitespace
  breaklines=true, 
}
\begin{lstlisting}[caption={JSON-Antwort für einen GET-Aufruf der Route /api/users/\$id/subjects},label={lst:code:rest:api:users:id:subjects:read:ret},frame=tlrb]
[
  "<STRING>",
	...
]
\end{lstlisting}

\begin{longtable}{|p{0.2\textwidth}|p{0.2\textwidth}|p{0.58\textwidth}|}
		\caption{Beschreibung der Zeichenkette in der JSON-Liste für Unterrichtsfächer eines Benutzers}
\endfoot
		\caption{Beschreibung der Zeichenkette in der JSON-Liste für Unterrichtsfächer eines Benutzers}
		\label{tab:rest:api:users:id:subjects:read:ret}
\endlastfoot 
\hline
			\textbf{Feldname} & \textbf{Datentyp} & \textbf{Beschreibung} \\ \hline
\endhead
			 & STRING &  Referens auf das Unterrichtsfach, kann als ID im Pfad /api/subjects/\$id verwendet werden.  \\ \hline
\end{longtable}


\begin{longtable}{|c|p{0.7\textwidth}|}
\caption{Berechtigungen auf dem Endpunkt}
\endfoot
		\caption{Berechtigungen auf dem Endpunkt}
		\label{tab:rest:api:users:id:subjects:read:right}
\endlastfoot
\hline
\textbf{Benutzergruppen} & \textbf{Zugelassene Daten} \\ \hline
\endhead
guest & Darf den Endpunkt nicht aufrufen und keine Daten vom Endpunkt erhalten. \\ \hline
user &  \\ \hline 
students & \\ \hline
external-students & \\ \hline
guardians & \\ \hline
teacher & \\ \hline
principal & \\ \hline
school-admin & \\ \hline
school-board & \\ \hline
fed-school-board & \\ \hline
sync-systems & \\ \hline
	\end{longtable}