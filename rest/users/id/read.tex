\subsubsection{READ}
\label{sec:rest:api:users:id:read}
Es sind nur Anfragen mit der HTTP-GET-Methode für ein READ auf die Daten zugelassen.
Der Pfad gibt für den per ID ausgewählten Benutzer seine persönlichen Daten wieder. Das gelieferte JSON beinhaltet neben den persönlichen Informationen auch alle Rollen in Bezug auf Schule, Klasse, Unterrichtsfächer und andere Benutzer.

Die Antwort erfolgt mit dem JSON-Objekt \reflisting{lst:code:rest:api:users:id:read:ret}. 
Die einzelnen Felder der Antwort werden in \reftabllec{tab:rest:api:users:id:read:ret} beschrieben.
\lstset{
  language=json,
  tabsize=2,
  captionpos=b,
  numbers=left,
  commentstyle=\color{green},
  backgroundcolor=\color{white},
  numberstyle=\color{gray},
  keywordstyle=\color{blue} \textbf,%otherkeywords={xdata},
  keywordstyle=[2]\color{red}\textbf,
  identifierstyle=\color{black},
  stringstyle=\color{red}\ttfamily,
  basicstyle = \ttfamily \color{black} \footnotesize,
  showstringspaces=false,
	breakatwhitespace=false,         % sets if automatic breaks should only happen at whitespace
  breaklines=true, 
}
\begin{lstlisting}[caption={JSON-Antwort für einen GET-Aufruf des Pfads /api/users/\$id},label={lst:code:rest:api:users:id:read:ret},frame=tlrb]
{
	id: "<STRING>",
	student-id: "<STRING>"
 	first-name: "<STRING>",
 	preferred-name: "<STRING>",
 	last-name: "<STRING>",
 	date-of-birth: "<DATE>",
 	sex: "<ENUM>",
 	roles: [
 	{
 		id: "<STRING>",
 		display-name: "<STRING>",
 		role: "<STRING>",
 		start: "<DATE>",
 		end: "<DATE>"
 	},
 	...
 	],
 	classes: [
 	{
 		id: "<STRING>",
 		display-name: "<STRING>",
 		summary: "<STRING>",
 		school: {
 			id: "<STRING>",
 			display-name: "<STRING>"
 		},
 		school-year: {
 			id: "<STRING>",
 			display-name: "<STRING>"
 		},
 		roleWithinClass: "<STRING>",
 		start: "<DATE>",
 		end: "<DATE>"
 	},
 	...
	],
	subjects: [
	{
		id: "<STRING>",
		display-name: "<STRING>",
		school-subject: [
 		{
 			id: "<STRING>",
 			display-name: "<STRING>"
 		},
 		...
 		],
 		school: {
 			id: "<STRING>",
 			display-name: "<STRING>"
 		},
 		school-year: {
 			id: "<STRING>",
 			display-name: "<STRING>"
 		},
 		roleWithinSubject: "<STRING>",
 		start: "<DATE>",
 		end: "<DATE>" 		
	},
	...
	],
	guardians: [
	{
		id: "<STRING>",
		role: "<STRING>",
		user: {
			id: "<STRING>",
			display-name: "<STRING>",
		}
	},
	...
	]
}
\end{lstlisting}

\begin{longtable}{|p{0.2\textwidth}|p{0.2\textwidth}|p{0.58\textwidth}|}
		\caption{Beschreibung der Felder in einem JSON-Objekt von einem Benutzer}
\endfoot
		\caption{Beschreibung der Felder in einem JSON-Objekt von einem Benutzer}
		\label{tab:rest:api:users:id:read:ret}
\endlastfoot 
\hline
			\textbf{Feldname} & \textbf{Datentyp} & \textbf{Beschreibung} \\ \hline
\endhead
id & STRING & ID des Benutzers, mit der unter /api/users/\$id weitere Informtionen abgefragt werden können. \\ \hline
student-id & STRING & Landesweit gültige Schüler-Identifikationsnummer \\ \hline
name & STRING & Vorname des Benutzers \\ \hline
last-name & STRING & Nachname des Benutzers \\ \hline
date-of-birth & DATE & Geburtstag des Benutzers \\ \hline
sex & ENUM & Geschlecht des Benutzers 
\begin{itemize}
	\item divers steht für divers
	\item female steht für weiblich
	\item male steht für männlich
\end{itemize}
 \\ \hline
roles &  & Liste von Rollen in Bezug auf eine Schule \\ \hline
classes &  & Liste von Rollen in Bezug auf eine Klasse \\ \hline
subjects &  & Liste von Rollen in Bezug auf eine Unterrichtsfach \\ \hline
guardians &  & Liste von Rollen in Bezug auf einen anderen Benutzer \\ \hline
\end{longtable}
