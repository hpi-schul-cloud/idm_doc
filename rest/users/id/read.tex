\subsubsection{READ}
\label{sec:rest:api:users:id:read}
Es sind nur Anfragen mit der HTTP-GET-Methode für ein READ auf die Daten zugelassen.
Der Pfad gibt für den per ID ausgewählten Benutzer seine persönlichen Daten wieder. Das gelieferte JSON beinhaltet neben den persönlichen Informationen auch alle Rollen in Bezug auf Schule, Klasse, Unterrichtsfächer und andere Benutzer.

Die Antwort erfolgt mit dem JSON-Objekt \reflisting{lst:code:rest:api:users:id:read:ret}. 
Die einzelnen Felder der Antwort werden in \reftabllec{tab:rest:api:users:id:read:ret} beschrieben.
\lstset{
  language=json,
  tabsize=2,
  captionpos=b,
  numbers=left,
  commentstyle=\color{green},
  backgroundcolor=\color{white},
  numberstyle=\color{gray},
  keywordstyle=\color{blue} \textbf,%otherkeywords={xdata},
  keywordstyle=[2]\color{red}\textbf,
  identifierstyle=\color{black},
  stringstyle=\color{red}\ttfamily,
  basicstyle = \ttfamily \color{black} \footnotesize,
  showstringspaces=false,
	breakatwhitespace=false,         % sets if automatic breaks should only happen at whitespace
  breaklines=true, 
}
\begin{lstlisting}[caption={JSON-Antwort für einen GET-Aufruf des Pfads /api/users/\$id},label={lst:code:rest:api:users:id:read:ret},frame=tlrb]
{
	id: "<STRING>",
	studentId: "<STRING>"
 	firstName: "<STRING>",
 	preferredName: "<STRING>",
 	lastName: "<STRING>",
 	dateOfBirth: "<DATE>",
 	salutation: "<ENUM>",
 	grade: "<STRING>",
 	schools: [
 	{
 		school: {
 			id: "<STRING>",
 			displayName: "<STRING>"
 		},
 		role: "<STRING>",
 		substitute: "<BOOLEAN",
 		start: "<DATE>",
 		end: "<DATE>"
 	},
 	...
 	],
 	classes: [
 	{
 		id: "<STRING>",
 		displayName: "<STRING>",
 		school: {
 			id: "<STRING>",
 			displayName: "<STRING>"
 		},
 		schoolPeriod: {
 			id: "<STRING>",
 			displayName: "<STRING>"
 		},
 		roleWithinClass: "<STRING>",
 		substitute: "<BOOLEAN",
 		start: "<DATE>",
 		end: "<DATE>"
 	},
 	...
	],
	subjects: [
	{
		id: "<STRING>",
		displayName: "<STRING>",
		referenceSubject: [
 		{
 			id: "<STRING>",
 			displayName: "<STRING>"
 		},
 		...
 		],
 		school: {
 			id: "<STRING>",
 			displayName: "<STRING>"
 		},
 		schoolPeriod: {
 			id: "<STRING>",
 			displayName: "<STRING>"
 		},
 		roleWithinSubject: "<STRING>",
 		substitute: "<BOOLEAN",
 		start: "<DATE>",
 		end: "<DATE>" 		
	},
	...
	],
	guardians: [
	{
		id: "<STRING>",
		role: "<STRING>",
		user: {
			id: "<STRING>",
			firstName: "<STRING>",
			lastName: "<STRING>"		
		}
	},
	...
	],
	status: "<ENUM>",
	deleteOn: "<DATE>"
}
\end{lstlisting}

\begin{longtable}{|p{0.18\textwidth}|p{0.18\textwidth}|p{0.52\textwidth}|}
		\caption{Beschreibung der Felder in einem JSON-Objekt von einem Benutzer}
\endfoot
		\caption{Beschreibung der Felder in einem JSON-Objekt von einem Benutzer}
		\label{tab:rest:api:users:id:read:ret}
\endlastfoot 
\hline
			\textbf{Feldname} & \textbf{Datentyp} & \textbf{Beschreibung} \\ \hline
\endhead
id & STRING & ID des Benutzers, mit der unter /api/users/\$id weitere Informtionen abgefragt werden können. \\ \hline
studentId & STRING & OPTIONAL; Landesweit gültige Schüler-Identifikationsnummer \\ \hline
name & STRING & Vorname des Benutzers \\ \hline
preferredName & STRING & Bevorzugter Vorname des Benutzers \\ \hline
lastName & STRING & Nachname des Benutzers \\ \hline
dateOfBirth & DATE & eine Datumsangabe nach ISO-8601 in dem Format YYYY-MM-DD. Geburtstag des Benutzers \\ \hline
salutation & ENUM & Anrede des Benutzers 
\begin{itemize}
	\item divers steht für divers
	\item female steht für weiblich
	\item male steht für männlich
\end{itemize}
 \\ \hline
grade & STRING & Klassen- bzw. Jahrgangsstufe des Benutzers \\ \hline 
schools & Array of Objects & Liste von Rollen in Bezug auf eine Schule. Der Aufbau der Objekte kann \reftabllec{tab:rest:api:users:id:schools:read:ret} entnommen werden. \\ \hline
classes & Array of Objects & Liste von Rollen in Bezug auf eine Klasse. Der Aufbau der Objekte kann \reftabllec{tab:rest:api:users:id:classes:read:ret} entnommen werden. \\ \hline
subjects & Array of Objects & Liste von Unterrichtsfächern und den Rollen, welcher der Benutzer in diesen innehat. Der Aufbau der Objekte kann \reftabllec{tab:rest:api:users:id:subjects:read:ret} entnommen werden. \\ \hline
guardians & Array of Objects & Liste von direkten Verbindungen zu einem anderen Benutzer. Der Aufbau der Objekte kann \reftabllec{tab:rest:api:users:id:guardians:read:ret} entnommen werden. \\ \hline
status & ENUM & 
\begin{itemize}
	\item active steht für aktiver Benutzer
	\item inactive steht für inaktiver Benutzer
\end{itemize}
 \\ \hline
deleteOn & DATE & OPTIONAL; eine Datumsangabe nach ISO-8601 in dem Format YYYY-MM-DD. Datum, an dem die Daten des Benutzers gelöscht werden sollen. \\ \hline
\end{longtable}

Darüber hinaus kann der OpenID-Connect-Endpunkt UserInfo dazu genutzt werden, die genannten Daten für den eigenen User bereitzustellen.