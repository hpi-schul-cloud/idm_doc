\subsubsection{READ}
\label{sec:rest:api:users:id:classes:read}
Es sind nur Anfragen mit der HTTP-GET-Methode für ein READ auf die Daten zugelassen.
Die Route gibt für den per ID ausgewählten Benutzer seine für ihn hinterlegten Klassen im System wieder.
Für diese Daten gelten die Einschränkungen die durch den Kontext des anfragenden Benutzers gegeben sind.

Die Antwort erfolgt mit dem JSON-Objekt \reflisting{lst:code:rest:api:users:id:classes:read:ret}. 
Die einzelnen Felder der Antwort werden in \reftabllec{tab:rest:api:users:id:classes:read:ret} beschrieben.
Die Berechtigungen auf den Endpunkt können \reftabllec{tab:rest:api:users:id:classes:read:right} entnommen werden.
\lstset{
  language=json,
  tabsize=2,
  captionpos=b,
  numbers=left,
  commentstyle=\color{green},
  backgroundcolor=\color{white},
  numberstyle=\color{gray},
  keywordstyle=\color{blue} \textbf,%otherkeywords={xdata},
  keywordstyle=[2]\color{red}\textbf,
  identifierstyle=\color{black},
  stringstyle=\color{red}\ttfamily,
  basicstyle = \ttfamily \color{black} \footnotesize,
  showstringspaces=false,
	breakatwhitespace=false,         % sets if automatic breaks should only happen at whitespace
  breaklines=true, 
}
\begin{lstlisting}[caption={JSON-Antwort für einen GET-Aufruf der Route /api/users/\$id/classes},label={lst:code:rest:api:users:id:classes:read:ret},frame=tlrb]
[
  {
	 class: "<STRING>",
   school: "<STRING>",
	 school-year: "<STRING>",
	 start: "<CALENDARDATE>",
	 end: "<CALENDARDATE>",
	}, 
	...
]
\end{lstlisting}

\begin{longtable}{|p{0.2\textwidth}|p{0.2\textwidth}|p{0.58\textwidth}|}
		\caption{Beschreibung der Felder in einem JSON-Objekt aus der Liste der Klassen eines Benutzers}
\endfoot
		\caption{Beschreibung der Felder in einem JSON-Objekt aus der Liste der Klassen eines Benutzers}
		\label{tab:rest:api:users:id:classes:read:ret}
\endlastfoot 
\hline
			\textbf{Feldname} & \textbf{Datentyp} & \textbf{Beschreibung} \\ \hline
\endhead
class & STRING & ID der Klasse im System. \\ \hline
school & STRING & ID, zu welcher Schule die Klasse gehört, die Schule kann damit unter /api/schools/\$id abgefragt werden. \\ \hline
school-year & STRING & Schuljahr, in welchem die Klasse existiert, die daten des Schuljahres können unter /api/school-years/\$id abgrufen werden. \\ \hline
start & CALENDARDATE & Zeitpunkt, ab wann das Klassenobjekt existiert, falls dies abweichend zum Startzeitpunkt des Schuljahres ist. \\ \hline 
end & CALENDARDATE & Zeitpunkt, bis wann das Klassenobjekt existiert, falls dies abweichend zum Endzeitpunkt des Schuljahres ist. \\ \hline 
\end{longtable}


\begin{longtable}{|c|p{0.7\textwidth}|}
\caption{Berechtigungen auf dem Endpunkt}
\endfoot
		\caption{Berechtigungen auf dem Endpunkt}
		\label{tab:rest:api:users:id:classes:read:right}
\endlastfoot
\hline
\textbf{Benutzergruppen} & \textbf{Zugelassene Daten} \\ \hline
\endhead
guest & Darf den Endpunkt nicht aufrufen und keine Daten vom Endpunkt erhalten. \\ \hline
user &  \\ \hline 
students & \\ \hline
external-students & \\ \hline
guardians & \\ \hline
teacher & \\ \hline
principal & \\ \hline
school-admin & \\ \hline
school-board & \\ \hline
fed-school-board & \\ \hline
sync-systems & \\ \hline
	\end{longtable}