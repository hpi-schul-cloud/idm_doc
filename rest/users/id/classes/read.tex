\subsubsection{READ}
\label{sec:rest:api:users:id:classes:read}
Es sind nur Anfragen mit der HTTP-GET-Methode für ein READ auf die Daten zugelassen.
Der Pfad gibt für den per ID ausgewählten Benutzer seine für ihn hinterlegten Klassen im System wieder.
Für diese Daten gelten die Einschränkungen die durch den Kontext des anfragenden Benutzers gegeben sind.

Die Antwort erfolgt mit dem JSON-Objekt \reflisting{lst:code:rest:api:users:id:classes:read:ret}. 
Die einzelnen Felder der Antwort werden in \reftabllec{tab:rest:api:users:id:classes:read:ret} beschrieben.
\lstset{
  language=json,
  tabsize=2,
  captionpos=b,
  numbers=left,
  commentstyle=\color{green},
  backgroundcolor=\color{white},
  numberstyle=\color{gray},
  keywordstyle=\color{blue} \textbf,%otherkeywords={xdata},
  keywordstyle=[2]\color{red}\textbf,
  identifierstyle=\color{black},
  stringstyle=\color{red}\ttfamily,
  basicstyle = \ttfamily \color{black} \footnotesize,
  showstringspaces=false,
	breakatwhitespace=false,         % sets if automatic breaks should only happen at whitespace
  breaklines=true, 
}
\begin{lstlisting}[caption={JSON-Antwort für einen GET-Aufruf des Pfads /api/users/\$id/classes},label={lst:code:rest:api:users:id:classes:read:ret},frame=tlrb]
[
{
	id: "<STRING>",
	displayName: "<STRING>",
	school: {
		id: "<STRING>",
		displayName: "<STRING>"
	},
	schoolPeriod: {
		id: "<STRING>",
		displayName: "<STRING>"
	},
	roleWithinClass: "<STRING>",
  substitute: "<BOOLEAN>",
	start: "<DATE>",
	end: "<DATE>"
},
...
]
\end{lstlisting}

\begin{longtable}{|p{0.2\textwidth}|p{0.2\textwidth}|p{0.58\textwidth}|}
		\caption{Beschreibung der Felder in einem JSON-Objekt aus der Liste der Klassen eines Benutzers}
\endfoot
		\caption{Beschreibung der Felder in einem JSON-Objekt aus der Liste der Klassen eines Benutzers}
		\label{tab:rest:api:users:id:classes:read:ret}
\endlastfoot 
\hline
			\textbf{Feldname} & \textbf{Datentyp} & \textbf{Beschreibung} \\ \hline
\endhead
id & STRING & ID der Klasse, mit der unter /api/classes/\$id weitere Informationen abgefragt werden können. \\ \hline
displayName & STRING & OPTIONAL; Anzeigename der Klasse \\ \hline
school & Object & Objekt mit Informationen zu Schule zu der die Klasse gehört \\ \hline
schoolPeriod & Object & Objekt mit Informationen zum Schuljahr, in welchem der Benutzer die Klasse besucht hat. Der Aufbau des Objektes kann \reftabllec{tab:rest:api:users:id:classes:schoolperiod} entnommen werden. \\ \hline
roleWithinClass & STRING & Rolle des Benutzers in der Klasse. Werte dafür können \reftabllec{tab:intro:rolesclass} entnommen werden. \\ \hline 
substitute & BOOLEAN & Gibt an, ob die Rolle in Vertretung wahrgenommen wird. \\ \hline
start & DATE & OPTIONAL; eine Datumsangabe nach ISO-8601 in dem Format YYYY-MM-DD. Gibt Datum an, ab dem der Benutzer in der Klasse war oder ist. \\ \hline
end & DATE & OPTIONAL; eine Datumsangabe nach ISO-8601 in dem Format YYYY-MM-DD. Gibt Datum an, bis zu dem der Benutzer in der Klasse war oder ist. \\ \hline
\end{longtable}


\begin{longtable}{|p{0.2\textwidth}|p{0.2\textwidth}|p{0.58\textwidth}|}
        \caption{Beschreibung der Felder im JSON-Objekt für die Schule }
\endfoot
        \caption{Beschreibung der Felder im JSON-Objekt für die Schule }
        \label{tab:rest:api:users:id:classes:school}
\endlastfoot 
\hline
            \textbf{Feldname} & \textbf{Datentyp} & \textbf{Beschreibung} \\ \hline
\endhead
id & STRING & ID der Schule, mit der unter /api/schools/\$id weitere Informationen abgefragt werden können. \\ \hline
displayName & STRING & OPTIONAL; Anzeigename der Schule \\ \hline
\end{longtable}

\begin{longtable}{|p{0.2\textwidth}|p{0.2\textwidth}|p{0.58\textwidth}|}
        \caption{Beschreibung der Felder im JSON-Objekt für das Schuljahr }
\endfoot
        \caption{Beschreibung der Felder im JSON-Objekt für das Schuljahr }
        \label{tab:rest:api:users:id:classes:schoolperiod}
\endlastfoot 
\hline
            \textbf{Feldname} & \textbf{Datentyp} & \textbf{Beschreibung} \\ \hline
\endhead
id & STRING & ID des Schuljahres, mit der unter /api/school-periods/\$id weitere Informationen abgefragt werden können. \\ \hline
displayName & STRING & OPTIONAL; Anzeigename des Schuljahres \\\hline
\end{longtable}

