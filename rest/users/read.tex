\subsubsection{READ}
\label{sec:rest:api:user:read}
Es sind nur Anfragen mit der HTTP-GET-Methode für ein READ auf die Daten zugelassen.
Der Pfad gibt für den eingeloggten Benutzer seine persönlichen Daten wieder.
Die Antwort erfolgt mit dem JSON-Objekt \reflisting{lst:code:rest:api:user:read:ret}. 
Die einzelnen Felder der Antwort werden in \reftabllec{tab:rest:api:user:read:ret} beschrieben.
\lstset{
  language=json,
  tabsize=2,
  captionpos=b,
  numbers=left,
  commentstyle=\color{green},
  backgroundcolor=\color{white},
  numberstyle=\color{gray},
  keywordstyle=\color{blue} \textbf,%otherkeywords={xdata},
  keywordstyle=[2]\color{red}\textbf,
  identifierstyle=\color{black},
  stringstyle=\color{red}\ttfamily,
  basicstyle = \ttfamily \color{black} \footnotesize,
  showstringspaces=false,
	breakatwhitespace=false,         % sets if automatic breaks should only happen at whitespace
  breaklines=true, 
}
\begin{lstlisting}[caption={JSON-Antwort für einen GET-Aufruf des Pfads /api/users},label={lst:code:rest:api:user:read:ret},frame=tlrb]
[
{
	id: "<STRING>",
	studentId: "<STRING>"
 	firstName: "<STRING>",
 	preferredName: "<STRING>",
 	lastName: "<STRING>",
 	dateOfBirth: "<DATE>",
 	sex: "<ENUM>",
 	memberships: [
 	{
 		school: {
 			id: "<STRING>",
 			displayName: "<STRING>"
 		},
 		role: "<STRING>",
 		start: "<DATE>",
 		end: "<DATE>"
 	},
 	...
 	],
 	classes: [
 	{
 		id: "<STRING>",
 		displayName: "<STRING>",
 		summary: "<STRING>",
 		school: {
 			id: "<STRING>",
 			displayName: "<STRING>"
 		},
 		schoolYear: {
 			id: "<STRING>",
 			displayName: "<STRING>"
 		},
 		roleWithinClass: "<STRING>",
 		start: "<DATE>",
 		end: "<DATE>"
 	},
 	...
	],
	subjects: [
	{
		id: "<STRING>",
		displayName: "<STRING>",
		schoolSubject: [
 		{
 			id: "<STRING>",
 			displayName: "<STRING>"
 		},
 		...
 		],
 		school: {
 			id: "<STRING>",
 			displayName: "<STRING>"
 		},
 		schoolYear: {
 			id: "<STRING>",
 			displayName: "<STRING>"
 		},
 		roleWithinSubject: "<STRING>",
 		start: "<DATE>",
 		end: "<DATE>" 		
	},
	...
	],
	guardians: [
	{
		id: "<STRING>",
		role: "<STRING>",
		user: {
			id: "<STRING>",
			displayName: "<STRING>",
		}
	},
	...
	]
}
]
\end{lstlisting}

\begin{longtable}{|p{0.2\textwidth}|p{0.2\textwidth}|p{0.58\textwidth}|}
		\caption{Beschreibung der Felder in einem JSON-Objekt von einem Benutzer}
\endfoot
		\caption{Beschreibung der Felder in einem JSON-Objekt von einem Benutzer}
		\label{tab:rest:api:user:read:ret}
\endlastfoot 
\hline
			\textbf{Feldname} & \textbf{Datentyp} & \textbf{Beschreibung} \\ \hline
\endhead
id & STRING & ID des Benutzers, mit der unter /api/users/\$id weitere Informtionen abgefragt werden können. \\ \hline
studentId & STRING & Optional; Landesweit gültige Schüler-Identifikationsnummer \\ \hline
name & STRING & Vorname des Benutzers \\ \hline
lastName & STRING & Nachname des Benutzers \\ \hline
dateOfBirth & DATE & Geburtstag des Benutzers \\ \hline
sex & ENUM & Geschlecht des Benutzers 
\begin{itemize}
	\item divers steht für divers
	\item female steht für weiblich
	\item male steht für männlich
\end{itemize}
 \\ \hline
roles & Object List & Liste von Rollen in Bezug auf eine Schule, der Aufbaue der Objekte kann aus \reftabllec{tab:rest:api:user:read:ret:roles} entnommen werden. \\ \hline
classes & Object List & Liste von Rollen in Bezug auf eine Klasse, der Aufbaue der Objekte kann aus \reftabllec{tab:rest:api:user:read:ret:classes} entnommen werden. \\ \hline
subjects & Object List & Liste von Unterrichtsfächern und den Rollen welcher der Benutzer in diesen Innen hatte, der Aufbaue der Objekte kann aus \reftabllec{tab:rest:api:user:read:ret:subjects} entnommen werden. \\ \hline
guardians & Object List & Liste von direkten Verbindungen zu einem anderen Benutzer, der Aufbaue der Objekte kann aus \reftabllec{tab:rest:api:user:read:ret:guardians} entnommen werden. \\ \hline
\end{longtable}

\begin{longtable}{|p{0.2\textwidth}|p{0.2\textwidth}|p{0.58\textwidth}|}
		\caption{Beschreibung der Felder in JSON-Objekt roles[]}
\endfoot
		\caption{Beschreibung der Felder in JSON-Objekt roles[]}
		\label{tab:rest:api:user:read:ret:roles}
\endlastfoot 
\hline
			\textbf{Feldname} & \textbf{Datentyp} & \textbf{Beschreibung} \\ \hline
\endhead
schoolId & STRING & ID der Schule \\ \hline
schoolName & STRING & Name der Schule \\ \hline
role & STRING & Role welche der User an der Schule innen hat, Werte dafür können aus \reftabllec{tab:intro:rolesschool} entnommen werden. \\ \hline
start: & DATE & OPTIONAL; ab wan der Benutzer die Rolle an der Schule innen hat \\ \hline
end & DATE & OPTIONAL; bis wan der Benutzer die Rolle an der Schule innen hatte \\ \hline
\end{longtable}

\begin{longtable}{|p{0.2\textwidth}|p{0.2\textwidth}|p{0.58\textwidth}|}
		\caption{Beschreibung der Felder in JSON-Objekt classes[]}
\endfoot
		\caption{Beschreibung der Felder in JSON-Objekt classes[]}
		\label{tab:rest:api:user:read:ret:classes}
\endlastfoot 
\hline
			\textbf{Feldname} & \textbf{Datentyp} & \textbf{Beschreibung} \\ \hline
\endhead
id & STRING & ID der Klasse, mit der unter /api/classes/\$id weiter Informatioen abgefragt werden können. \\ \hline
displayName & STRING & Anzeige name der Klasse \\ \hline
summary & STRING & Zusammfasende Beschreibung der Klasse \\ \hline
school & Object & Object mit Informationen zu Schule zu der die Klasse gehört \\ \hline
school.id & STRING & ID der Schule, mit der unter /api/schools/\$id weiter Informatioen abgefragt werden können. \\ \hline
school.displayName & STRING & Anzeigename der Schule \\ \hline
schoolYear & Object & Object mit Informationen zum Schuljahr in welche der Benutzer die Klasse besucht hat. \\ \hline
schoolYear.id & STRING & ID des Schuljahres, mit der unter /api/school-years/\$id weiter Informatioen abgefragt werden können. \\ \hline
schoolYear.displayName & STRING & Anzeigename des Schuljahres \\\hline
roleWithinClass & STRING & Rolle des Benutzers in der Klasse, Werte dafür können aus \reftabllec{tab:intro:rolesclass} entnommen werden. \\ \hline 
start & DATE & OPTIONAL; Ab wan der Benutzer in der Klasse war oder ist. \\ \hline
end & DATE & OPTIONAL; Bis wan der Benutzer in der Klasse war oder ist. \\ \hline
\end{longtable}

\begin{longtable}{|p{0.2\textwidth}|p{0.2\textwidth}|p{0.58\textwidth}|}
		\caption{Beschreibung der Felder in JSON-Objekt subjects[]}
\endfoot
		\caption{Beschreibung der Felder in JSON-Objekt subjects[]}
		\label{tab:rest:api:user:read:ret:subjects}
\endlastfoot 
\hline
			\textbf{Feldname} & \textbf{Datentyp} & \textbf{Beschreibung} \\ \hline
\endhead
id & STRING & ID des Unterrichtsfaches, mit der unter /api/subjects/\$id weiter Informatioen abgefragt werden können. \\ \hline
displayName & STRING & Anzegenamen des Unterrichtsfaches \\ \hline
schoolSubject & Array of Objects & Array mit Objecten zu referenz Schulfächer, der Aufbaue der Objekte kann aus \reftabllec{tab:rest:api:user:read:ret:subjects:schoolSubject} entnommen werden. \\ \hline
school & Object & Object der mit Informationen zu Schule wo was Unterrichtsfach angeboten wird. \\ \hline
school.id & STRING & ID der Schule, mit der unter /api/schools/\$id weiter Informatioen abgefragt werden können. \\ \hline
school.displayName & STRING & Anzeigename für die Schule \\ \hline
schoolYear & Object & Object mit Informationen zum Schuljahr in welche der Benutzer das Unterrichtsfach besucht hat. \\ \hline
schoolYear.id & STRING & ID des Schuljahres, mit der unter /api/school-years/\$id weiter Informatioen abgefragt werden können. \\ \hline
schoolYear.displayName & STRING & Anzeigename des Schuljahres \\\hline
roleWithinSubject & STRING & Rolle weleche der Benutzer in dem Unterrichtsfach hat, Werte dafür können aus \reftabllec{tab:intro:rolessubject} entnommen werden. \\ \hline
start & DATE & OPTIONAL; Ab wan der Benutzer das Unterrichtsfach hat.\\ \hline
end & DATE & OPTIONAL; Bis wan der Benutzer das Unterrichtsfach hat. \\ \hline
\end{longtable}

\begin{longtable}{|p{0.2\textwidth}|p{0.2\textwidth}|p{0.58\textwidth}|}
		\caption{Beschreibung der Felder in JSON-Objekt subjects[].schoolSubject[]}
\endfoot
		\caption{Beschreibung der Felder in JSON-Objekt subjects[].schoolSubject[]}
		\label{tab:rest:api:user:read:ret:subjects:schoolSubject}
\endlastfoot 
\hline
			\textbf{Feldname} & \textbf{Datentyp} & \textbf{Beschreibung} \\ \hline
\endhead
id & STRING & ID des Refrenzschulfaches, welches in der liste /api/school-subjects zufinden ist. \\ \hline
displayName & STRING & Anzeigename das Referenz Schulfaches \\ \hline
\end{longtable}

\begin{longtable}{|p{0.2\textwidth}|p{0.2\textwidth}|p{0.58\textwidth}|}
		\caption{Beschreibung der Felder in JSON-Objekt guardians[]}
\endfoot
		\caption{Beschreibung der Felder in JSON-Objekt guardians[]}
		\label{tab:rest:api:user:read:ret:guardians}
\endlastfoot 
\hline
			\textbf{Feldname} & \textbf{Datentyp} & \textbf{Beschreibung} \\ \hline
\endhead
id & STRING & ID des Eintrages \\ \hline
role & STRING & Angabe zu Beziehung zum anderen Benutzer, Werte dafür können aus \reftabllec{tab:intro:rolesuser} entnommen werden. \\ \hline
user & Object & Object welches die informationen des anderen Benutzers entählt. \\ \hline
user.id & STRING & ID des anderen Benutzers, mit der unter /api/users/\$id weitere Informtionen abgefragt werden können. \\ \hline
user.displayName & STRING & Name des anderen Benutzers \\ \hline
\end{longtable}