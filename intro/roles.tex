\section{Benutzerrollen}
\label{Benutzerrollen}
An dieser Stelle listen wir eine Reihe von bekannten Rollen in Bezug auf Entitäten im Kontext "'Schule"' auf, die auf IDM-Provider-Seite definiert und im Client für die Berechtigungssteuerung genutzt werden können.
\\


\begin{table}[htb]
	\begin{tabularx}{\textwidth}{|c|X|}
		\hline
\textbf{Rollenname} & \textbf{Beschreibung der Rolle} \\ \hline
guest & Gäste und nicht authentifizierte Benutzer \\ \hline
user & authentifizierte Benutzer \\ \hline
student & Schüler \\ \hline
external-student & Schüler von anderen Schulen, die nur einzelne Fächer oder Kurse besuchen \\ \hline
guardian & Eltern, Erziehungsberechtigte, Vormünder von Schülern \\ \hline
teacher & Lehrkräfte \\ \hline
external-expert & Externe Experten zur Unterstützung des Unterrichts (auch Schulassistenten und pädagische Mitarbeiter) \\ \hline
principal & Schulleitung \\ \hline
school-admin & Schul-Administrator \\ \hline
school-board & Schulträger \\ \hline
fed-school-board & Mitarbeiter/-in des Schulministeriums \\ \hline
alum & Ehemalige einer Schule \\ \hline
sync-system & Systeme, welchen ein Sync mit allen Daten erlaubt ist \\ \hline

	\end{tabularx}

		\caption{Mögliche Benutzerrollen in Bezug auf eine Schule}
		\label{tab:intro:rolesschool}
\end{table}

\begin{table}[htb]
	\begin{tabularx}{\textwidth}{|c|X|}
		\hline
\textbf{Rollenname} & \textbf{Beschreibung der Rolle} \\ \hline
student & Schüler \\ \hline
teacher & Lehrkräfte \\ \hline
class-representative & Klassensprecher \\ \hline
guardian-representative & Elternsprecher \\ \hline

	\end{tabularx}

		\caption{Mögliche Benutzerrollen in Bezug auf eine Klasse}
		\label{tab:intro:rolesclass}
\end{table}

\begin{table}[htb]
	\begin{tabularx}{\textwidth}{|c|X|}
		\hline
\textbf{Rollenname} & \textbf{Beschreibung der Rolle} \\ \hline
student & Schüler \\ \hline
teacher & Lehrkräfte \\ \hline
external-expert & Externe Experten zur Unterstützung des Unterrichts \\ \hline
	\end{tabularx}

		\caption{Mögliche Benutzerrollen in Bezug auf ein Unterrichtsfach}
		\label{tab:intro:rolessubject}
\end{table}

\begin{table}[htb]
	\begin{tabularx}{\textwidth}{|c|X|}
		\hline
\textbf{Rollenname} & \textbf{Beschreibung der Rolle} \\ \hline
guardian & Eltern, Erziehungsberechtigte, Vormünder von Schülern \\ \hline
resource-teacher & Förderlehrkräfte \\ \hline
	\end{tabularx}

		\caption{Mögliche direkte Verbindungen zu einem anderen Benutzer}
		\label{tab:intro:rolesuser}
\end{table}


Durch die Rollen auf die Entitäten Klasse, Unterrichtsfach und Benutzer lassen sich Berechtigungen im Kontext dieser besser steuern. 
Bei Bedarf können IDM-Provider und Client weitere Rollen festlegen, die additiv zu vergeben und nicht Teil dieser Konzeption sind.
