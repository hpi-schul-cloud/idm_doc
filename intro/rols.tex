\section{Benutzerrollen}
Um eine saubere Zugriffssteuerung auf die Daten gewährleisten zu können, definiert dieses Dokument zentral die möglichen Rollen der Akteure. 
Das korrekte Mapping aus dem jeweiligen IDM-System auf die im Dokument aufgelisteten Rollenbezeichnungen ist dabei vom IDM-Provider sicherzustellen.\\
\\
Im Nachfolgenden werden die Benutzerrollen des Systems aufgelistet. 
Die Auflistung der erlaubten und unerlaubten Aktionen erfolgt jeweils in der Definition der Endpunkte.


\begin{table}[htb]
	\begin{tabularx}{\textwidth}{|c|X|}
		\hline
\textbf{Rollenname} & \textbf{Beschreibung der Rolle} \\ \hline
guest & Gäste und nicht authentifizierte Benutzer \\ \hline
user & authentifizierte Benutzer \\ \hline
students & Schüler \\ \hline
external-students & Schüler von anderen Schulen, die nur einzelne Fächer oder Kurse besuchen \\ \hline
guardians & Eltern, Erziehungsberechtigte, Vormünder von Schülern \\ \hline
teacher & Lehrer \\ \hline
principal & Schulleitung \\ \hline
school-admin & Schul-Administrator \\ \hline
school-board & Schulträger \\ \hline
fed-school-board & Mitarbeiter/-in des Schulministeriums \\ \hline
sync-systems & Systeme, welchen ein Sync mit allen Daten erlaubt ist \\ \hline

	\end{tabularx}

		\caption{Benutzerrollen, die im zentralen IDM-System vorgesehen sind}
		\label{tab:intro:roles}
\end{table}