\section{Einschränkung der Daten}
\label{auth:limit_data}

Die Spezifikation der REST-API sieht vor, dass die Sichtbarkeit der Daten an den definierten Endpunkte bereits im JSON-Objekt gemäß der Berechtigung des anfragenden Benutzers berücksichtigt ist. 
Daher muss der \textit{ID Token} Informationen enthalten, über die der IdM-Provider den Benutzer eindeutig identifizieren kann.
Die Informationen aus dem ID Token reichen ggf. jedoch nicht aus, um die Sichtbarkeit der Daten korrekt zu beschränken, da einem Benutzer im IdM mehrere Schule-Rolle-Kombinationen zugewiesen sein können (z.B. Tätigkeit als Lehrkraft an verschiedenen Schulen oder sowohl Lehrkraft als auch Elternteil eines Schülers an derselben Schule). 
Daher muss die Information, mit welcher Kombination von Schule und Rolle innerhalb dieser Schule der Aufruf eines REST-API-Endpunkts durchgeführt wird, an den IdM-Provider übermittelt werden. 
Dies geschieht durch die Erweiterung der \textit{Scopes}-Liste um den Scope für die Schule und die Rolle als Parameterübergabe beim Autorisierungsvorgang. 
\reflisting{listing:authorization_request} zeigt einen exemplarischen Aufruf des \textit{Authorization Endpoints} mit Übergabe der \textit{Scopes}.
Die zusätzlichen \textit{Scopes} für Schule und Rolle sind somit im Access Token hinterlegt und können bei der Generierung des zurückzuliefernden JSON-Objekts verwendet werden.

\begin{lstlisting}[caption={Beispielhafter Aufruf des Authorization Endpoints},label={listing:authorization_request},frame=tlrb]
https://<URL zum Authorization Endpoint>?response_type=code
&client_id=<Identifier von Client-App>
&redirect_uri=<Redirect-URL>
&scope=openid teacher school
&state=<Undurchschaubarer Wert fuer Sicherheitszwecke>
\end{lstlisting}

Ein Sonderfall stellen dabei Benutzerkonten mit der Rolle "'sync-system"' dar. 
Diese haben bei Aufruf eines REST-API-Endpunkts grundsätzlich keine Beschränkung auf eine einzelne Schule. 
Daher muss beim Autosierungsvorgang auch kein \textit{Scope} für die Schule an den Autorisierungsserver übergeben werden. \\

Sofern für ein Benutzerkonto im IdM mehrere Schule-Rolle-Kombinationen hinterlegt sein können, müssen Schule und Rolle bei Start des Authentifizierungsprozesses gegebenenfalls durch den Benutzer wählbar sein. 
Für einen Schule-Rolle-Wechsel ist eine Neuanmeldung bzw. Aktualisierung des Access Tokens bezüglich der \textit{Scopes} für Schule und Rolle notwendig.