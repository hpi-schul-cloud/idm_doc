\lstset{
  language=json,
  tabsize=2,
  captionpos=b,
  numbers=left,
  commentstyle=\color{green},
  backgroundcolor=\color{white},
  numberstyle=\color{gray},
  keywordstyle=\color{blue} \textbf,%otherkeywords={xdata},
  keywordstyle=[2]\color{red}\textbf,
  identifierstyle=\color{black},
  stringstyle=\color{red}\ttfamily,
  basicstyle = \ttfamily \color{black} \footnotesize,
  showstringspaces=false,
	breakatwhitespace=false,         % sets if automatic breaks should only happen at whitespace
  breaklines=true, 
}
\begin{lstlisting}[caption={Klassen-Datenmodell
 Beispiel 2: Grundschulklasse},frame=tlrb]
{
 class: "KLASSE-01",
 name: "Klasse 1-A",
 school: "SCHULE-01",
 school-year: "SJ-09/10",
 start: "2009-09-01",
 end: "2010-08-31",
 grade: [ "1", ],
 subjects: [ "SUBJECT-0001", "SUBJECT-0003", "SUBJECT-0004", ],
 students: [
  { 
   user: "USER-01",
	}, { 
	 user: "USER-06",
	}, { 
	 user: "USER-07",
	 end: "2009-12-31",
	}, { 
	 user: "USER-10",
	}, { 
	 user: "USER-11",
	},
 ],
 teachers: [
  { 
   user: "USER-208",
	 order: [
	  {
		 order: "1",
		},
	 ],
	}, { 
	 user: "USER-209",
	 order: [
	  {
		 order: "2",
		},
	 ],
	},
 ],
 representatives: [
  {
	 user: "USER-10",
	 role: "student",
	 order: "1",	 
	}, {
	 user: "USER-07",
	 end: "2009-12-31",
	 role: "student",
	 order: "2",	 
	}, {
	 user: "USER-01",
	 start: "2010-01-04",
	 role: "student",
	 order: "2",	 
	}, {
	 user: "USER-114",
	 role: "guardian",
	 order: "1",	 
	}, {
	 user: "USER-115",
	 role: "guardian",
	 order: "2",	 
	},  
 ]
}
\end{lstlisting}
