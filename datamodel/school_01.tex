\lstset{
  language=json,
  tabsize=2,
  captionpos=b,
  numbers=left,
  commentstyle=\color{green},
  backgroundcolor=\color{white},
  numberstyle=\color{gray},
  keywordstyle=\color{blue} \textbf,%otherkeywords={xdata},
  keywordstyle=[2]\color{red}\textbf,
  identifierstyle=\color{black},
  stringstyle=\color{red}\ttfamily,
  basicstyle = \ttfamily \color{black} \footnotesize,
  showstringspaces=false,
	breakatwhitespace=false,         % sets if automatic breaks should only happen at whitespace
  breaklines=true, 
}
\begin{lstlisting}[caption={Schulen-Datenmodell Beispiel 1},frame=tlrb]
{
 school: "SCHULE-01",
 name: "Zobel Grundschule",
 classes: [ 
  {
   class: "KLASSE-0001",
	 start: "01-09-2020",
   end: "31-08-2021",
  }, {
   class: "KLASSE-0002",
	 start: "01-09-2010",
	 end: "31-08-2011",
  }, {
   class: "KLASSE-0003", 
	 start: "01-09-2010",
	 end: "31-08-2011",
  },
 ],
 users: [
  {
   user: "USER-02",
 	 role: "guardians",
	 start: "01-09-2009",
	 end: "31-08-2016",
	 school-years: ["SJ-09/10","SJ-10/11","SJ-11/12","SJ-13/14","SJ-14/15","SJ-15/16"],
  }, {
   user: "USER-01",
	 role: "students",
	 start: "01-09-2009",
	 end: "31-08-2016",
	 school-years: ["SJ-09/10","SJ-10/11","SJ-11/12","SJ-13/14","SJ-14/15","SJ-15/16"],
  }, {
   user: "USER-208",
	 role: "teacher",
	 start: "01-09-2001",
  }, {
   user: "USER-209",
   role: "teacher",
 	 start: "01-09-2007",
  },
 ],
 subjects: [
  {
   subject: "SUBJECT-0001",
   start: "01-09-2009",
   end: "28-02-2010",
  }, {
   subject: "SUBJECT-0101",
   start: "01-09-2009",
   end: "28-02-2010",
  }, 
 ],
}
\end{lstlisting}