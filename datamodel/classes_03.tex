\lstset{
  language=json,
  tabsize=2,
  captionpos=b,
  numbers=left,
  commentstyle=\color{green},
  backgroundcolor=\color{white},
  numberstyle=\color{gray},
  keywordstyle=\color{blue} \textbf,%otherkeywords={xdata},
  keywordstyle=[2]\color{red}\textbf,
  identifierstyle=\color{black},
  stringstyle=\color{red}\ttfamily,
  basicstyle = \ttfamily \color{black} \footnotesize,
  showstringspaces=false,
	breakatwhitespace=false,         % sets if automatic breaks should only happen at whitespace
  breaklines=true, 
}
\begin{lstlisting}[caption={Klassen-Datenmodell Beispiel 3: Jahrgangsübergreifende Klasse},frame=tlrb]
{
 class: "KLASSE-21",
 name: "Klasse 3,4 A",
 school: "SCHULE-07",
 school-year: "SJ-15/16",
 start: "2015-09-01",
 end: "2016-08-31",
 grade: [ "3", "4", ],
 subjects: [ "SUBJECT-0101", "SUBJECT-0102", "SUBJECT-0103", ],
 students: [
  { 
   user: "USER-51",
	}, { 
	 user: "USER-52",
	}, { 
	 user: "USER-53",
	}, { 
	 user: "USER-55",
	}, { 
	 user: "USER-54",
	 start: "2016-03-01",
	},
 ],
 teachers: [
  { 
   user: "USER-256",
	 order: [
	  {
		 order: "1",
		},
	 ],
	}, { 
	 user: "USER-257",
	 order: [
	  {
		 order: "1",
		},
	 ],
	}, { 
	 user: "USER-258",
	 order: [
	  {
		 order: "2",
		},
	 ],
	},
 ],
 representatives: [
  {
	 user: "USER-55",
	 role: "student",
	 order: "1",	 
	}, {
	 user: "USER-52",
	 role: "student",
	 order: "2",	 
	}, {
	 user: "USER-158",
	 start: "2015-09-10",
	 role: "guardian",
	 order: "1",	 
	}, {
	 user: "USER-159",
	 start: "2015-09-10",
	 role: "guardian",
	 order: "2",	 
	},  
 ]
}
\end{lstlisting}
