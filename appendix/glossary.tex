\section{Glossar}
\label{glossary}

\subsection{Begriffe der Fachdomäne}

\subsubsection*{Schüler}
Unterrichtsteilnehmer an einer Schule.

\subsubsection*{Externe Schüler}
Schüler, die einer anderen Schule angehören und bei einer anderen Schule schulübergreifende Unterrichtsangebote wahrnehmen.

\subsubsection*{Eltern}
Die für die Person der minderjährigen Schülerin oder des minderjährigen Schülers einzeln oder gemeinsam Sorgeberechtigten oder ihnen nach diesem Gesetz gleichgestellte Personen.

\subsubsection*{Lehrkräfte}
Lehrkräfte und pädagogische Fachkräfte an Schulen, die Unterricht erteilen.

\subsubsection*{Schulleitung}
Lehrkraft einer Schule sowie ggf. weitere Vertreter aus dem Lehrerkollegium.

\subsubsection*{Schulen}
Einrichtungen, in denen unabhängig vom Wechsel der Lehrkräfte, Schülerinnen und Schüler durch planmäßiges und gemeinsames Lernen und durch das gemeinsame Schulleben bestimmte Erziehungs- und Bildungsziele erreicht werden sollen.

\subsubsection*{Schulträger}
Eine juristische oder natürliche Person, die für die Errichtung, Organisation und Verwaltung der einzelnen Schule rechtlich unmittelbar die Verantwortung trägt und zur Unterhaltung der Schule eigene Leistungen erbringt.

\subsubsection*{Schuljahr}
Beschreibt einen Zeitraum in dem Klassenstufen und Klassen für gewöhnlich definiert sind.

\subsubsection*{Klasse}
Lerngemeinschaft aus Schülern, welche für einen Zeitraum (z.B. Schuljahr 2020/21) an einer Schule unter Betreuung von Lehrkräften am Unterricht im Klassenverband teilnimmt. Eine Klassenstufe umfasst mehrere Klassen einer Stufe, z.B. (5a, 5b, 5c im Schuljahr 2020/21).

\subsubsection*{Unterrichtsfach}
In unterschiedlichen Fächern werden Unterrichtsinhalte den Schülern durch eine oder mehrere Lehrkräfte zu bestimmten Zeiten in einem bestimmten Zeitraum vermittelt.  

\subsection{Technische Begriffe}

\subsubsection*{REST-API}

Technische Schnittstelle zum Austausch von Daten über HTTP-Methoden.

\subsubsection*{Endpunkt}

URL-Pfad innerhalb einer REST-API, unter dem bestimmte Daten abgerufen können.

\subsubsection*{Authentifizierung}

Zur Nachvollziehbarkeit des Zugriffs und generischem Zugriffsschutz sind Endpunkte durch technische Konzepte geschützt, die den öffentlichen Zugriff beschränken und bei API-Zugriff die Zuordnung eines (technischen) Nutzers ermöglichen.
 
\subsubsection*{Authorisierung}

Zum Schutz der über einen Endpunkt abrufbaren Daten kann die Berechtigung auf und Sichtbarkeit von Datensätzen je nach Rolle eines authentifiziertem Clients eingeschränkt werden.

\subsubsection*{IdM}
Kurz für Identitätsmanagement. Wird in diesem Dokument auch abkürzend für das System verwendet, in dem das Identitätsmanagement betrieben wird.

\subsubsection*{Client}

Eine Anwendung, welche über die REST-API lesend auf Daten des IdM zugreift oder diese verändert.

\subsubsection*{JavaScript Object Notation (JSON)}

Proprietäres Datenformat zum Austausch von Daten über teilweise menschenlesbare Textform.
